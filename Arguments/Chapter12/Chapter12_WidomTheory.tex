 \documentclass[../../Main/Main.tex]{subfiles}

\begin{document}

\chapter{Widom's scaling theory. Block-spin Kadanoff's transformation}

\section{Introduction}
We have seen  that as a given system approaches a critical point \( T \rightarrow T_c^{\pm} \), the distance \( \xi  \)  over which the fluctuations of the order parameter are correlated becomes comparable to the size of the whole system \( L \) and the microscopic aspects of the system become irrelevant. This means that near a critical point the system has no longer characteristic lengths (\( a,L \)), besides \( \xi  \) of course that becomes the only relevant length scale of the problem. We can therefore expect that if we "move" a little bit from a critical point (\( t \sim 0 \)), for example changing the temperature by a small amount, the free energy of the system as a function will not change its shape, hence it is invariant in form by a change of scale.

 This hypothesis is also suggested by experimental data such as the ones shown by Guggenheim for the gas phase diagrams and the ones shown for ferromagnetic materials at different temperatures. Let us consider the experiment in Figure \ref{fig:19_2}; we can see that data from different temperatures, if scaled properly, collapse into two (one for \( t<0 \) and one for \( t>0 \)) unique curves. It is clearly illustrated in Figure \ref{fig:19_1}. At the origin, the Widom's static scaling theory was introduced also to explain this collapse.

 \begin{figure}[h!]
 \centering
 \includegraphics[width=0.55\textwidth]{./img/1.pdf}
 \caption{\label{fig:19_1} Scaled magnetization \( m \) is plotted against scaled magnetic field \( h \).}
 \end{figure}

 \begin{figure}[h!]
 \centering
 \includegraphics[width=0.7\textwidth]{./img/exp.jpg}
 \caption{\label{fig:19_2} Scaled magnetic field \( h \) is plotted against scaled magnetization \( m \) for the insulating ferromagnet \( \text{CrBr}_3 \), using data from seven supercritical \( (T>T_c) \) and from eleven subcritical \( T<T_c \) isotherms. Here \( \sigma \equiv M/M_0 \). (1969) \cite{3_lesson_3}}
 \end{figure}




\section{Widom's static scaling theory}
We have seen that when a phase transition occurs the free energy of the system is such that the response functions exhibit singularities, often in the form of divergences. To make a concrete example (but of course all our statements are completely general) if we consider a magnetic system we can suppose to write its free energy density as:
\begin{equation*}
  f(T,H) = f_r (T,H) + f_s (T,H)
\end{equation*}
where \( t = (T-T_c)/T_c \) and \( h = (H-H_c)/k_B T \), \( f_r \)  is the "regular" part of the free energy (which does not significantly change near a critical point, it is an analytic function), while \( f_s \) is the "singular" one, which contains the non-analytic behaviour of the system near a critical point (i.e. \( t \sim 0 \) and \( h \sim 0 \)).

Widom's static scaling hypothesis consists in assuming that the singular part \( f_s \) of the free energy is a generalized homogeneous function, i.e.:
\begin{equation*}
  f_s (\lambda ^{p_1}t, \lambda ^{p_2} h) = \lambda f _s (t,h), \quad \forall \lambda \in \R
\end{equation*}
Note that assuming that one thermodynamic potential is a generalized homogeneous function implies that all the other thermodynamic potentials are so.

Therefore, in order to properly define the scaling hypothesis, we should rely on the mathematical concept of homogeneous functions and now we discuss the main properties of such functions.



\subsection{Homogeneous functions of one or more variables}

\subsubsection{Single variable}
Let us begin with the definition of homogeneous function for a single variable \( r \).
\begin{definition}{Homogeneous function}{}
A function \( f(r) \) is said to be homogeneous in \( r \) if
\begin{equation}
  f (\lambda r) = g (\lambda ) f(r), \qquad \forall \lambda \in \R
\end{equation}
where \( g \) is, for the moment, an unspecified function (we will shortly see that it has a precise form).
\end{definition}
\begin{example}{Parabola \( \pmb{ f(r) = B r^2} \) }{}
  An example of homogeneous function is
\begin{equation*}
  f(r) = B r^2
\end{equation*}
in fact
\begin{equation*}
  f ( \lambda r) = B ( \lambda r)^2 = \lambda ^2 f (r)
\end{equation*}
and so in this case \( g ( \lambda ) = \lambda ^2 \).
\end{example}

A very interesting property of an homogeneous functions is that,  once its value in a point \( r_0 \) (i.e. \( f(r_0) \)) and the function \( g(\lambda ) \) are known,
the entire \( f(r) \) can be reconstructed for all \( r \in \R \); indeed, any \( r \) can be written in the form \( r=\lambda r_0 \) (of course with \( \lambda = r / r_0 \)), so that
\begin{equation}
  f(r) = f (\lambda r_0) = g(\lambda ) f(r_0)
\end{equation}
We now want to show that \( g (\lambda ) \)  has a precise form.


\begin{theorem}{}{}
  The function \( g(\lambda ) \) is not arbitrary, but it must be of the form
\begin{equation}
  g(\lambda ) = \lambda ^p
\end{equation}
where \( p \) is the degree of the homogeneity of the function.
\end{theorem}

\begin{proof}
  From the definition of homogeneous function, for \( \lambda, \mu \in \R  \) we have on one hand that:
\begin{equation*}
  f(\lambda \mu r) = f(\lambda (\mu r)) = g (\lambda ) f(\mu r) = g(\lambda ) g(\mu ) f(r)
\end{equation*}
on the other hand,
\begin{equation*}
  f((\lambda \mu )r) = g (\lambda  \mu ) f(r)
\end{equation*}
and so:
\begin{equation*}
  g (\lambda \mu ) = g(\lambda ) g (\mu )
\end{equation*}
If we now suppose \( g \) to be differentiable\footnote{Actually \( g(\lambda ) \) continuous is sufficient, but proof becomes more complicated.}, then differentiating with respect to \( \mu  \)  this last equation we get:
\begin{equation*}
  \pdv{}{\mu } \qty[g(\lambda \mu )] = \pdv{}{\mu } \qty[ g(\lambda )g(\mu  )] \quad
  \Rightarrow \lambda g' (\lambda \mu ) = g (\lambda ) g' (\mu )
\end{equation*}
Setting \( \mu =1 \) and defining \( p \equiv g'( \mu =1 ) \), we have:
\begin{equation*}
   \lambda g'(\lambda ) = g(\lambda ) p \quad \Rightarrow \frac{g'(\lambda )}{g(\lambda )} = \frac{p}{\lambda }
\end{equation*}
which yields:
\begin{equation*}
  \dv{}{\lambda } \qty(\ln{g(\lambda )} ) = \frac{p}{\lambda }   \quad \Rightarrow \ln{g(\lambda )} = p \ln{\lambda } + c \quad \Rightarrow g (\lambda ) = e^c \lambda ^p
\end{equation*}
Now, \( g'(\lambda ) = p e^c \lambda ^{p-1} \), so since \( g'(1) = p \) by definition we have  \( p = p e^c \) and thus \( c=0 \).
Therefore:
\begin{equation*}
  g(\lambda ) = \lambda ^ p
\end{equation*}
A homogeneous function such that \(   g(\lambda ) = \lambda ^ p \)  is said to be homogeneous of degree \( p \).
\end{proof}

\subsubsection{Generalized homogeneous functions (more variables)}
Let us now define homogeneous functions for more than only one variable:
\begin{equation*}
  f(\lambda x, \lambda y) = \lambda ^p f(x,y), \quad \forall \lambda \in \R
\end{equation*}
 The function  \( f(x,y) \) is a \emph{generalized homogeneous function} if has as more general form
\begin{empheq}[box=\myyellowbox]{equation}
  f (\lambda ^a x, \lambda ^b y) = \lambda f(x,y), \quad \forall \lambda \in \R
  \label{eq:19_2}
\end{empheq}


\begin{remark}
  If we consider instead
  \begin{equation*}
      f (\lambda ^a x, \lambda ^b y) = \lambda^p f(x,y)
  \end{equation*}
  we can always choose \( \lambda ^p \equiv s \) such that
  \begin{equation*}
    f ( s^{a/p} x, s^{b/p} y) = s f(x,y)
  \end{equation*}
  and choosing \( a'=a/p \) and \( b'=b/p \) we are back to \eqref{eq:19_2}. Hence, it is the most general form an homogeneous function can have.
\end{remark}


\begin{remark}
Since \( \lambda  \) is arbitrary, we can choose \( \lambda  = y ^{-1/b} \), thus we get
\begin{equation*}
  f(x,y) = y^{1/b} f \qty( \frac{x}{y^{a/b}},1)
\end{equation*}
in that way \( f \) depends on \( x \) and \( y \) only through the ratio \( \frac{x}{y^{a/b}} \)! Similarly, for \( x \), one can choose \( \lambda = x^{-1/a} \), obtaining
\begin{equation*}
  f(x,y) = x^{1/a} f \qty(1, \frac{y}{x^{b/a}})
\end{equation*}

\end{remark}
\begin{example}{}{}
The function \( f(x,y) = x^3 + y^7 \) is an homogeneous one. Indeed, we have:
\begin{equation*}
  f(\lambda ^{1/3} x, \lambda ^{1/7}y) = \lambda x^3 + \lambda y^7 = \lambda f (x,y)
\end{equation*}
Instead, examples of non-homogeneous functions are:
\begin{equation*}
  f(x) = e^{-x}, \qquad f(x) = \log{x}
\end{equation*}

\end{example}





\subsection{Widom's scaling hypothesis}

As said, the Widom's static scaling hypothesis consists in assuming that the singular part of the free energy, \( f_s \), is a generalized homogeneous function, i.e.:
\begin{empheq}[box=\myyellowbox]{equation}
  f_s ( \lambda ^{p_1} t, \lambda ^{p_2} h) = \lambda f_s (t,h), \quad \forall \lambda \in \R
  \label{eq:19_12}
\end{empheq}
where \( p_1 \) and \( p_2 \) are the degrees of the homogeneity.

The exponents \( p_1 \)  and \( p_2 \) are not specified by the scaling hypothesis; however, we are shortly going to show that all the critical exponents of a system can be expressed in terms of \( p_1 \)  and \( p_2 \); this also implies that if two critical exponents are known, we can write \( p_1 \) and \( p_2 \) in terms of them (since in general we will have a set of two independent equations in the two variables \( p_1 \) and \( p_2 \)  and therefore determine all the critical exponents of the system. In other words, we just need to know two critical exponents to obtain all the others.

\begin{remark}
Since \( f_s \) is a generalized homogeneous function, it is always possible to choose \( \lambda  \) to remove the dependence on one of their arguments;
for example, one can choose \(   \lambda = h^{-1/p_2} \) to obtain
\begin{equation*}
 f_s (t,h) = h^{1/p_2} f_s (h^{-p_1/p_2} t , 1)
\end{equation*}
where
\begin{equation*}
  \Delta  \equiv \frac{p_2}{p_1}
\end{equation*}
is called the \emph{gap exponent}.
\end{remark}



\section{Relations between critical exponents}
Let us now explore the consequences of Widom's assumption on the critical exponents of a system, again on a magnetic one for concreteness. Indeed, let us see how this simple hypothesis allow us, by simple differential calculus, to obtain relations between the thermodynamic critical exponents.

\subsection{Exponent \( \pmb{\beta } \) (scaling of the magnetization)}
Let us start from the scaling hypothesis
\begin{equation*}
  f_s (\lambda ^{p_1}t, \lambda ^{p_2}h) = \lambda f_s (t,h)
\end{equation*}
Since
\begin{equation*}
  M = \pdv{f}{H}
\end{equation*}
deriving both sides of Widom's assumption with respect to \( h \)\footnote{We should in principle derive with respect to \( H \), but since \( h \propto \beta H \), the  \( \beta  \) factors simplify on both sides.} we get:
\begin{equation*}
  \lambda ^{p_2} \pdv{f_s}{h} \qty(\lambda ^{p_1}t, \lambda ^{p_2}h) = \lambda \pdv{f_s}{h} (t,h)
\end{equation*}
and thus:
\begin{equation*}
   \lambda ^{p_2} M_s ( \lambda ^{p_1} t, \lambda ^{p_2} h) = \lambda M_s (t,h)
\end{equation*}
On the other hand, we know that, for \( h=0 \) and \( t \rightarrow 0^- \), \( M_s (t) \sim (-t)^{\beta } \). Hence, in order to determine \( \beta  \), we set \( h=0 \) so that this becomes
\begin{equation*}
  M_s (t,0) = \lambda ^{p_2 -1} M_s ( \lambda ^{p_1} t,0)
\end{equation*}
Since \( \lambda  \) is arbitrary, using the properties of generalized homogeneous functions, we set
\begin{equation*}
  \lambda ^{p_1} t = -1 \quad \Rightarrow \lambda = (-t)^{-1/p_1}
\end{equation*}
to eliminate the dependence on \( t \). Hence, we get
\begin{equation*}
  M_s (t,0) =  (-t)^{(1-p_2)/p_1} M_s (-1,0)
\end{equation*}
By definition of the \( \beta  \) critical exponent, we have:
\begin{equation}
  \beta = \frac{1-p_2}{p_1}
  \label{eq:19_10}
\end{equation}


\subsection{Exponent \( \pmb{\delta}  \)}
Let us consider again the relation
\begin{equation*}
   \lambda ^{p_2} M_s ( \lambda ^{p_1} t, \lambda ^{p_2} h) = \lambda M_s (t,h)
\end{equation*}
We can determine the exponent \( \delta  \) by setting \( t=0 \) (\( T=T_c  \)), obtaining:
\begin{equation*}
  M(0,h) = \lambda ^{p_2-1} M (0, \lambda ^{p_2} h)
\end{equation*}
Now, using again the same property of generalized homogeneous functions we set
\begin{equation*}
  \lambda ^{p_2} h = 1 \quad \Rightarrow \lambda = h^{-1/p_2}
\end{equation*}
and we get:
\begin{equation*}
  M_s (0,h) =  h^{(1-p_2)/p_2} M_s (0,1)
\end{equation*}
Since \(  M_s  \overset{h \rightarrow 0^+}{\sim} h^{1/\delta } \), we have:
\begin{equation}
  \delta = \frac{p_2}{ 1 - p_2 }
  \label{eq:19_11}
\end{equation}

Now we can also express \( p_1 \) and \( p_2 \)  in terms of \( \beta  \) and \( \delta  \) from the two relations Eq.\eqref{eq:19_10} and Eq.\eqref{eq:19_11}. The result is:
\begin{equation}
  p_1 = \frac{1}{\beta (\delta +1)}, \qquad p_2 = \frac{\delta }{\delta + 1}
\end{equation}
from which we see that the gap exponent is:
\begin{equation}
  \Delta \equiv \frac{p_2}{p_1}  = \beta \delta
\end{equation}


\subsection{Exponent \( \pmb{\gamma  } \) }
In order to obtain the magnetic susceptibility, we derive twice the expression of Widom's assumption with respect to \( h \), to get:
\begin{equation*}
  \lambda ^{2p_2} \chi _T \qty(\lambda ^{p_1}t, \lambda ^{p_2}h) = \lambda \chi _T (t,h)
\end{equation*}

The exponent \( \gamma   \) describes the behaviour of \( \chi _T \)  for \( t \rightarrow 0 \)  when no external field is present (\( h=0 \)). What we can now see is that the scaling hypothesis leads to the equality of the exponents for \( t \rightarrow 0^+ \)  and \( t \rightarrow 0^- \).

\begin{itemize}
\item Case \( t \rightarrow 0^- \): setting \( h=0 \)  and \( \lambda = (-t)^{-1/p_1} \) we get
\begin{equation*}
  \chi _T (t,0) = \qty(-t)^{-\frac{2p_2-1}{p_1}} \chi _T (-1,0)
\end{equation*}
and if we call \( \gamma ^-  \)  the critical exponent for \( t \rightarrow 0^- \), we see that, since
\begin{equation*}
  \chi _T (t,0) \overset{t \rightarrow 0^-}{\sim } \qty(-t)^{-\gamma_-  }
\end{equation*}
we get
\begin{equation*}
  \gamma _- = \frac{2p_2 -1}{p_1} = \beta (\delta -1)
\end{equation*}

\item Case \( t \rightarrow 0^+ \): setting \( h=0 \)  and \( \lambda = (t)^{-1/p_1} \) we get
\begin{equation*}
  \chi _T (t,0) = t^{-\frac{2p_2-1}{p_1}} \chi _T (1,0)
\end{equation*}
and if we call \( \gamma ^+  \)  the critical exponent for \( t \rightarrow 0^+ \), we see that, since
\begin{equation*}
  \chi _T (t,0) \overset{t \rightarrow 0^+}{\sim } t^{-\gamma_+  }
\end{equation*}
we get
\begin{equation*}
  \gamma _+ = \frac{2p_2 -1}{p_1} = \beta (\delta -1)
\end{equation*}
\end{itemize}
We therefore see explicitly that:
\begin{equation}
  \gamma _- = \gamma _+ \equiv \gamma  = \frac{2p_2 -1}{p_1} = \beta (\delta -1)
\end{equation}





\subsection{Exponent \( \pmb{\alpha } \) (scaling of the specific heat)}
In order to determine the behaviour of the specific heat (at constant external field) near the critical point, we derive the expression of Widom's assumption twice with respect to the temperature \( t \), so that:
\begin{equation*}
  \lambda ^{2p_1}c_H \qty(\lambda ^{p_1}t, \lambda ^{p_2}h) = \lambda c_H (t,h)
\end{equation*}
We want to see again that the scaling hypothesis leads to the equality of the exponents for \( t \rightarrow 0^+ \)  and \( t \rightarrow 0^- \).

\begin{itemize}
\item Case \( t \rightarrow 0^- \): setting \( h=0 \) and \( \lambda = (-t)^{-1/p_1} \) we get
\begin{equation*}
  c_H (t,0) = \qty(-t)^{-\qty(2- \frac{1}{p_1}) } c_H (-1,0)
\end{equation*}
and if we call \( \alpha ^-  \)  the critical exponent for \( t \rightarrow 0^- \), we see that, since
\begin{equation*}
  c_H (t,0) \overset{t \rightarrow 0^-}{\sim } (-t)^{-\alpha _-}
\end{equation*}
we get
\begin{equation*}
  \alpha _- = 2 - \frac{1}{p_1}
\end{equation*}

\item Case \( t \rightarrow 0^+ \): setting \( h=0 \) and \( \lambda = (t)^{-1/p_1} \) we get
\begin{equation*}
  c_H (t,0) = t^{-\qty(2- \frac{1}{p_1}) } c_H (1,0)
\end{equation*}
and if we call \( \alpha ^+  \)  the critical exponent for \( t \rightarrow 0^+ \), we see that, since
\begin{equation*}
  c_H (t,0) \overset{t \rightarrow 0^-}{\sim } t^{-\alpha _+}
\end{equation*}
we get
\begin{equation*}
  \alpha _+ = 2 - \frac{1}{p_1}
\end{equation*}

\end{itemize}

We have again:
\begin{equation}
  \alpha  _- = \alpha  _+ \equiv \alpha  = 2 - \frac{1}{p_1}
  \label{eq:19_6}
\end{equation}




\subsection{Griffiths and Rushbrooke's equalities}
\label{sec:19_1}
If we now substitute \( p_1 = \frac{1}{\beta (\delta +1)} \) into \( \alpha  =  2 - \frac{1}{p_1} \), we get:
\begin{empheq}[box=\myyellowbox]{equation}
  \alpha + \beta (\delta +1) = 2
\end{empheq}
This is the \emph{Griffiths equality}, which we have already encountered in inequalities between critical exponents as an inequality (see Sec.\ref{sec:3_2}).

On the other hand, \emph{Rushbrooke's equality} is obtained by combining Griffith equality with the relation \( \gamma = \beta (\delta -1)  \):
\begin{empheq}[box=\myyellowbox]{equation}
\alpha + 2 \beta + \gamma = 2
\end{empheq}
We therefore see, as anticipated in Sec.\ref{sec:3_2}, that the static scaling hypothesis allows to show that they are indeed exact equalities.




\subsection{An alternative expression for the scaling hypothesis}
We can re-express Widom's assumption in another fashion often used in literature. Let us consider the Widom's assumption
\begin{equation*}
  f_s ( \lambda^{p_1} t, \lambda ^{p_2}h) = \lambda f_s (t,h)
\end{equation*}
If we set \( \lambda = t ^{-1/p_1} \), then:
\begin{equation*}
  f_s (1,t^{-p_2/p_1}h) = t^{-1/p_1} f_s (t,h)
\end{equation*}
From \( \Delta = \frac{p_2}{p_1} \) and \( \alpha = 2 - \frac{1}{p_1} \), we can rewrite this as:
\begin{equation}
  f_s (t,h) = t^{2- \alpha } f_s \qty(1, \frac{h}{t^{\Delta }})
  \label{eq:19_13}
\end{equation}
which is the most used form of the scaling hypothesis in statistical mechanics.

As we can notice, we have not considered the critical exponents \( \eta  \)  and \( \nu  \); this will be done shortly in Kadanoff's scaling and correlation lengths.

\subsection{Scaling of the equation of state}
Besides the relations between critical exponents, Widom's static scaling theory allows us to make predictions on the shape of the state equation of a given system.
By predicting the scaling form of the equation of state, we can explain the collapse of the experimental data. Let us now see how, again for a magnetic system. We start from the relation
\begin{equation*}
  M_s (t,h) = \lambda ^{p_2 - 1} M_s (\lambda ^{p_1} t, \lambda ^{p_2} h)
\end{equation*}
Using the property of generalized homogeneous functions we set \( \lambda = \abs{t}^{-1/p_1}  \). Hence,
\begin{equation*}
  M_s (t,h) = \abs{t}^{\frac{1-p_2}{p_1}} M_s ( \frac{t}{\abs{t} }, \frac{h}{\abs{t}^{p_2/p_1 } })
\end{equation*}
Since \( \beta = (1-p_2)/p_1 \) and \( \Delta = p_2/p_1 \), we have
\begin{equation}
  \frac{M_s (t,h)}{\abs{t}^{\beta } } = M_s \left( \frac{t}{\abs{t} }, \frac{h}{\abs{t}^{\Delta } }\right)
  \label{eq:19_5}
\end{equation}
Hence, we can define the \emph{scaled magnetization} and \emph{scaled magnetic field}  as
\begin{equation}
    \bar{m} \equiv \abs{t}^{-\beta } M(t,h), \qquad \bar{h} \equiv \abs{t}^{-\Delta } h (t,M)
\end{equation}
and
\begin{equation}
  F_{\pm} ( \bar{h} ) \equiv M_s (\pm1, \bar{h} )
\end{equation}
where \( +1 \)  corresponds to \( t>0 \)  (namely \( T>T_c \)) and \( -1 \)  to \( t<0 \)  (i.e. \( T<T_c \)).
Using these definitions, Eq.\eqref{eq:19_5} becomes
\begin{equation}
  \bar{m} = F_{\pm} (\bar{h} )
\end{equation}
The meaning of this equation is that if we measure \( M \) and \( h \) and rescale them as we have just seen, all the experimental data should fall on the same curve independently of the temperature \( T \); there are of course two possible curves (not necessarily equal), one for \( T>T_c \)  and one for \( T<T_c \)  (which correspond to \( M(1,h) \) and \( M(-1,h) \)). These predictions are in perfect agreement with experimental results shown in Figure \ref{fig:19_2}, and are one of the greatest successes of Widom's static scaling theory.






\section{Kadanoff's block spin and scaling of the correlation function}
As we have seen, Widom's static scaling theory allows us to determine exact relations between critical exponents, and to interpret the scaling properties of systems near a critical point. However, this theory is based upon the following equation:
\begin{equation*}
  f (T,H) = f_r (T,H) + f_s(T,H)
\end{equation*}
but gives no physical interpretation of it; in other words, it does not tell anything about the physical origin of scaling laws. Furthermore, as we have noticed Widom's theory does not involve correlation lengths, so it tells nothing about the critical exponents \( \nu  \)  and \( \eta  \).


We know that one of the characteristic traits of critical phenomena is the divergence of the correlation length \( \xi  \), which becomes the only physically relevant length near a critical point. However, by now we are unable to tell if and how this is related to Widom's scaling hypothesis; everything will become more clear within the framework of the Renormalization Group, in which we will see that Widom's assumption is a consequence of the divergence of correlation length.

Nonetheless, before the introduction of the Renormalization Group, Kadanoff (1966) proposed a plausibility argument for his assumption applied to the Ising model, which we are now going to analyse. We will see that Kadanoff's argument, which is based upon the intuition that \emph{the divergence of \( \xi  \)  implies a relation between the coupling constants of an effective Hamiltonian \( \mathcal{H}_{eff} \) and the length on which the order parameter \( m \) is defined}, is correct in principle but not in detail because these relations are in reality more complex than what predicted by Kadanoff; furthermore, Kadanoff's argument does not allow an explicit computation of critical exponents. We will have to wait for the Renormalization Group in order to solve these problems.


\subsection{Kadanoff's argument for the Ising model}
Let us consider a \( d \)-dimensional  Ising model with hypercubic lattice with lattice constant \( a \); assuming nearest-neighbour interactions the Hamiltonian of the system will be:
\begin{equation}
  - \beta \mathcal{H}_{\Omega } = K \sum_{\expval{ij}}^{N} \sigma _i \sigma _j + h \sum_{i=1}^{N} \sigma _i
\end{equation}
where \( \sigma _i = \pm 1 \), \( K=\beta J \) and \( h=\beta H \), as usual.

The Kadanoff's argument is based on a coarse-grained operation on the system and on two basic assumptions.

\subsubsection{Coarse graining operation}
Since the values of the spin variables are correlated on lengths of order \( r < \xi (t) \), we partition the system into blocks of size \( la \) (\( l \) is an adimensional scale) such that
\begin{equation*}
a \ll l a \ll \xi (t)
\end{equation*}
The spins contained in this regions of linear dimension \( la \) will behave, statistically, as a single unit.
We can therefore imagine to carry out, similarly to what we have seen for the Ginzburg-Landau theory, a coarse graining procedure were we substitute the spins \( \sigma _i \)  inside a "block" of linear dimension \( la \)  (which will therefore contain \( l^d \)  spins) with a single \emph{block spin} (or \emph{superspin}) \( S_I \); the total number of blocks will of course be:
\begin{equation*}
  N_l = \frac{N}{l^d}
\end{equation*}
 Considering the \( I \)-th block, we can define the block spin \( S_I \) as:
\begin{equation}
  S_I \equiv \frac{1}{\abs{m_l} } \frac{1}{l^d} \sum_{i \in I}^{}  \sigma _i
  \label{eq:19_7}
\end{equation}
where the mean magnetization of the \( I \)-th block \( m_l \) is:
\begin{equation}
  m_l \equiv \frac{1}{l^d} \sum_{i \in I}^{} \expval{ \sigma _i }
\end{equation}

\begin{remark}
The division by \( \abs{m_l}  \) in equation \eqref{eq:19_7} is crucial because it rescales the new variables \( S_I \) to assume only the values \( \pm 1 \), just like the original ones (rescaling of the fields).
\end{remark}

In the end we are left with a system of block spins on a hypercubic lattice with lattice constant \( la \). We can therefore rescale the spatial distances between the degrees of freedom of our system:
\begin{equation*}
  \va{r}_l = \frac{\va{r}}{l}
\end{equation*}
In other words, since \( la \) is now the characteristic length of the system we are measuring the distances in units of \( la \)  (just like in the original one we measured distances in units of \( a \)). The coarse graining procedure we have just seen is described by Figure \ref{fig:19_3} for a two-dimensional Ising model.

\begin{figure}[h!]
\begin{minipage}[c]{0.32\linewidth}
\subfloat[][Original system.]{ \includegraphics[width=0.8\textwidth]{./img/2.png}  \label{fig:} }
\end{minipage}
\begin{minipage}[]{0.32\linewidth}
\centering
\subfloat[][Block spins.]{\includegraphics[width=0.8\textwidth]{./img/3.png}  \label{fig:} }
\end{minipage}
\begin{minipage}[]{0.32\linewidth}
\centering
\subfloat[][Final rescaled system.]{\includegraphics[width=0.8\textwidth]{./img/4.png}  \label{fig:} }
\end{minipage}
\caption{\label{fig:19_3} Coarse graining procedure for a two-dimensional Ising model.}
\end{figure}

\noindent Kadanoff's argument now proceeds with two assumptions.

\subsubsection{\(  1^{st} \) crucial assumption}
The first assumption states that, in analogy to what happens in the original system, we assume that the block spins interact with the nearest neighbours and an external effective field (just like the original ones do). Hence, the Hamiltonian of the new system \( \mathcal{H}_l \) is equal in form to \( \mathcal{H}_ \Omega  \), the original one, of course provided that the spins, coupling constants and external fields are redefined.  If we call \( K_l \)  and \( h_l \) these new constants, the new effective Hamiltonian is:
\begin{equation}
  - \beta \mathcal{H}_l = K_l \sum_{\expval{IJ} }^{N_b} S_I S_J  + h_l \sum_{I=1}^{N_b} S_I
\end{equation}
\begin{remark}
This assumption is in general wrong!
\end{remark}
Since in the new system the lengths have been rescaled by a factor \( l \), this
means that in the new system all the lengths will be measured in units of \( la \).  Hence, also the correlation length  has to be measured in units of \( la \), and in particular we will have:
\begin{equation*}
  \xi _l = \frac{\xi }{l}
\end{equation*}
This means that the new system has a lower correlation length \( \xi _l < \xi  \), and so the system described by \( \mathcal{H}_l \)   is more distant from the critical point than the original one \( \mathcal{H}_{\Omega }\). Hence, we will have a new effective temperature:
\begin{equation*}
  t_l > t
\end{equation*}
 Similarly, in the coarse grained system the magnetic field \( h_l \) will be rescaled to an effective one:
\begin{equation*}
  h \sum_{i}^{} \sigma _i = h \sum_{I}^{} \sum_{i \in I}^{}  \sigma _i = h \sum_{I}^{}    \abs{m_l} l^d S_I
  = \underbrace{h \abs{m_l} l^d}_{h_l} \sum_{I}^{} S_I = h_l \sum_{I}^{} S_I
\end{equation*}
which implies that there is a relation between the new magnetic field and the mean magnetization:
\begin{equation*}
  h_l = h \abs{m_l} l^d
\end{equation*}

Since the Hamiltonian of the block spin system \( \mathcal{H}_l \) has the same form of the original one \( \mathcal{H}_ \Omega  \), the same will be true also for the partition function \( Z_l \) and the free energy, provided that \( h,K \) and \( N \) are substituted with \( h_l,K_l \) and \( N/l^d \); in particular, considering the singular part \( f_s \) of the free energy density we will have:
\begin{equation*}
  N_l f_s (t_l,h_l)= \frac{N}{l^d} f_s (t_l,h_l) = N f_s (t,h)
\end{equation*}
and so
\begin{equation*}
  \Rightarrow  f_s (t_l,h_l) = l^d f_s (t,h)
\end{equation*}

\begin{remark}
Note that the homogeneity condition is recovered with \( \lambda \equiv l^d \).
\end{remark}


\subsubsection{\(  2^{st} \) crucial assumption}
In order to proceed, we should ask how \( t \) and \( h \) change under the block spin transformation; hence, we now need the second assumption. We assume that:
\begin{equation}
   t_l = t l^{y_t}, \quad   h_l = h l^{y_h}, \quad y_t,y_h >0
  \label{eq:19_8}
\end{equation}
where the \( y_t, \, y_h \) are called \emph{scaling exponents} and are for now unspecified, apart from the fact that they must be positive (so that the coarse grained system is indeed farther from the critical point with respect to the original one).

The justification of this assumption lies in the fact that we are trying to understand the scaling properties of our system near a critical point, and these are the simplest possible relations between \( (t,h) \) and \( (t_l,h_l) \) that satisfy the following symmetry requirements:
\begin{itemize}
\item when \( h \rightarrow -h \), then \( h_l \rightarrow -h_l \);
\item when \( h \rightarrow -h \), then \( t_l \rightarrow t_l \);
\item when \( t=h=0 \), then \( t_l=h_l=0 \).
\end{itemize}

\noindent If we use this assumption in the free energy equation
\begin{equation*}
  f_s (t_l,h_l) = l^d f_s (t,h)
\end{equation*}
we get:
\begin{equation}
  f_s (t,h) = l^{-d} f_s (t l^{y_t}, h l^{y_h})
\end{equation}
This is very similar to Widom's scaling hypothesis (Eq.\eqref{eq:19_12}), but with the parameter \( \lambda  \)  that is the inverse of the block volume \( l^d \).
Since \( l \)  has no specified value, we can choose the one we want and again we use the properties of generalized homogeneous functions to eliminate one of the arguments of \( f_s \).
In particular, setting
\begin{equation*}
  l= \abs{t}^{-1/y_t}
\end{equation*}
we get:
\begin{equation}
  f_s (t,h) = \abs{t}^{d/y_t} f_s (1, h \abs{t}^{-y_h/y_t} )
\end{equation}
where the gap exponent is now
\begin{equation}
  \Delta = \frac{y_h}{y_t}
\end{equation}
comparing this equation with the alternative expression of the scaling hypothesis (Eq.\eqref{eq:19_13}) we have:
\begin{equation}
  2 - \alpha = \frac{d}{y_t}
  \label{eq:19_16}
\end{equation}


\subsection{Kadanoff's argument for two-point correlation functions}
Let us now compute the two-point correlation function of the block spin system:
\begin{equation}
  G_{IJ} (\va{r}_l, t_l) \equiv \expval{S_I S_J} - \expval{S_I}\expval{S_J}
\end{equation}
where \( \va{r}_l \) is the vector of the relative distance between the centers of the \( I \)-th and \( J \)-th block (measured in units of \( la \), as stated before). We want now to see how this correlation length is related to the one of the original system \( G(\va{r},t) \).
Since from the first assumption we have
\begin{equation*}
  h_l = h \abs{m_l} l^d \quad \Rightarrow \abs{m_l} = \frac{h_l l^{-d}}{h}
\end{equation*}
Using the second assumption, for which \( h_l = h l^{y_h} \), we obtain:
\begin{equation*}
  \abs{m_l} = l^{y_h-d}
\end{equation*}
Since
\begin{equation*}
  S_I = \frac{1}{\abs{m_l} } \frac{1}{l^D} \sum_{i \in I}^{} \sigma _i
\end{equation*}
the two-point correlation function becomes
\begin{equation*}
\begin{split}
  G_{IJ} (\va{r}_l, t_l) &=  \expval{S_I S_J} - \expval{S_I} \expval{S_J}
   = \frac{1}{l^{2 (y_h - d)} l^{2d}} \sum_{i \in I}^{} \sum_{j \in J}^{} \qty[ \underbrace{\expval{\sigma _i \sigma _j} - \expval{\sigma _i} \expval{\sigma _j} }_{ G_{ij} }    ] \\
  & = \frac{l^d l^d}{l^{2(y_h-d)}l^{2d}} \qty[ \expval{\sigma _i \sigma _j} - \expval{\sigma _i} \expval{\sigma _j}  ]
\end{split}
\end{equation*}
where in the last step we have made the assumption that, since \( la \ll \xi  \), \( G_{ij} \) inside a block is fairly constant, we can bring it outside the sum. Hence, the sum over \( i,j \) becomes \( l^{2d} \).
Hence, we have:
\begin{equation}
  G_{IJ} (\va{r}_l, t_l) = l^{2(d-y_h)} G_{ij} (\va{r},t)
\end{equation}
Introducing also the dependence on \( h \), we have:
\begin{equation}
  G_{IJ} \qty(\frac{\va{r}}{l}, t l^{y_t}, h l^{y_h}) = l^{2 (d- y_h)} G_{ij} (\va{r},t,h)
 \end{equation}
 Again, we can remove the dependence on \( t \) by setting \(  l = t^{-1/y_t}  \)  so that:
\begin{equation}
  G (\va{r},t,h) = t^{\frac{2(d-y_h)}{y_t}} G (\va{r}t^{1/y_t}, 1 , h t ^{-y_h/y_t})
  \label{eq:19_9}
\end{equation}
Now, \( \va{r} \) scales with \( l \) as all the lengths of our system, and since we have set \(  l = t^{-1/y_t}  \)  we have
\begin{equation*}
  \abs{\va{r}} t^{1/y_t} = 1  \quad \Rightarrow  t = \abs{\va{r}}^{-y_t}
\end{equation*}
Therefore, inserting in \eqref{eq:19_9}:
\begin{equation}
  G (\va{r},t,h) = \abs{\va{r}}^{-2(d-y_h)} F_G \qty(h t^{-y_h/y_t})
\end{equation}
where we have defined
\begin{equation}
  F_G \qty(h t^{-y_h/y_t})  \equiv G (1, 1, h t ^{-y_h/y_t})
\end{equation}
Let us remember that the power law behaviour of \( G \) in proximity of the critical point is
\begin{equation*}
  G \sim \abs{\va{r}}^{2-d - \eta }
\end{equation*}
we get
\begin{equation}
  2 (d- y_h) = d - 2 + \eta
  \label{eq:19_14}
\end{equation}
With the choice \( l = t^{-1/y_t} \) we further have that the correlation length scales as:
\begin{equation*}
  \xi = l \xi _l = \xi _l t^{-1/y_t}
\end{equation*}
and remembering that the correlation length diverges as
\begin{equation*}
  \xi \sim t^{-\nu }
\end{equation*}
 we also have:
\begin{equation}
    \nu = \frac{1}{y_t}
    \label{eq:19_15}
\end{equation}
The Eq.\eqref{eq:19_15} together with Eq.\eqref{eq:19_16} leads to the \emph{hyperscaling relation}:
\begin{empheq}[box=\myyellowbox]{equation}
  \Rightarrow 2 - \alpha = \nu d
\end{empheq}
Hyperscaling relations are known to be less robust than the normal scaling relations between critical exponents (for example, for Hamiltonians with long-ranged power law interactions hyperscaling relations don't hold).





\end{document}
