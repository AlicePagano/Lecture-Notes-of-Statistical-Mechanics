\documentclass[../main/main.tex]{subfiles}

\newdate{date}{13}{11}{2019}


\begin{document}

\chapter{The role of dimension, symmetry and range of interactions in phase transitions}


\marginpar{ \textbf{Lecture 10.} \\  \displaydate{date}. \\ Compiled:  \today.}


Which is the role of dimensionality in phase transition? Consider \emph{D}, the dimension of the system.
For the Ising model we have seen that in \( D=1 \) there is no phase transition while the Onsanger solution tell us that for \( D=2 \) there is a paramagnetic-ferromagnetic transition for \( T_c >0 \).
Therefore, dimensionality seems a crucial parameter!
Since in general analytic solutions are not available is there a simple argument to establish the existence of a phase transition?
In the case of a para-ferro transition may we establish wheter a phase with long range order exists and is stable withn a range of \( T>0 \)?

We need an argument that can tell us which kind of system has a phase transition. The idea is to use the entropy energy argument. Indeed, our systems are ruled by a free energy and the previous states are found by making derivative. We have energy and entropy: low energy state can be stable respect thermal fluctuations, but the fluctuations will destroy the long range order. This idea can be generalized.

\section{Energy-entropy argument}

\begin{equation}
  \dd[]{F} = \underbrace{\dd[]{U}}_{\text{energy}}  - T \underbrace{ \dd[]{S}}_{\text{entropy}}
\end{equation}
We expect that:
\begin{itemize}
\item \( T \gg 1\): entropy should dominates.
\item \( T \ll 1 \): energy should dominates.
\end{itemize}

\noindent Question: there is a temperature different to zero in which this is compatible?

\subsection{Ising \( D=1 \) }
Consider \emph{N} the size of the system.
Study the stability of the states with minimum energy to fluctuations for \( T \neq 0 \).
We already know that in the case \( T=0 \), two ground states exist, either all spins up or all spins down.

For instance, suppose that we have the ground state with all the spin up; the energy of the state is
\begin{equation}
   E_G = -JN
\end{equation}
Now consider \( T \neq 0 \), there could be a given number of elementary excitations of the kind spin up/down. What happens if we swap one or more spins?
These are defects respect the ground state and they are also called \emph{domain walls}.
This is in one dimentional, but is valid also in many dimensional.

Which is the variation in energy \( \Delta E \) respect the ground state?
For each excitation there is an energy penalty \( \Delta E = 2J \), indeed
\begin{equation}
  E_G = -JN, \qquad E^* = -J(N-1) +J \qquad \Rightarrow \Delta E = 2J
\end{equation}
Here we have supposed that we have only one swap. If there is a finite concentration of domains, we have \emph{x} such that there are \(   M = Nx \) domain walls.
Therefore:
\begin{equation}
  \Delta E_M = 2MJ
\end{equation}

Now compute the change in entropy.
We can compute immediately the entropy of the ground state: this is zero because is the logarithm of the number of configurations, but in this case we have only one configuration, namely  \( S_G = \ln{1} =  0 \). Therefore, the different of the entropies is just the entropy of the last state.

Now we estimate the entropy of the states with \emph{M} domain walls.  The number of possible ways to insert \emph{M} domains in \emph{N} positions, namely the number of configurations, is
\begin{equation}
  \# = \begin{pmatrix}
  N \\
  M
  \end{pmatrix} = \begin{pmatrix}
  N \\
  xN
  \end{pmatrix}
\end{equation}
We have:
\begin{equation}
  S_M = k_B \log{\begin{pmatrix}
  N \\
  M
  \end{pmatrix}}
\end{equation}
the difference is
\begin{equation}
  \Delta S = S_M -S_G = S_M = -k_B \ln{\begin{pmatrix}
  N \\
  xN
  \end{pmatrix} }
\end{equation}
Consider:
\begin{equation}
\begin{split}
\Delta F  &= F_M - F_G = \Delta E - T \Delta S  \\
          &= 2 M J - k_B T \ln{\begin{pmatrix}
          N \\
          M
          \end{pmatrix} } \\
          & = 2xNJ - k_B T \ln{\begin{pmatrix}
          N \\
          xN
          \end{pmatrix} } \\
          & = N \qty{ 2xJ +k_B T \qty[x \ln{x} + (1-x)\ln{(1-x)}  ] }
\end{split}
\end{equation}
were we have used the Stirling approximation: \( \ln{N!} = N \ln{N} - N   \).

Since equilibrium states are obtained by the minima of \emph{F} we can minimize with respect to \emph{x}. We are interested in the free energy in the bulk:
\begin{equation}
  \Delta f_{b_N} = \frac{\Delta F_N}{N}, \qquad \pdv{\Delta f_b}{x} = 0
\end{equation}
this gives
\begin{equation}
\begin{split}
  \pdv{}{x} \qty{ 2xJ +k_B T \qty[x \ln{x} + (1-x)\ln{(1-x)}  ] } &=  2J + k_B T \qty[\ln{x} +1 -\ln{(1-x)} -1 ]\\
  & = 2J + k_B T \qty[\ln{x}- \ln{(1-x)}  ] = 0
\end{split}
\end{equation}
therefore
\begin{equation}
  \ln{\frac{x}{1-x}} = -\frac{2J}{k_B T} \quad \Rightarrow \frac{x}{1-x} = e^{-2J/k_BT}
\end{equation}
and finally the results is
\begin{equation}
 x = \frac{1}{1+e^{2J/k_BT} }
\end{equation}
It means that  \( \forall T \neq 0 \) exist a finite concentration \emph{x}. The ground state is unstable \( \forall T>0 \). Indeed, if you have a finite density of \emph{x} it means no long range order exist for \( T>0 \).  It tells you that at \( T=0 \), \( x=0 \).

Let us try to do the same for \emph{D} dimensions.

\subsection{\( D \) dimension}
What is a \emph{domain wall} in \emph{D} dimension? The domain walls is an hypersurface of size \( L^{D-1} \)
\begin{equation}
  \Delta E \propto 2JL^{D-1}
\end{equation}
Computing the entropy it is a very difficult problem. In fact, the entropy of a fluctuating hypersurface is difficult to estimate. For a single domain wall we can say
\begin{equation}
  S^* \ge k_B \ln{L}
\end{equation}
the number of ways to place a straight wall within a system of linear size \emph{L}.
The \( \Delta S \)  is just \emph{S}  because the entropy of the ground state is again zero.
\begin{remark}
Understimate of \emph{S}
\begin{equation}
  \Delta F = 2JL^{D-1}-k_B T \ln{L}
\end{equation}
it means that now energy can win if the temperature is different from zero. Therefore, for \( D=2 \)  or greater (\( D>1 \)) that long range order can survive  thermal fluctuations, the system could present an ordered phase!
\end{remark}

\subsubsection{Peierls argument}
Idea: try to perturbe the system using an external magnetic field as perturbation (it is very small \emph{h}). We are breaking explicity the symmetry, but then you take the limit \( h \rightarrow 0 \) and swich off the magnetic field. This is the typical way to see this stability.

An argument due to \emph{Pierls} shows that this is indeed the case. We now that for finite systems, from the \( \mathbb{Z}^2 \) symmetry it follows
\begin{equation}
  \expval{m}_N = 0
\end{equation}
If, as \( N \rightarrow \infty  \) we have \( \expval{m}_{\infty } \neq 0  \) with \( T < T_c \). We have spontaneous symmetry breaking. This can be seen as a thermodynamical instability, namely the value of \( \expval{m}_ \infty   \) in the ferromagnetic phase, is determined by small perturbations. Usually the value of \( \expval{m}_ \infty   \)  is determined by using an infinitesimal magnetic field:
\begin{equation}
  \expval{m}_{\infty } = \lim_{h \rightarrow 0^+} \lim_{N \rightarrow \infty } \expval{m}_N^{(h)}
\end{equation}
\begin{remark}
It is crucial to take first the thermodynamic limit and then the limit \( h \rightarrow 0^+ \)!
\end{remark}
That is to see the stability.  Another way to do that is instead using a small \emph{h}, using periodic boundary condition.

A different infinitesimal perturbation that trigger the instability is to chose appropriate boundary conditions. For example, if we want \( \expval{m}_ \infty  > 0\) one can choose a boundary condition such that at the surface of the system all the spins are up (\( S_i = +1 \) with \( i \in \partial{\Omega }  \)  ), as in Figure \ref{fig:10_1}.

\begin{figure}[h!]
\centering
\includegraphics[width=0.5\textwidth]{../lessons/10_image/1.pdf}
\caption{\label{fig:10_1} }
\end{figure}

 It is equivalent, because on the border we have a very high field, by it is at the border so it do not really count for the surface internal.
This is a very smart way to do this perturbation.  The idea is therefore to perturbe the surface. This particular configuration with all the spin up will give you a particular shape, you can do this also in higher dimensional. From this you can also exstimate the temperature.

In the paramagnetic phase this constraint at the boundary does not affect the disorder bulk phase. In the ferromagnetic phase, however, this boundary condition has the same effect of considering an infinitesim magnetic field.

This is the boundary condition chosen by Pierls to establish the existence of a \( T_c \neq 0 \) for the \( < D \) Ising model.

Let us gives just a qualitative presentation of the (rigorous) result. If one is interested to the full proof refer for example to \cite{10_lesson_1}.

Let \( N_+,N_- \) the number of spin up and down respectively. Clearly, \( N=N_+ + N_- \) and
\begin{equation}
  \expval{m}_N = \frac{\expval{N_+} - \expval{N_-}  }{N} = 1 - 2 \frac{\expval{N_-} }{N}
\end{equation}
In order to show that \( \expval{m}_ \infty >0  \) (remember that we are considerin boundary conditions with spin ap at \( \partial{\Omega }  \)) it is sufficient to show that \( \forall N \)
\begin{equation}
  \frac{\expval{N_-} }{N}  < \frac{1}{2}- \varepsilon
  \label{eq:10_1}
\end{equation}
where \( \varepsilon >0 \) and indipendent on \emph{N}.
Indeed, if \eqref{eq:10_1} holds
\begin{equation}
  \expval{m}_N \ge 2 \varepsilon \quad \forall N
\end{equation}
Pierles was indeed able to show that
\begin{equation}
  \frac{\expval{N_-} }{N} \le f_D (x)
\end{equation}
where \( f_D \) is a continuous function of \( x \) indipendent on \emph{N} and so
\begin{equation}
  x = q e^{-4J \beta }
\end{equation}
and such that \( \lim_{x \rightarrow 0} f_D (x) = 0 \).

In particular for \emph{T} sufficiently small

\begin{equation}
  \frac{\expval{N_-} }{N} < \frac{1}{2} - \varepsilon
\end{equation}
More precisely one has
\begin{equation}
    \frac{\expval{N_-} }{N} \le \frac{x^2}{36} \frac{2-x}{(1-x)^2}
\end{equation}
where \(   x = q e^{-4J \beta } < 1 \).


Note that above bound gives also a lower bound on the critical temperature
\begin{equation}
    \frac{\expval{N_-} }{N} \le \frac{x^2}{36} \frac{2-x}{(1-x)^2} < \frac{1}{2} - \varepsilon
\end{equation}
As long as \(   \frac{\expval{N_-} }{N}  < \frac{1}{2} - \varepsilon  \) the system is in the ferromagnetic phase. The critical value \( x_c \equiv x (\beta _c) \) must be outside the interval \( [0,x_{1/2}] \) where \( x_{1/2} \)   is the smallest positive solution of the equation
\begin{equation}
  \frac{x^2}{36} \frac{2-x}{(1-x)^2} = \frac{1}{2}
\end{equation}
From the solution \( x_{1/2} \) and the condition \( x_c > x_{1/2} \) one has
\begin{equation}
  J \beta _c \le J \beta_{1/2}
\end{equation}
where \( J \beta _{1/2} = \frac{1}{4} \log{q/x_{1/2}}  \) which implies \(  T_c > T_{1/2} \).
\begin{exercise}{}{}
The equation
\begin{equation}
  x^3 + 16x^2 - 36 x + 18 = 0
\end{equation}
gives \( x_{1/2} \). Found \( T_{1/2} \).
\end{exercise}

\section{Role of the symmetry}
Interacting systems can be classified with respect to their \emph{global symmetry group}.
\begin{example}{Ising model}{}
  \begin{equation}
    \mathcal{H}_{\text{Ising}} = - \sum_{i<j}^{} J_{ij} \sigma _i \sigma _j
  \end{equation}
  where \( \sigma _i \in \{ -1,1 \}   \). The symmetry gropu of this Hamiltonian is \( \mathbb{Z}^2 \), which has two elements \( \{ \mathbb{1}, \mu  \}   \). We have
  \begin{equation}
    \mathbb{1}: \text{ identity}, \quad \mu \sigma _i = - \sigma _i, \quad \mu ^2 = \mathbb{1}
  \end{equation}
\end{example}
\begin{example}{Potts model}{}
  \begin{equation}
    \mathcal{H}_{q- \text{Potts}} = - \sum_{i<j}^{} J_{ij} \delta _{\sigma _i, \sigma _j}
  \end{equation}
  where \( \sigma _i \in [1,2,3,\dots,q] \). \( \mathcal{H}_{q- \text{Potts}} \)  is invariant under the permutation group of the sequence \( \{ 1,2,3,\dots,q \}   \). There are \( q! \) elemtns, for example \( \{ 2,1,3,\dots,q \}   \). The symmetry group is denoted by \( S_q \).
\end{example}
\begin{remark}
The difference between a \( \mathbb{Z}_q \) and \( S_q \) symmetry is that an Hamiltonian has symmetry \( \mathbb{Z}_q \) if it is invariant with respect to \emph{cyclic permutations}
\begin{equation}
  \mu = \begin{pmatrix}
    1 & 2 & \dots & q-1 & q \\
    2 & 3 & \dots & q & 1
  \end{pmatrix}
\end{equation}
and its powers \( \mu ^l \) with \( l=0, \dots, q-1 \). Both models satisfy a \emph{discrete global symmetry}.
\end{remark}
Now, we jump into the case in which we consider \emph{continuous} symmetries.

\section{XY model}
This is a spin model that is invariant with respect to the continuous global symmetry
\( \theta _i \rightarrow \theta _i + \alpha  \).
Indeed the Hamiltonian of this model is
\begin{equation}
  \mathcal{H}_{XY} = - \sum_{i<j}^{} J_{ij} \va{S}_i \vdot \va{S}_j
\end{equation}
where \(\va{S}_i  \) is a \( 2D \) spin vector
\begin{equation}
  \va{S}_i = (S_{x_i}, S_{y_i})
\end{equation}
that can assume values on the unit circle ( \(  \abs{\va{S}_i} =1 \)).

Suppose that you have spins that are sitting in hyper dimensional. Rotate along a circle this spins. They can assume all the value as in Figure \ref{fig:10_2}.

\begin{figure}[h!]
\centering
\includegraphics[width=0.5\textwidth]{../lessons/10_image/2.pdf}
\caption{\label{fig:10_2} }
\end{figure}
The simplest way to parametrize the Hamiltonian is by the angle.
Denoting by \( \theta _i \) the direction angle of spins \( \va{S}_i \), the \( \mathcal{H}_{XY} \) can be written as
\begin{equation}
  \mathcal{H}_{XY} = - \sum_{i<j}^{} J_{ij} \cos(\theta _i - \theta _j)
\end{equation}
with \( \theta _i \in [0,2 \pi ] \).
\begin{remark}
The interaction term \( \cos(\theta _i - \theta _j)  \) can be written also as
\begin{equation}
  \frac{1}{2} \qty(Z_i^* Z_j + Z_i Z_j^*)
\end{equation}
where \(   Z_j = \exp (i \theta _j) \).
\end{remark}
The model is invariant under the global transformation
\begin{equation}
  Z_i \rightarrow e^{i \alpha } Z_i
\end{equation}
The phase  \( \exp (i \alpha )  \) form a group under multiplication known as \( U(1) \) that is equivalent to \( O(2) \). Indeed the interaction term can be written also as
\begin{equation}
  \vu{\Omega }_i \vdot \vu{\Omega }_j
\end{equation}
where \( \vu{\Omega }_i = ( \cos \theta _i, \sin \theta _i) \).
\begin{remark}
In \emph{n}-dimensions \( \vu{\Omega } \) has \emph{n} components  \( \vu{\Omega } = \{ \Omega ^1, \Omega ^2, \dots, \Omega ^n \}   \)  and the corresponding Hamiltonian
\begin{equation}
  \mathcal{H} = - \sum_{i>j}^{} J_{ij} \vu{\Omega }_i \vdot  \vu{\Omega }_j
\end{equation}
It is symmetric with respect to the global symmetry group \( O(n) \).
\end{remark}
 Which are the domain walls for continuous symmetries? Which are the implications for the stability of the ordered phase?



\section{Continuous symmetries and phase transitions}
The questions is: when it is not continuous which is the boundary?
When the symmetry is continuous the domain walls interpolate smootly between two ordered regions (see Figure \ref{fig:10_3}).

\begin{figure}[h!]
\centering
\includegraphics[width=0.4\textwidth]{../lessons/10_image/3.pdf}
\caption{\label{fig:10_3} }
\end{figure}
The energy term that in Ising is proportional to \( 2JL^{D-1} \) how does it change here?
Suppose that the variation of the direction between two neirest neighbours sites is very small, i.e. \( (\theta _i - \theta _j) \ll 1 \) for \( i,j \) neirest neighbours.
 Now we can diluite the energy, in other words weak the energy term.

Let us do a Taylor expainding of the interaction term
\begin{equation}
  \cos (\theta _i - \theta _j) \simeq 1 - \frac{1}{2} (\theta _i - \theta _j)^2 \Rightarrow \sum_{\expval{ij} }^{}  \qty(1 - \frac{1}{2} (\theta _i - \theta _j)^2)
  \label{eq:10_2}
\end{equation}
The Hamiltonian can be written as
\begin{equation}
  \mathcal{H} \simeq - J \sum_{\expval{ij} }^{} \qty(1- \frac{1}{2}(\theta _i - \theta _j)^2)
\end{equation}
The \eqref{eq:10_2} corresponds to the discrete differential operator where
\( \theta _i - \theta _j = \partial_x{\theta }  \),
therefore
\begin{equation}
  \mathcal{H} = E_0 + \underbrace{\frac{J}{2} \int_{}^{} \dd[]{\va{r}} \qty(\grad \theta )^2 }_{E \equiv \text{Stifness energy}}
\end{equation}
where \( E_0 = 2JN \) is the energy corresponding to the case in which all the spins are oriented along a given direction.

\begin{definition}[Stifness energy]
  The Stifness energy is defined as
  \begin{equation}
    E = \frac{J}{2} \int_{}^{} \dd[]{\va{r}} \qty(\grad \theta )^2
  \end{equation}
where \( \theta (\va{r})  \) is the angle of a local rotation around an axis and \emph{J} is the \emph{spin rigidity}. For an ordered phase \( \theta (\va{r}) = \theta _0 \).
\end{definition}
Let us now imagine a domain wall where \(\theta (\va{r})  \)  rotates by \( 2 \pi  \) (or \( 2 \pi m \)) by using the entire length of the system (see again Figure \ref{fig:10_3}):
\begin{equation}
  \theta (\va{r}) = \frac{2 \pi n x}{L}
\end{equation}
where \emph{n} is the total number of \( 2 \pi  \) turn of \( \theta  \) in \emph{L}. Note that there is no variation along the other \( D-1 \) dimensions, therefore we just doing over one dimension.

 Consider only the term called \emph{E}
 \begin{equation}
   E = \frac{J}{2} L^{D-1} \int_{0}^{L} \dd[]{x} \qty(\dv{}{x} \qty(\frac{2 \pi n x}{L})  )^2 = \frac{J}{2} L^{D-1} \int_{0}^{L} \dd[]{x} \qty(\frac{2 \pi n}{L})^2 \approx 2 \pi ^2 n^2 J L^{D-2}
 \end{equation}
\begin{remark}
Unlike the Ising model where \( E \sim L^{D-1} \), here \( E \sim L^{D-2} \)!

If \( S \ge k_B \ln{L}  \) for a single domain wall, \emph{S} should dominate if \( D \le 2 \), the ordered phase is always unstable and no phase transition is expected for \( T \neq 0 \)!
\end{remark}

\begin{definition}[Lower critical dimension]
  The Lower Critical dimension \( D_c \) is the dimension at which (and below which) the system does not display a ordered phase (there is no long range order).
  In other words if \( D \le D_c \), we have \( T_c = 0 \).
\end{definition}
\begin{theorem}[Mermin-Wagner]
  For continuous global symmetries the \( D_c =2 \).
\end{theorem}

 From what we have dound before we can say that
\begin{itemize}
\item Discrete global symmetries: \( D_c =1 \).
\item Continuous global symmetries: \( D_c =2 \).
\end{itemize}

\begin{remark}
The \emph{XY} model in \( D=2 \) is rather special. Although it dos not display an ordered phase, there exist at \( T \neq 0 \) a special phase transition known as the \emph{Kosterlitz-thouless transition}. This transition dows not imply the spontaneous breaking of the \( O(2) \) symmetry! There is no long range order for \( T<T_{KT} \) (statistic of vortices, topological defects in \( D=2 \)).
\end{remark}


\section{Role of the interaction range}
So far we have considered models where the interactions were short range. How things change if long range are considered instead? How does the symmetry broken depends on the range of interactions? One can show, for example, that if
\begin{equation}
  J_{ij} = \frac{J}{\abs{\va{r}_i - \va{r}_j}^\alpha  }
\end{equation}
with \( 1 \le \alpha \le 2 \), phase with long range order is stable for \( 0 < T < T_c \) also for \( D=1 \)!   So we have a long range order with \( T_c \neq 0 \), even for \( D=1 \)!

\begin{remark}
If \( \alpha > 2 + \varepsilon  \) we get back the physics found for short range interactions. If \( \alpha <1 \) the thermodynamic limit does not exist.
\end{remark}
A limiting case of long range interaction is the infinite range case where all the spins interact one to another with the same intensity indipendently on their distance. No metric is involed (instead of previously where the definition of \emph{J} of before is a metric.). It can be solve exactly and later we will seen why.


\section{Ising model with infinite range}

The Hamiltonian is the following:
\begin{equation}
  -\mathcal{H}_N (\{ S \} ) = \frac{J_0}{2} \sum_{i,j}^{N} S_i S_j + H \sum_{i}^{}  S_i
\end{equation}
with \( S_i \in [-1,+1] \). The problem is the double sum over \emph{i,j}, indeed
\begin{equation}
  \sum_{i,j}^{} S_i S_j \propto O(N^2)
\end{equation}
and the thermodynamic limit is ill-defined. To circumvent this problem \emph{Kac}  suggested to consider a strength
\begin{equation}
  J_0 = \frac{J}{N}
\end{equation}
this is called the \emph{kac} approximation
\begin{equation}
  -\mathcal{H}_N (\{ S \} ) = \frac{J}{2N} \sum_{i,j}^{N} S_i S_j + H \sum_{i}^{}  S_i
\end{equation}
with this choice we recover \( E \sim O(N) \).  In this Hamiltonian since you have no metric you have no dimension.





\end{document}
