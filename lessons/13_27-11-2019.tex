\documentclass[../main/main.tex]{subfiles}

\newdate{date}{27}{11}{2019}


\begin{document}

\marginpar{ \textbf{Lecture 13.} \\  \displaydate{date}. \\ Compiled:  \today.}

Using the constraints
\begin{equation}
  \begin{cases}
    \Tr^{(i)}(\rho _i) = 1  &\rightarrow  a+b=1\\
    \Tr^{(i)}((\rho _i S_i) = m_i  & \rightarrow a-b = m_i
  \end{cases}
\end{equation}
where \( a,b \) are the functions of the order parameter.
In that case we have not to write the functions for all the \emph{i}. For \( S_i = 1 \) we have one value, for all the other values another one.

The results of the previous equation are:
\begin{equation}
  \begin{cases}
   a = \frac{1-m_i}{2} \\
   b = \frac{1+m_i}{2}
  \end{cases}
\end{equation}
Hence,
\begin{equation}
  \rho _i =   \frac{1-m_i}{2}  (1- \delta _{S_i,-1}) + \frac{1+m_i}{2} \delta _{S_i,-1}
\end{equation}
In matrix form
\begin{equation}
\begin{pmatrix}
\frac{(m_i+1)}{2}   & 0 \\
  0 &    \frac{(1-m_i)}{2}
\end{pmatrix}
\end{equation}
\subsubsection{Mean-field energy term}
Let us consider the Hamiltonian
\begin{equation}
  \expval{\mathcal{H}}_{\rho _{MF}} = \expval{-J \sum_{\expval{ij} }^{} S_i S_j - \sum_{i}^{} H_i S_i   }_{\rho _{MF}}
  = - J   \sum_{\expval{ij} }^{} \expval{S_i S_j}_{\rho _{MF}} - \sum_{i}^{} H_i \expval{S_i}_{\rho _{MF}}
\end{equation}
Since
\begin{equation}
  \rho _{MF} = \prod_{i=1}^{N} \rho _i
\end{equation}
the term \( \expval{S_i S_j}_{\rho _{MF}}  \)  will trasform into
\begin{equation}
  \expval{S_i S_j}_{\rho _{MF}} = \expval{S_i}_{\rho _{MF}} \expval{S_j}_{\rho _{MF}}
\end{equation}
Moreover, for all function \( g \) of \( S_i \) we can write
\begin{equation}
\begin{split}
  \expval{g(S_i)}_{\rho _{MF}} &= \Tr^{(i)}(g(S_i)\rho _i) = \sum_{S_i = \pm 1}^{} g(S_i) \rho _i    \\
  &= \sum_{S_i = \pm 1}^{} g(S_i) \qty[\frac{1+m_i}{2} \delta _{S_i,1} + \frac{1-m_i}{2} (1- \delta _{S_i,1})  ] \\
  & = \frac{1+m_i}{2}g(1) + \frac{1-m_i}{2} g(-1)
\end{split}
\end{equation}
Note that, if \( g(S_i) = S_i \),
\begin{equation*}
  \expval{S_i}_{\rho _{MF}} = m_i
\end{equation*}
as expected.

Hence,
\begin{equation}
  \expval{\mathcal{H}}_{\rho _{MF}} = -J \sum_{\expval{ij} }^{} m_i m_j - \sum_{i}^{} H_i m_i
\end{equation}
\begin{remark}
This has the form of the original Hamiltonian where \( S_i \) have been replaced by their statistical averages.
\end{remark}
The entropy term is:
\begin{equation}
\begin{split}
  \expval{\ln{\rho } }_{\rho _{MF}} & = \Tr(\rho \ln{\rho } )  \overset{MF}{=} \sum_{i}^{} \Tr^{(i)} (\rho _i \ln{\rho _i} ) \\
 &  = \sum_{i}^{} \qty[ \frac{1+m_i}{2} \ln{\frac{1+m_i}{2}} + \frac{1-m_i}{2} \ln{\frac{1-m_i}{2}} ]
\end{split}
\end{equation}
The total free energy becames:
\begin{equation}
\begin{split}
  F_{\rho _{MF}} &= \expval{\mathcal{H}}_{\rho _{MF}} + k_B T \expval{\ln{\rho } }_{\rho _{MF}} \\
  & = - J \sum_{\expval{ij} }^{} m_i m_j - \sum_{i}^{} H_i m_i
  + k_B T   \sum_{i}^{} \qty[ \frac{1+m_i}{2} \ln{\frac{1+m_i}{2}} + \frac{1-m_i}{2} \ln{\frac{1-m_i}{2}} ]
\end{split}
\end{equation}
We now look for the values \( m_i = \bar{m_i}  \), that minimizes \( F_{\rho _{MF}} \) (equilibrium phases):
\begin{equation}
 \eval{ \pdv{F_{\rho _{MF}} }{m_i} }_{m_i = \bar{m_i} } = 0
\end{equation}
This gives:
\begin{equation}
  0 = - J \sum_{j \in \, n.n. \,\text{of}\, i}^{} \bar{m_j} - H_i + \frac{k_B T}{2} \ln{\qty[\frac{1+\bar{m_i} }{1- \bar{m_i} }] }
\end{equation}
To solve it, remember that
\begin{equation}
  \tanh^{-1} (x) = \frac{1}{2} \ln{\frac{1+x}{1-x}} \quad (TO\, DO)
\end{equation}
Hence,
\begin{equation}
  k_B T \tanh^{-1} ( \bar{m_i} ) = J \sum_{j \in \, n.n. \,\text{of}\, i}^{} \bar{m_j} + H_i
\end{equation}
which implies
\begin{equation}
  \bar{m_i} = \tanh \qty[(k_B T)^{-1} \qty(J \sum_{j \in \, n.n. \,\text{of}\, i}^{} \bar{m_j} + H_i ) ]
\end{equation}
Defining
\begin{equation}
  z \bar{m_i} \equiv  \sum_{j \in \, n.n. \,\text{of}\, i}^{} \bar{m_j}
\end{equation}
we get
\begin{equation}
  \bar{m_i} = \tanh \qty[\beta \qty(Jz \bar{m_i} +H_i) ]
\end{equation}
this is the Bragg-William approximation.

\begin{example}[Ising antiferromagnet in an external field]
Consider the model
\begin{equation}
  \mathcal{H} = \mathcolorbox{green!20}{+} J \sum_{\expval{ij} }^{} S_i S_j - H \sum_{i}^{} S_i,
\end{equation}
\begin{remark}
Note the \( + \) before \( J \). This means that the interactions are antiferromagnetic.
\end{remark}
\begin{itemize}
\item  If \( H=0 \) ferromagnetic and antiferromagnetic behave similarly when the interactions are between nearest neighbours on a \emph{bipartite lattice}, i.e. a lattice that can be divided into two sublattices, say \( A \) and \( B \), such that a \( A \) site has only \( B \) neighbours and a \( B \) site only \( A \) ones.
\begin{remark}
FCC is not bipartite, while BCC it is. See Figure \ref{fig:13_1}.
\end{remark}

\begin{figure}[h!]
\begin{minipage}[c]{0.5\linewidth}
\subfloat[][Square lattice is bipartite.]{ \includegraphics[width=0.8\textwidth]{../lessons/13_image/1.pdf}  \label{fig:} }
\end{minipage}
\begin{minipage}[]{0.5\linewidth}
\centering
\subfloat[][Triangular lattice is not bipartite.]{\includegraphics[width=0.8\textwidth]{../lessons/13_image/2.pdf}  \label{fig:} }
\end{minipage}
\caption{\label{fig:13_1} }
\end{figure}

If the lattic is bipartite and \( J_{ij} \) is non zero only when \( i \) and \( j \) belong to different sublattices (they do not have to be only n.n.!), one can redefine the spins such that
\begin{equation}
  S_j' \begin{cases}
    + S_j & j \in A \\
    - S_j & j \in B
\end{cases}
\end{equation}
Clearly, \( S_i'S_j' = - S_i S_j \). It is like if the \( J_{ij} \) have changed sign and we are formally back to ferromagnetic model for the two sublattices:
\begin{equation}
  \mathcal{H}^{*} = - J \sum_{\expval{ij} }^{} S_i' S_j'
\end{equation}
i.e. a ferromagnetic Ising.

\item In presence of a magnetic field \( H \), we need to reverse its sign when applied to sites \( B \).

The thermodynamic of a ferromagnetic Ising model on a bipartite lattice in a uniform magnetic field \( H \) is identical to the one of the Ising antiferromagnetic model in presence of the so called \emph{staggered field}, i.e. \( H_A = H \) and \( H_B = -H \).
\begin{equation}
  \mathcal{H}^* [S] = -J \sum_{\expval{r_A r_B} }^{} S(r_A) S(r_B) - H \sum_{r_A}^{} S(r_A) + H \sum_{r_B}^{} S(r_B), \quad J>0, H>0
\end{equation}
The average magnetization per spin is
\begin{equation}
  m \equiv \frac{1}{2}(m_A+m_B)
\end{equation}
  while
  \begin{equation}
    m_S = \frac{1}{2}(m_A-m_B)
  \end{equation}
  is the \emph{staggered magnetization}.

In order to use the variational density matrix method for this problem we consider two independent variational parameters \( m_A \) and \( m_B \) for sublattice \( A \) and \( B \) respectively. On each sublattice, the model is like the standard Ising
\begin{equation}
  \begin{cases}
   \rho _A^{(1)}(S) = \frac{1+m_A}{2} \delta _{S,1}+ \frac{1-m_A}{2}\delta _{S,-1}\\
   \rho _B^{(1)}(S) = \frac{1+m_B}{2} \delta _{S,1}+ \frac{1-m_B}{2}\delta _{S,-1}
  \end{cases}
\end{equation}
\begin{remark}
Note that, being \( H \) uniform, \( \expval{S_i} = m \), i.e. does not depend on \( i \). Same for the \( 1- \)particle distribution functions \( \rho _A^{(1)}(S) \) and  \( \rho _B^{(1)}(S) \).
\end{remark}
By performing the calculation for the terms
\begin{equation}
  \expval{\mathcal{H}}_{\rho _{MF}} = - J \sum_{\expval{ij} }^{} \expval{S_i S_j}_{\rho _{MF}} - H \sum_{i}^{} \expval{S_i}_{\rho _{MF}}
\end{equation}
\begin{equation}
  \expval{\ln{\rho } }_{\rho _{MF}} = \sum_{i}^{} \Tr^{(1)}(\rho _i \ln{\rho _i} )
\end{equation}
as before, but remembering to partition the procedure into the two sublattices \( A \) and \( B \), one can show (TO DO) that the variational free energy is given by
\begin{equation}
  \frac{F(m_A,m_B)}{N} = \frac{z \hat{J} }{2}m_A m_B - \frac{1}{2}H (m_A+m_B)
  - \frac{1}{2} k_B T s(m_A) - \frac{1}{2}k_B T s(m_B)
\end{equation}
where the entropy is given by
\begin{equation}
  s(m) = \qty[\frac{1+m}{2} \ln{\qty(\frac{1+m}{2}) } + \frac{1-m}{2} \ln{\qty(\frac{1-m}{2}) }  ]
\end{equation}
By differentiating \( \frac{F}{N} \) with respect to \( m_A \) and \( m_B \), one gets
\begin{subequations}
\begin{align}
   \pdv{(F/N)}{m_A} &= 0 & \Rightarrow m_B = \frac{H}{z \hat{J} } - \frac{k_B T}{z \hat{J} }\ln{\qty(\frac{1+m_A}{1-m_A}) } \\
   \pdv{(F/N)}{m_B} &= 0 & \Rightarrow m_A = \frac{H}{z \hat{J} } - \frac{k_B T}{z \hat{J} }\ln{\qty(\frac{1+m_B}{1-m_B}) }
\end{align}
\end{subequations}
As before, since
\begin{equation}
  \tanh^{-1} (x) = \frac{1}{2} \ln{\frac{1+x}{1-x}}
\end{equation}
these self-consisten equations can be written as
\begin{equation}
  \begin{cases}
   m_A = \tanh ( \beta \qty(H - z \hat{J} m_B ) )\\
   m_B = \tanh ( \beta \qty(H - z \hat{J} m_A ) )
  \end{cases}
\end{equation}
The sites \( \in A \) experience an internal field \( H_{A,MF} = - z \hat{J} m_B \) from the \( B \) neighbours and viceversa for the sites \( \in B \).
\end{itemize}
\end{example}

\subsection{Second approach: Blume-Emery-Griffith model}
We apply this approach to the so called Blume-Emery-Griffith model.
This is a spin model with vacancies that describes the phase diagram and the critical properties of an interacting system displaying a \emph{tricritical point}. Perhaps the most famous of these systems is the \( \text{He}^3-\text{He}^4 \) mixture undergoing a fluid-superfluid transition.

\begin{remark}
\( \text{He}^4 \)  is a non radiative isotope with two protons and two neutrons. Roughly \( 1/4 \) of the universe matter is \( \text{He}^4 \)!
From a quantum statistical point of view \( \text{He}^4 \) is a \emph{boson}.

\end{remark}
 A gas of \( \text{He}^4 \) undergoes a fluid-superfluid transition at \( T_ \lambda =2.17 K \) and a \( P=P_0 \). It is known as \( \lambda - \)transition since at \( T \sim T_ \lambda  \)  the specific heat \( C(T) \) behaves as in Figure \ref{fig:13_2_1}.

\begin{remark}
The \( \lambda - \)transition is a genuine critical point (second order). For \( T < T_{\lambda } \), \( \text{He}^4 \)  is in the superfluid phase and it can be described by a two-fluids model in which one component has zero viscosity and zero entropy.
\end{remark}

\begin{figure}[h!]
\begin{minipage}[c]{0.5\linewidth}
\subfloat[][d]{ \includegraphics[width=0.9\textwidth]{../lessons/13_image/3.pdf}  \label{fig:13_2_1} }
\end{minipage}
\begin{minipage}[]{0.5\linewidth}
\centering
\subfloat[][\( (P,T) \) phase diagram.]{\includegraphics[width=0.9\textwidth]{../lessons/13_image/4.pdf}  \label{fig:13_2_2} }
\end{minipage}
\caption{\label{fig:13_2} }
\end{figure}

Now, the question is: what happens to the system if a given amount of \( \text{He}^3 \) is inserted to form a \( \text{He}^3- \text{He}^4  \) mixture?
\( \text{He}^3 \) is a non-radiative isotope with 2 protons and 1 neutron. From a quantum statistical point of vieq is a \emph{fermion}.

Hence, if inserted in a system of \( \text{He}^4 \) it will "dilute" its bosonic property. Then, one expects that \( T_ \lambda  \) decreases. Indeed, denoting by \( x \) the concentration of \( \text{He}^3 \)  one observes
\begin{equation}
  T_{\lambda} = T_ \lambda (x)
\end{equation}
with \( T_ \lambda (x) \)  that decreases as \( x \) increases.

At the same time, at a given point the mixture undergoes a separation between a phase rich and a phase poor of \( \text{He}^3 \). In particular one observes that, for
\begin{equation}
  x > x_t = \frac{n_3}{n_3+n_4} \sim 0.67
\end{equation}
the fluid-superfluid transition becames first order! It is accompained by a phase separation. The point \( (x_T,T_t) \) is a \emph{tricritical point}, i.e. a critical point that separates a line of second order transition from a line of first order transition.
\subsubsection{BEG Model}
The BEG Model is the model of a diluited ferromagnetic system.
The spins are \( S_i = \pm 1,0 \) (similar to a lattice gas model), we have \( S_i = \pm 1 \) for \( \text{He}^4 \) atom at site \( i \), \( S_i = 0 \) for \( \text{He}^3 \) atom at site \( i \).

Let us consider:
\begin{itemize}
\item \( \expval{S_i} = m_i  \), order parameter.
\item \( \expval{S_i^2}  \) is the density \( \text{He}^4 \) atoms.
\end{itemize}
Let us define the density of \( \text{He}^3 \) atoms as
\begin{equation}
  x \equiv 1 - \expval{S_i^2}
\end{equation}
The chemical potentials difference is
\begin{equation}
  \Delta \propto \mu _3 - \mu _4
\end{equation}
and controls the number of \( \text{He}^3 \) atoms.

If:
\begin{itemize}
\item \( \Delta \rightarrow - \infty \quad \Rightarrow x \rightarrow 0  \).
\item \( \Delta \rightarrow + \infty \quad \Rightarrow x \rightarrow 1  \).
\end{itemize}
and the order parameter for the \( \lambda - \)transition becomes
\begin{equation}
\expval{S_i} =
  \begin{cases}
   0 & T > T_{\lambda }\\
   m & T < T_{\lambda }
  \end{cases}
\end{equation}
The minimal version of the model is:
\begin{equation}
\mathcal{H} = - J \sum_{\expval{ij} }^{N} S_i S_j + \Delta \sum_{i=1}^{N} S_i^2 - \Delta N
\end{equation}
\begin{remark}
The \( \Delta N \) term is a typical term for a gas in gran canonical ensemble.
\end{remark}


\subsubsection{Variational mean field approach to BEG}
Since \( \rho _{MF} = \prod_{i}^{} \rho _i   \),
\begin{equation}
  G(T,J,\Delta ) = \expval{\mathcal{H}}_{\rho _{MF}} + k_B T \sum_{i}^{} \Tr(\rho _i \ln{\rho _i } )
\end{equation}
where the first term can be written as
\begin{equation}
\begin{split}
\expval{\mathcal{H}}_{\rho _{MF}}   &= - J \sum_{\expval{ij} }^{} \expval{S_i S_j}  + \Delta \sum_{i}^{} \expval{S_i^2} - N \Delta      \\
& \overset{MF}{\simeq } - J \sum_{\expval{ij} }^{} \expval{S_i} \expval{S_j} + \Delta \sum_{i}^{} \expval{S_i^2} - \Delta N
\end{split}
\end{equation}
where
\begin{equation}
  \expval{S_i} = \expval{S_j} \equiv m
\end{equation}
Hence,
\begin{equation}
  G(T,J, \Delta )_{MF}  = - \frac{1}{2} N J z \qty( \Tr_{S_i}(\rho _i S_i) )^2
  + N \Delta \Tr_{S_i}(\rho _i S_i^2) - N \Delta +
  N k_B T \Tr_{S_i}(\rho _i \ln{\rho _i} )
\end{equation}
We noe minimize \( G(T,J,\Delta )_{MF} \) with respect to the function \( \rho _i \) with constraint \( \Tr_{S_i}(\rho _i) = 1  \):
\begin{equation}
  \dv{G}{\rho _i} = 0
\end{equation}
Let us consider each term
\begin{subequations}
\begin{align}
  \dv{}{p_i} \qty(\Tr(\rho _i S_i) )^2& = 2 \qty(\Tr(\rho _i S_i) ) S_i = 2 \expval{S_i} S_i = 2 m S_i \\
    \dv{}{p_i} \qty(\Tr(\rho _i S_i^2) ) &= S_i^2 \\
      \dv{}{p_i} \qty(\Tr(\rho _i \ln{\rho _i} ) ) &=  \ln{\rho _i }  +1
\end{align}
\end{subequations}
\begin{remark} Remind that
\( \rho _i = \rho ^{(1)} (S_i) \).
\end{remark}
\begin{equation}
  0 = - J N z m S_i + \Delta N S_i^2 + N k_B T \ln{\rho _i} + N k_B T
\end{equation}
Dividing by \( N k_B T \),
\begin{equation}
  \ln{\rho _i} \equiv \ln{\rho ^{(1)} (S_i)} =   \beta J z m S_i - \beta \Delta S_i^2 - 1
\end{equation}
which implies
\begin{equation}
  \rho ^{(1)} (S_i) = A^{-1} e^{\beta (zJmS_i - \Delta S_i^2)}
\end{equation}
\begin{remark}
In \( A^{-1} \) it is included the term \( e^{-1}  \).
\end{remark}
The constant \( A \) can be found by imposing \( \Tr_{S_i}\rho ^{(1)}(S_i) =1 \), (TO DO)
\begin{equation}
  A = 1 + 2 e^{-\beta \Delta } \cosh (\beta z J m)
\end{equation}
Given \( \rho ^{(1)}(S_i) \) it is possible to show (TO DO)
\begin{equation}
  \expval{S_i^2} = \Tr_{S_i}(\rho _i S_i^2) = \frac{1}{A} 2 e^{-\beta \Delta } \cosh ( \beta z J m)
\end{equation}
and
\begin{equation}
  x = 1 - \expval{S_i^2} = \frac{A - 2 e^{-\beta \Delta } \cosh (\beta z J m) }{A} \quad \Rightarrow x = \frac{1}{A}
\end{equation}
Hence,
\begin{equation}
  \frac{G(T,\Delta ,m,J)}{N} = \frac{z}{2} J m^2 - \Delta - k_B T \ln{A}
\end{equation}
\begin{remark}
Now we should minimize \( G(T,\Delta ,m,J) \) with respect to \( m \) to obtain the equilibrium phases.
\end{remark}
The expansion for small values of \( m \) is
\begin{equation}
  \cosh (t) = 1 + \frac{t^2}{2} + \frac{t^4}{24}, \quad \ln{(1+t)} = t - \frac{t^2}{2}
\end{equation}
(TO DO)
\begin{equation}
  G (T, \Delta , J, m) = a_0 (T, \Delta ) + a (T, \Delta ) m^2 + b (T,\Delta )m^4 + \frac{c(T, \Delta )}{6} m^6
\end{equation}
where
\begin{equation}
  \begin{cases}
   a (T, \Delta ) = \frac{z J}{2} \qty(1 - \frac{z J}{\delta k_B T}) \\
   b (T, \Delta ) = k \qty(1 - \frac{\delta }{3}) \\
    c (T, \Delta ) > 0
  \end{cases}
\end{equation}
and the parameter
\begin{equation}
  \delta \equiv  1 + \frac{e^{\beta \Delta } }{2} = \delta (T,\Delta )
\end{equation}
is related to the concentration of \( \text{He}^3-x \).
This can be seen as follows.

Since
\begin{equation}
  x (T, \Delta , J) \equiv  1 - \expval{S_i^2} = \frac{1}{A} = \qty(1 + 2 e^{-\beta \Delta }  \cosh (\beta z J m))^{-1}
\end{equation}
In the disordered phase \( (m=0) \) one has
\begin{equation}
  x (T,\Delta , J) = \qty(1 + 2 e^{-\beta \Delta } )^{-1} = \frac{\delta -1}{\delta }
\end{equation}
By combining this result with the order-disorder transition
\begin{equation}
  a (T_c (\Delta )) = \frac{z J}{2} \qty(1 - \frac{z J}{\delta k_B T_c}) = 0, \quad \qty[\delta = \frac{z J}{k_B T_c}]
\end{equation}
one obtains (TO DO)
\begin{equation}
  T_c (x) = T_c (0) (1-x)
\end{equation}
There is a dependence of the critical temperature \( \lambda  \) on the \( \text{He}^3 \) concentration \( x \).

The tricritical point is the one that satisfies the conditions
  \begin{equation}
    \begin{cases}
     a (T_t, \Delta _t) = 0 \\
     b (T_t, \Delta _t) = 0
    \end{cases} \Rightarrow
    \begin{cases}
      \delta _t = \frac{zJ}{k_B T_t} \\
      \delta _t = 3
    \end{cases}
\end{equation}
\begin{equation}
  x (T_t, \Delta _t) = \frac{\delta _t - 1}{\delta _t} = \frac{2}{3}
\end{equation}
\begin{remark}
Experimental estimate of \( x_t \) is \( \sim 0.67 \).
\end{remark}


\begin{exercise}
Expand the free-energy per site
\begin{equation}
  \frac{G}{N} = \frac{z}{2} J m^2 - \Delta - k_B T \ln{A}
\end{equation}
where \( A = 1 + 2 e^{-\beta \Delta } \cosh(\beta z J m) \) for small values of \( m \).
\begin{equation}
  x \equiv \beta z J m, \quad B \equiv 2 e^{-\beta \Delta }
\end{equation}
Since \( \cosh x \simeq 1 + \frac{x^2}{2} + \frac{x^4}{24} \),
\begin{equation}
  A = 1 + B \cosh x \simeq 1 + B \qty(1 + \frac{x^2}{2} + \frac{x^4}{24})
\end{equation}
\begin{equation}
\begin{split}
  \ln{A} &= \ln{\qty(1 + B + \frac{B x^2}{2} + \frac{B x^4}{24}) } \\
  &\simeq
  \ln{\qty[(1+B)\qty(1+ \frac{B}{2(1+B)}x^2 + \frac{B}{24(1+B)}x^4) ] } \\
  & = \ln{(1+B)} + \ln{(1+t)}
\end{split}
\end{equation}
where
\begin{equation}
  t \equiv \frac{B}{2(1+B)}x^2 + \frac{B}{24(1+B)}x^4
\end{equation}
Let us first consider the term
\begin{equation}
  \frac{B}{1+B} = \frac{2 e^{-\beta \Delta } }{1 + 2 e^{-\beta \Delta } } = \frac{2}{2+ e^{\beta \Delta } } = \delta ^{-1}
\end{equation}
\begin{equation}
  \Rightarrow \ln{A} = \ln{(1+B)} + \frac{x^2}{2 \delta } + \qty(\frac{1}{24 \delta } - \frac{1}{8 \delta ^2})x^4 - \frac{1}{24 \delta ^2}x^6
\end{equation}
\begin{remark}
We have that \( x \equiv \beta z J m \).
\end{remark}
\begin{equation}
\begin{split}
  - \frac{\ln{A} }{B} + \frac{z}{2} J m^2 - \Delta \simeq & a_0 (T,\Delta )
  + \qty(\frac{z}{2}J - \frac{\beta z^2 J^2}{2 \delta })m^2 \\
  &+ \qty(\frac{1}{8 \delta }- \frac{1}{24 \delta })\beta ^3 z^4 J^4 m^4
  + \frac{1}{24 \delta ^2} \beta ^5 z^6 J^6 m^6
\end{split}
\end{equation}
\begin{equation}
  G (T, \Delta , J, m) = a_0 (T,\Delta ) + a(T, \Delta )m^2 + b (T, \Delta ) m^4 + c(T, \Delta )m^6
\end{equation}
where
\begin{subequations}
\begin{align}
  a(T,\Delta ) &= \frac{z J}{2} \qty(1 - \frac{\beta z J}{\delta })   \\
  b(T,\Delta ) &= \frac{\beta ^3 z^4 J^4}{8 \delta } \qty(\frac{1}{\delta } - \frac{1}{3}) = \frac{\beta ^3 z^4 J^4}{8 \delta^2 } \qty(1 - \frac{\delta }{3})    \\
  c(T,\Delta ) &= \frac{\beta ^5 z^6 J^6}{24 \delta^2 } > 0
\end{align}
\end{subequations}

\end{exercise}


\subsection{Mean field again}
Another way to introduce the variational approach and the mean field approximation often discussed starts from the general expression of the variational free energy
\begin{equation}
  F_{var} = \expval{\mathcal{H}}_{\rho _{TR}} + k_B T \expval{\ln{\rho _{TR}} }_{\rho _{TR}}
\end{equation}
If one assumes that the family of trial distribution is of the Gibbs-Boltzmann form
\begin{equation}
  \rho _{TR} = \frac{e^{- \beta \mathcal{H}_{TR}} }{Z_{TR}}
\end{equation}
with
\begin{equation}
  Z_{TR} = e^{-\beta F_{TR}} = \sum_{\{ \Phi _i \}  }^{} e^{-\beta \mathcal{H}_{TR} ( \{ \Phi _i \}  )}
\end{equation}
then, since
\begin{equation}
  \ln{\rho _{TR}} = - \beta \mathcal{H}_{TR} - \ln{Z_{TR}}
\end{equation}
we have
\begin{equation}
  k_B T \expval{\ln{\rho _{TR}} }_{\rho _{TR}} = k_B T \expval{\frac{- \mathcal{H}_{TR}}{k_B T}} + k_B T \underbrace{\expval{- \ln{Z_{TR}} } }_{\beta F_{TR}}
\end{equation}
\begin{equation}
  k_B T \expval{\ln{\rho _{TR}} }_{\rho _{TR}} = \expval{- \mathcal{H}_{TR}} + F_{TR}
\end{equation}
\begin{equation}
  \Rightarrow F_{var} = \expval{\mathcal{H}}_{\rho _{TR}} - \expval{\mathcal{H}_{TR}}_{\rho _{TR}}  + F_{TR}
  = \expval{\mathcal{H}- \mathcal{H}_{TR}}_{\rho _{TR}} + F_{TR}
\end{equation}
Clearly, \( F \le F_{var} \) and one has to look for the minima of \( F_{var} \) by varying \( \rho _{TR} \).
Within this approach, the  mean field approximation is still given by
\begin{equation}
  \rho _{TR}^{MF} (\Phi _1, \dots, \Phi _N) = \prod_{i=1}^{N} \rho _{TR}^{(1)} (\Phi _i)
\end{equation}
that in this case becomes
\begin{equation}
  \prod_{i}^{} \rho _{TR}^{(1)} (\Phi _i) = \frac{1}{Z_{TR}^{MF}} e^{-\beta \sum_{i}^{} b_i \Phi _i  }
\end{equation}
and
\begin{equation}
  Z_{TR} = \sum_{\{ \Phi  \}  }^{}  e^{-\beta \sum_{i}^{} b_i \Phi _i  }
\end{equation}
where \( b_i \) are the variational parameters.
\begin{equation}
  \mathcal{H}_{TR} = - \sum_{i}^{} b_i \Phi _i
\end{equation}
If we consider again the Ising model
\begin{equation}
  \mathcal{H} = - J \sum_{\expval{ij} }^{} S_i S_j - H \sum_{i}^{} S_i
\end{equation}
\begin{equation}
\begin{split}
F_{var}  &= \expval{\mathcal{H} - \mathcal{H}_{TR}}_{\rho _{TR}} + F_{TR}   \\
& = F_{TR} + \expval{\qty(- J \sum_{\expval{ij} }^{} S_i S_j - H \sum_{i}^{} S_i    ) - \qty(- \sum_{i}^{} b_i S_i )  }_{\rho _{TR}} \\
& = F_{TR} + \expval{-J \sum_{\expval{ij} }^{} S_i S_j + \sum_{i}^{} (b_i-H) S_i   }_{\rho _{TR}} \\
& = F_{TR} - J \sum_{\expval{ij} }^{} \expval{S_i S_j}_{\rho _{TR}} + \sum_{i}^{} (b_i - H) \expval{S_i}_{\rho _{TR}}
\end{split}
\end{equation}
Since \( \rho _{TR} = \prod_{i}^{} \rho _i  \),
\begin{equation}
  \expval{S_i S_j}_{\rho _{TR}}  = \expval{S_i}_{\rho _{TR}}  \expval{S_j}_{\rho _{TR}}
\end{equation}
\begin{equation}
  F_{var} = F_{TR} - J \sum_{\expval{ij} }^{} \expval{S_i}_{\rho _{TR}}  \expval{S_j}_{\rho _{TR}}  + \sum_{i}^{} (b_i - H) \expval{S_i}_{\rho _{TR}}
\end{equation}
\begin{equation}
  \pdv{F_{var}}{b_i} = 0, \quad \forall i
\end{equation}
\begin{equation}
  0 = \pdv{F_{var}}{b_i} = \qty[-J \sum_{j \, n.n. \, i}^{} \expval{S_i}_{\rho _{TR}}  + b_i - H  ] \pdv{\expval{S_i} }{b_i}
\end{equation}
or
\begin{equation}
  b_i = J \sum_{j \, n.n. \, i}^{} \expval{S_j}_{\rho _{TR}} + H
\end{equation}
\begin{equation}
\begin{split}
\expval{S_i}_{\rho _{TR}}    &= \frac{1}{Z_{TR}} \sum_{\{ S \}  }^{} S_i e^{\beta \sum_{k}^{} S_k b_k } = \frac{\prod_{k}^{}  \sum_{S_k}^{} S_i e^{\beta S_k b_k}   }{\prod_{k}^{} \sum_{S_k}^{} e^{\beta S_k b_k}    }  \\
& = \frac{\sum_{S_i}^{} S_i e^{\beta S_i b_i}  }{\sum_{S_i = \pm 1}^{} e^{\beta S_i b_i}   } = \frac{\cosh (\beta b_i)}{\sinh (\beta b_i)} \\
&= \tanh (\beta b_i)
\end{split}
\end{equation}
\begin{equation}
  b_i = J \sum_{j \, n.n. \, i}^{} \tanh (\beta b_j) + H
\end{equation}
\begin{exercise}
Consider again the antiferromagnetic Ising model
\begin{equation}
  \mathcal{H}[\{ S \}  ] = - J \sum_{\expval{\va{r}_A \va{r}_B} }^{} S (\va{r}_A) S(\va{r}_B) - H \sum_{\va{r}_A}^{} S (\va{r}_A) + H \sum_{\va{r}_B}^{} S (\va{r}_B)
\end{equation}
where \( J>0 \) and \( H>0 \).
\begin{itemize}
\item \( \va{r}_A \) denotes the site on the \( A \) sublattice.
\item \( \va{r}_B \) denotes the site on the \( B \) sublattice.
\end{itemize}

Let us find again the mean-field solution but now using the variational ansatz
\begin{equation}
  F \le F_{var} = \expval{\mathcal{H}}_{\rho _{TR}} - \expval{\mathcal{H}_{TR} }_{\rho _{TR}}  + F_{TR} = \expval{\mathcal{H} - \mathcal{H}_{TR}}_{\rho _{TR}} + F_{TR}
\end{equation}
\begin{remark}
  Since the problem can be splitted in two sublattices, it is convenient to use
  \begin{equation}
    \mathcal{H}_{TR} = - H_A \sum_{r_A}^{} S(r_A) - H_B \sum_{r_B}^{}  S(r_B)
  \end{equation}
\end{remark}

\begin{itemize}
\item Show that \( F_{var} \) has the following expression:
\begin{equation}
\begin{split}
F_{var}  = &  F_{TR} (\beta H_A, \beta H_B)
- 4 NJ \expval{S_A}_{\rho _{TR}}  \expval{S_B}_{\rho _{TR}} \\
 &- \frac{1}{2} NH \qty(\expval{S_A}_{\rho _{TR}}
 - \expval{S_B}_{\rho _{TR}}   )
 + \frac{1}{2} N \qty(H_A \expval{S_A}_{\rho _{TR}}  + \expval{S_B}_{\rho _{TR}}   )
\end{split}
\end{equation}

where
\begin{subequations}
\begin{align}
   \expval{S_A}_{\rho _{TR}}  & \equiv m_A + n \\
    \expval{S_B}_{\rho _{TR}}  & \equiv m_B - n
\end{align}
\end{subequations}
with \( m=m_A+m_B \), and
\begin{subequations}
\begin{align}
   m_A &= \tanh ( \beta H - 4 \beta J m_B) \\
    m_B &= \tanh ( \beta H - 4 \beta J m_A)
\end{align}
\end{subequations}
\item Expand the free energy \( F_{var} \) in powrs of \( m \) of the form
\begin{equation}
  F_{var} = A + B m^2 + c m^4 + O (m^6)
\end{equation}
and find the explicit expression of \( A,B \) and \( C \) as a function of \( T,H \) and \( n \).
\end{itemize}

\end{exercise}





\end{document}
