\documentclass[../main/main.tex]{subfiles}

\newdate{date}{18}{10}{2019}

\begin{document}

\marginpar{ \textbf{Lecture 4.} \\  \displaydate{date}. \\ Compiled:  \today.}

\section{Statistical mechanics of ensambles}

Statistical mechanics roughly speaking was born as a sort of theory from microscopic and try to compute the macroscopic lenght using thermodynamics. Problems: go from the counting problem to the macroscopic problem.
In origin was statistical mechanics of equilibrium system.
Each microstate with a given energy fixed, will have the same probability, this is the equal probability statement.

In general, if we consider a system with \emph{N,V} fixed and also \emph{E} fixed, we call \( \Omega (E,V,N) \)  the number of microstate with given energy \emph{E} , volume \emph{V} and number of particles \emph{N}.
We call  a single configuration of a given microstate \emph{C}. A configuration is just when you have the spatial part, because momentum can be obtained by integrating.
Suppose you want to compute the probability of a given configuration \emph{C}, \( P_C \), because there is equal probability:
\begin{equation}
  P_C = \frac{1}{\Omega (E,V,N)}
  \label{eq:}
\end{equation}
Now, let the energy to be free to fluctuate in time at fixed temperature (isotherma ensambles). For doing that, consider a system and insert it in a thermal bath (Figure 1), where \( N_T = N_B + N\)  with \( N_B \gg N \), where \( N_B \) are the particles in the thermal bath and \( N \) the particle of the system. We have also \( V_B \gg V \) and a fixed \( T_B \).
What is fixed now is the temperature of the bath \( T_B \) , the number of the total number of the system \( N_T \), and also the total volume is fixed.

Suppose to consider a \emph{weak} interaction between the particles of the bath and of the system (IMPORTANT): since the total system as total energy \( E_T = E_B + E_C\), where \( E_C \) is the energy of a given configuration. For the system we consider \( E_T \) is fixed, so we are in the microcanonical ensamble for the whole system: \( \Omega (E_T,V_T,N_T) \). We see that the probability of a given configuration is related to the number of microstate of the bath:
\begin{equation}
  P_C \propto \Omega _B (E_T - E_C) \sim k_B \ln{\Omega _B (E_T - E_C)} = S_B (E_T - E_C)
  \label{eq:}
\end{equation}
what we get by expanding considering the huge number of particles
\begin{equation}
    P_C \propto e^{-\frac{E_C}{k_B T_{\cancel{B} }}} = e^{- \beta E_C}   \qquad \beta \equiv \frac{1}{k_B T}
  \label{eq:}
\end{equation}
Note that we can consider a general \( T \), because the temperature is fixed.
This is the canonical ensamble, this probability has to be normilized. Therefore we divide by the normalization factor that consists on the sum of all configurations:
\begin{equation}
  P_C = \frac{e^{-\beta E_C} }{\sum_{C}^{} e^{-\beta E_C}   }
  \label{eq:}
\end{equation}
Now, we called \( Q (T,V,N) \) the isothermal canonical partition function:
\begin{equation}
  Q (T,V,N) \equiv \sum_{C}^{} e^{-\beta E_C}
  \label{eq:}
\end{equation}
this is the sum over all configuration with given \( \{V,N\} \) fixed.
We define the Heldmonds free energy as:
\begin{equation}
  A(T,V,N) = -k_B T \ln{Q(T,V,N)}
  \label{eq:}
\end{equation}
This is actually in not to thermodynamic  limit, remember that thermodynamic assume that the number of number of freedom is related to the number of avogadro.
Now, we do a \emph{foliation} in energy of the space, that is a sum over the energy (keeping \( \{V,N\} \) fixed):
\begin{equation}
  Q(T,V,N) = \sum_{E}^{} e^{-\beta E} \Omega (E,V,N) = \sum_{E}^{} e^{-\beta E} e^{S/k_B} = \sum_{E}^{} e^{-\beta (E-T S)}
  \label{eq:}
\end{equation}
This is the Legendre transform in thermodynamic theory. The idea is now: consider the same system with the bath. Now the system can exchange energy but also volume (we continue to keep the temperature of the bath fixed). At this point the ensamble is \emph{isothermal} and \emph{isobaric}. All the assumputions done before are valid and we have also \( V_T = V_B + V \):
\begin{equation}
  P_C \propto \Omega _B (E_T-E_C, V_T-V_C) \sim k_B \ln{\Omega _B (E_T-E_C, V_T-V_C)}
  \label{eq:}
\end{equation}
At this point the only quantity fixed is \emph{N}:
\begin{equation}
  P_C = \frac{e^{-\frac{1}{k_B T} (E_C + P V_C)} }{\sum_{\substack{ C \\  \text{\emph{N} fixed } } }^{} e^{-\beta (E_C + P V_C)}   } \qquad \substack{T_B \equiv T \\ P_B \equiv P}
  \label{eq:}
\end{equation}
This partition function is calle the \emph{Gibbs partition function}:
\begin{equation}
  \begin{cases}
   \Delta (T,P,N) \equiv \sum_{C}^{} e^{-\beta (E(C) + P V(C))}   \\
   G(T,P,N) = -k_B T \ln{\Delta (T,P,N)}
  \end{cases}
\label{eq:}
\end{equation}
We have the laplace transform
\begin{equation}
  \Delta (T,P,N) = \sum_{V}^{} e^{-\beta P V} (\sum_{\substack{ C \\  \text{\emph{V} fixed } } }^{} e^{-\beta E(C)}  )  = \sum_{V}^{} e^{-\beta P V} Q(T,V,N)
  \label{eq:}
\end{equation}
We can rewrite and view that the entropy of the two partition function is the same. Therefore:
\begin{equation}
  \Delta (T,P,N) = \int_{0}^{\infty } \dd[]{V}  e^{-\beta V P \qty[\frac{1}{h^{3N} N!} e^{-\beta \mathcal{H} (p,n)} ] \dd{\va{r_1} } \dots \dd{\va{r_N} } \dd{\va{p_1} } \dots \dd{\va{p_N} } }
  \label{eq:}
\end{equation}
\begin{equation}
  P_C = \frac{e^{-\beta E_C + \beta \mu N} }{ \sum_{C}^{} e^{-\beta E_C + \beta \mu N} }
  \label{eq:}
\end{equation}

In principle, if one is able to compute the partition function is able to compute the thermodynamic quantitites.
Suppose a system \( \Omega  \) carachterzied by \( V(\Omega ) \) and that have a boundary \( \partial{\Omega }  \) carachterized by \( S(\Omega ) \) (Figure 2).
\emph{L} is the characteristic lenght, and we have \( V(\Omega ) \propto L^d \) and \( S(\Omega) \propto L^{d-1} \)  in \emph{d} dimension:
\begin{equation}
  \mathcal{H}_{\Omega } (C) = - \sum_{n}^{} \underbrace{ k_n \Theta _n (C)}_{\text{must satisfy symmetry}}
  \label{eq:}
\end{equation}
it is important that in principle the term satisfies the symmetry of the system.
This is a master rule!

For instance, consider the magnetic system. In a bravais lattice we put a spin up (+1) or down (-1). We can define the vector \( \va{S_i}  \) as the \emph{spin at i-esim site} (Figure 3) with \( 1 \le i \le N(\Omega ) \).
A configuration is the orientation of the spin in each site \( C = \{ \va{S_1},\dots,\va{S_N} \}  \).
In this case
\begin{subequations}
\begin{align}
  k_1 \Theta _1 (C) &= k_1 \sum_{i}^{} \va{S_i}  \\
  k_2  \Theta _2 (C) &= \frac{1}{2} \sum_{ij}^{} \va{S_i} \vdot \va{S_j} k_2 (i,j)
\end{align}
\label{}
\end{subequations}
Now, we sum over all configurations, but first of all define the trace operation:
\begin{equation}
  \Tr \equiv \sum_{\{C\}}^{} = \sum_{\va{S_1} }^{} \sum_{\va{S_2} }^{}  \dots \sum_{\va{S_N} }^{}
  \label{eq:}
\end{equation}
\begin{equation}
  Q_{\Omega } (T, \{k_n\}) = \Tr(e^{-\beta \mathcal{H}(C)} )
  \label{eq:}
\end{equation}
All the configuration are \( \{C\} = \{ (\va{r_i},\va{p_i}  )_{i=1,\dots,N}\} \)
we have:
\begin{equation}
  \Theta _1 (C) = \sum_{i}^{} \qty[\frac{\va{p_i}^2 }{2m}+ U_1 (\{\va{r}_i \})]
  \label{eq:}
\end{equation}
\begin{equation}
  \Theta _2 (C) = \frac{1}{2} \sum_{ij}^{} U ( \abs{\va{r_i} - \va{r_j}  } )
  \label{eq:}
\end{equation}
We have:
\begin{equation}
  \Tr \equiv \sum_{\{C\}}^{} = \sum_{N=0}^{\infty } \frac{1}{N!} \int_{}^{} product_{i=1}^{N} \frac{\dd[]{\va{r_i} \dd[]{\va{p_i} } } }{h^{3N}}
  \label{eq:}
\end{equation}
\begin{equation}
  \mathcal{L} = \Tr(e^{-\beta (\mathcal{H}_\Omega - \mu N)} )
  \label{eq:}
\end{equation}
Let us write:
\begin{equation}
 F_\Omega [T, \{k_n\}]= -k_B \ln{Z_N} (T,\qty{k_n} )
  \label{eq:}
\end{equation}
with \( F_\Omega \propto V(\Omega ) \sim L^d \)
In general, we can write:
\begin{equation}
  F_\Omega = V (\Omega ) \underbrace{f_b }_{\substack{ \text{bulk} \\  \text{free energy } \\ \text{density}} } + S (\Omega ) f_s + O (L^{d-2})
  \label{eq:}
\end{equation}
We have:
\begin{equation}
  f_b [T,\qty{k_n} ] = \lim_{V(\Omega )\rightarrow \infty } \frac{F_\Omega (T, \qty{k_n} )}{V(\Omega )}
  \label{eq:}
\end{equation}
\begin{equation}
  f_s [T,\qty{k_n} ] = \lim_{V(\Omega )\rightarrow \infty } \frac{F_\Omega - V (\Omega ) f_b}{S(\Omega )}
  \label{eq:}
\end{equation}
Therefore, spin
\begin{equation}
  f_b = \lim_{N(\Omega)\rightarrow \infty } \frac{F_N}{N} \qquad \substack{V \rightarrow \infty  \\ N \rightarrow \infty } \quad \rho \quad \text{fixed}
  \label{eq:}
\end{equation}
One interesting things is to prove the existence of the limit. Given that, the questions is: can we describe the thermodynamics singularities underneat the phase transition? This is not completely clear. The partition function of a given \( \Omega  \) is an analitic function (it converges):
\begin{equation}
  Z (\Omega ) = \Tr(e^{-\beta \mathcal{H}_\Omega} )
  \label{eq:}
\end{equation}
There is no way in you can produce singularities out of this. The singularities will develop in the thermodynamic limits.
The next question is the following: there are singularities that come out in the thermodynamic limit, for reach singulariities you have to reach so precision in thermodynamic that you are not able to go extactly into the critical point. Next: how you can relate singularities in the behaviour of the system geometrically?












\end{document}
