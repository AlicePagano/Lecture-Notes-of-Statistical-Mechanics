\documentclass[../main/main.tex]{subfiles}

\newdate{date}{18}{12}{2019}


\begin{document}

\marginpar{ \textbf{Lecture 19.} \\  \displaydate{date}. \\ Compiled:  \today.}

\section{Scaling theory}
It is used whenever you have a collective behaviour. The length scale of the problem are \( a,L,\xi  \), but  \( \xi  \) is the only relevant length scale in the problem.

Which are the experimental data which gives us this ideas? What you can see from experiment is
figure 1

Widom \( \rightarrow  \) static scaling theory \( \rightarrow  \) homogeneous functions.

\subsection{Single variable \( r \) }
\( f(r) \) is homogeneous in \( r \), if \( \forall  \lambda  \in \R \) we have
\( f( \lambda r) = \lambda f(r) \). More general,
\begin{equation}
  f( \lambda r) = g (\lambda) f(r)
\end{equation}
\begin{example}
\begin{equation}
  f(r) = B r^2
\end{equation}
\begin{equation}
  f ( \lambda r) = B ( \lambda r)^2 = \lambda ^2 f (r) \quad \Rightarrow g ( \lambda ) = \lambda ^2
\end{equation}
\end{example}
Sine it is valid for any \( \lambda  \), you can choose also a \( \lambda  \) in that way
\begin{equation}
  f(r) = f (\lambda r_0) = g (\lambda ) f (r_0)
\end{equation}
\begin{theorem}[]
\begin{equation}
  g(\lambda ) = \lambda ^p
\end{equation}
where \( p \) is the degree of the homogeneity of the function.
\end{theorem}
We can make it for any variable, not only for a single one.

\subsection{Generalized homogeneous functions}
We are discussing \( f(x,y) \), that is a generalized homogeneous function if \( f (\lambda ^a x, \lambda ^b y) = \lambda f(x,y)\). In general any polynomial is a generalized homogeneous function.
If we choose \( \lambda ^p \equiv s \), we have
\begin{equation}
  f ( s^{a/p} x, s^{b/p} y) = s f(x,y)
\end{equation}

Consider \( t \equiv  \frac{T - T_c}{T_c}, h = \frac{H - H_c}{H_c} \)
\begin{equation}
  f (T,H) = f_{ANA} (T,H) + f_{SING} (t,h)
\end{equation}
where \( f_{ANA} \) is an analytic term and \( f_{SING} \) diverges, has a singularity.

The singular part of the free energy
\begin{equation}
  f_s ( \lambda ^{p_1} t, \lambda ^{p_2} h) = \lambda f_s (t,h)
\end{equation}
where \( \forall \lambda \in \R \).

Another important feature, choose \( \lambda  \) as
\begin{equation}
  \lambda = h^{-1/p_2} \quad \Rightarrow f_s (t,h) = h^{1/p_2} f_s (h^{-p_1/p_2} t , 1)
\end{equation}
\begin{equation}
  \Delta  \equiv \frac{p_1}{p_2}
\end{equation}
is called the \emph{gap exponent}.

\( M \) is the first derivative of \( f \) with respect to \( H \).
\begin{equation}
  \lambda ^{p_2} M_s ( \lambda ^{p_1} t, \lambda ^{p_2} h) = \lambda M_s (t,h)
\end{equation}
so you have the same story for the magnetization.
Consider \( h=0 \) and \( t \rightarrow 0^- \), we have
\begin{equation}
  M_s (t) \sim (-t)^{\beta }
\end{equation}
starting from this one try to figure out what is happening at this level.
\begin{equation}
  M_s (t,0) = \lambda ^{p_2 -1} M_s ( \lambda ^{p_1} t,0)
\end{equation}
\begin{equation}
  \lambda ^{p_1} t = -1 \quad \Rightarrow \lambda = (t)^{-1/p_1}
\end{equation}
so
\begin{equation}
  M_s (t,0) = - (t)^{\frac{1-p_2}{p_1}} M_s (-1,0)
\end{equation}
\begin{equation}
  \beta = \frac{1-p_2}{p_1}
\end{equation}
For \( \delta  \), we have  \( T=T_c \) and \( h \rightarrow 0^+ \), so the magnetization goes like \( M_s \sim h^{1/\delta } \).
\begin{equation}
  M_s (0,h) = \lambda ^{p_2 - 1} M_s (0,\lambda ^{p_2} h)
\end{equation}
Now we want
\begin{equation}
  \lambda ^{p_1} h = 1 \quad \Rightarrow \lambda = h ^{-1/p_2}
\end{equation}
\begin{equation}
  M_s (0,h) = h ^{\frac{1-p_2}{p_2}} M_s (0,+1)
\end{equation}
\begin{equation}
  \delta = \frac{p_2}{ 1 - p_2 }
\end{equation}
From this you have a very simple relation
\begin{equation}
  p_1 = \frac{1}{\beta (\delta +1)}, \quad p_2 = \frac{\delta }{\delta + 1}
\end{equation}










\end{document}
