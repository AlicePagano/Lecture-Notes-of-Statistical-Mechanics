\documentclass[../main/main.tex]{subfiles}

\newdate{date}{06}{11}{2019}


\begin{document}

\marginpar{ \textbf{Lecture 8.} \\  \displaydate{date}. \\ Compiled:  \today.}
\noindent Since \( \sum_{k}^{} \lambda _k^N = Z_N \) for \( k=+,-,1,\dots,n \):
\begin{equation}
  \expval{S_1 S_R}_N = \frac{\sum_{ij}^{} \bra{t_j} \mathbb{S}_1 \ket{t_i} \lambda _i^{R-1} \bra{t_i} \mathbb{S}_R \ket{t_j} \lambda _j^{N-R+1}      }{\sum_{k=1}^{n} \lambda _k^N  }
\end{equation}
If we now multiply and divide by \( \lambda _+^N \), we get
\begin{equation}
  \expval{S_1 S_R}_N = \frac{\sum_{ij}^{} \bra{t_j} \mathbb{S}_1 \ket{t_i} (\lambda _i/ \lambda _+)^{R-1}  \bra{t_i} \mathbb{S}_R \ket{t_j} (\lambda _j / \lambda _+)^{N-R+1}    }{\sum_{k=1}^{n} (\lambda _k  / \lambda _+)^{N} }
\end{equation}
\begin{remark}
In the thermodynamic limit \( N \rightarrow \infty  \), only the terms with \( j=+ \) and \( k=+ \) survive in the sum. Remind that \( R \) is fixed.
\end{remark}
\begin{equation}
\expval{S_1 S_R}_N =   \lim_{N \rightarrow \infty } \expval{S_1 S_R}_N = \sum_{i= \pm,1 \dots n}^{} \qty(\frac{\lambda _i}{\lambda _+})^{R-1} \bra{t_+} \mathbb{S}_1 \ket{t_i} \bra{t_i} \mathbb{S}_R \ket{t_+}
\end{equation}
Rembember that \( \lambda _+ > \lambda _T \ge \lambda _1 \dots \lambda _n \):
\begin{equation}
  \expval{S_1 S_R}_N = \bra{t_+} \mathbb{S}_1 \ket{t_+} \bra{t_+} \mathbb{S}_R \ket{t_+} +   \sum_{i \neq +}^{n} \qty(\frac{\lambda _i}{\lambda _+})^{R-1} \bra{t_+} \mathbb{S}_1 \ket{t_i} \bra{t_i} \mathbb{S}_R \ket{t_+}
\end{equation}
Since one can prove, by a method entirely analogous to that followed above, that
\begin{equation}
  \lim_{N \rightarrow \infty } \expval{S_R}_N = \bra{t_+} \mathbb{S}_R \ket{t_+}
\end{equation}
we obtain
\begin{equation}
  \expval{S_1 S_R} = \expval{S_1} \expval{S_R} + \sum_{i \neq +}^{}  \qty(\frac{\lambda _i}{\lambda _+})^{R-1} \bra{t_+} \mathbb{S}_1 \ket{t_i} \bra{t_i} \mathbb{S}_R \ket{t_+}
\end{equation}
The correlation function then follows immediately as
\begin{equation}
\Gamma _R =   \expval{S_1 S_R} - \expval{S_1}\expval{S_R} = \sum_{i \neq +}^{n} \qty(\frac{\lambda _i}{\lambda _+})^{R-1} \bra{t_+} \mathbb{S}_1 \ket{t_i} \bra{t_i} \mathbb{S}_R \ket{t_+}
\end{equation}
\begin{remark}
\( \Gamma_R \) depends only on the eigenvalues and eigenvectors of the transfer matrix \( \mathbb{T} \) and by the values of the spins \( S_1 \) and \( S_R \).
\end{remark}
A much simpler formula is obtained for the correlation length \eqref{eq:7_2}. Taking the limit \( R \rightarrow \infty  \) the ratio \( (\lambda _-/ \lambda _+) \) dominates the sum and hence
\begin{equation}
\begin{split}
\xi ^{-1} &=  \lim_{R \rightarrow \infty } \qty{-\frac{1}{R-1} \log{\qty|\expval{S_1S_R} - \expval{S_1} \expval{S_R}   |} } \\
& = \lim_{R \rightarrow \infty }  \qty{-\frac{1}{R-1} \log{\qty[  \qty(\frac{\lambda _-}{\lambda _+})^{R-1} \bra{t_+} \mathbb{S}_1 \ketbra{t_-}{t_-} \mathbb{S}_R \ket{t_+}  ]  } } \\
& = -\log{\qty[\qty(\frac{\lambda _-}{\lambda _+}) ] } - \lim_{R \rightarrow \infty } \frac{1}{R-1} \log{ \bra{t_+} \mathbb{S}_1 \ketbra{t_-}{t_+} \mathbb{S}_R \ket{t_+}    }  \\
& = - \log{\qty(\frac{\lambda _-}{\lambda _+}) }
\end{split}
\end{equation}
The important result is
\begin{empheq}[box=\myyellowbox]{equation}
  \xi ^{-1} = -  \log{\qty(\frac{\lambda _-}{\lambda _+}) }
\end{empheq}
It means that the correlation length does depend only on the ratio between the two largest eigenvalues of the transfer matrix \( \mathbb{T} \).


\subsection{Results for the \emph{1-dimensional} Ising model}
The transfer matrix is given by
\begin{equation}
  \mathbb{T} =
  \begin{pmatrix}
  \exp (K+h)     & \exp (-K)  \\
  \exp (-K)    & \exp (K-h)
  \end{pmatrix}
\end{equation}
Calculate the eigenvalues:
\begin{equation}
  \abs{\mathbb{T}-\lambda \mathbb{1}} =  (e^{K+h}- \lambda  )(e^{K-h}-\lambda  )-e^{-2K}=0
\end{equation}
The solutions are
\begin{equation}
  \lambda _{\pm} = e^{K} \cosh(h) \pm \sqrt{e^{2K}\sinh^2 (h)+e^{-2K} }
\end{equation}
\subsubsection{The free energy}
The free energy is
\begin{equation}
\begin{split}
    f_b & \equiv \lim_{N \rightarrow \infty } \frac{-k_B T}{N} \log{Z_N (K,h)}   \\
    & = -k_B T \lim_{N \rightarrow \infty } \frac{1}{N}  \log{\qty[\lambda _+^N \qty(1+\qty(\frac{\lambda _-}{\lambda _+})^N ) ] } \\
    & = -k_B T \log{\lambda _+}
\end{split}
\end{equation}
and inserting the explicit expression of \( \lambda _+ \) for the Ising model, we get
\begin{equation}
\begin{split}
f_b  &=  -k_B T \log{ \qty( e^{K} \cosh h + \sqrt{e^{2K} \sinh^2 (h) + e^{-2K}  } ) } \\
& = -K k_B T - k_B T \log{\qty( \cosh(h)+\sqrt{\sinh^2(h)+e^{-4K}  } )}
\end{split}
\end{equation}
\begin{remark}
Rembember that \( K \equiv \beta J, h \equiv \beta H\).
\end{remark}
\begin{exercise}
Check that if \( h=0 \) we get back the expression found previously with the iterative method (what is the important of boundary conditions?).
\end{exercise}
Let us now consider the limits \( T \rightarrow 0 \) and \( T \rightarrow \infty  \) by keeping \emph{H} fixed and \emph{J} fixed.
\begin{itemize}
\item Case: \( T \rightarrow 0  \Rightarrow K \rightarrow \infty , h \rightarrow \infty  \).
\begin{subequations}
\begin{align}
  e^{-4K} & \overset{K \rightarrow \infty }{\longrightarrow} 0  \\
  \sqrt{\sinh^2 h} & \overset{h \rightarrow \infty }{\sim } \sinh (h)
\end{align}
\end{subequations}
This implies that
\begin{equation}
\cosh(h) + \sinh h \sim \frac{2 e^{h} }{2} \simeq e^{h}
\end{equation}
and
\begin{equation}
  f \overset{\substack{h \rightarrow \infty  \\ K \rightarrow \infty  } }{\sim } - K k_B T - k_B T \log{e^{h} } \sim -J -H \quad const
\end{equation}
Therefore, as \( T \rightarrow 0^+ \), \emph{f} goes to a constant that depends on \emph{J} and \emph{H}.

\item  Case: \( T \rightarrow \infty   \Rightarrow K \rightarrow 0 , h \rightarrow 0  \).
In this case we suppose also that \emph{H} and \emph{J} that are fixed, are also finite.
\begin{subequations}
\begin{align}
  e^{-4K} & \simeq 1  \\
  \sqrt{\sinh^2 h + e^{-4K} } & \sim \sqrt{1}
\end{align}
\end{subequations}
Since \( \cosh h \overset{h \rightarrow 0}{\sim } 1  \):
\begin{equation}
  f_B \sim  -K k_B T - k_B T \log{(1+1)} \sim  -J -k_B T \ln{2}
\end{equation}
Therefore, as \( T \rightarrow \infty  \), the free energy goes linearly to zero, as in figure \ref{fig:8_1}.
\begin{figure}[h!]
\centering
\includegraphics[width=0.5\textwidth]{../lessons/8_image/1.pdf}
\caption{\label{fig:8_1} Plot of the free energy \( f_b \) in function of the temperature \( T \). For \( T \rightarrow 0 \), the free energy becomes constant, while for \( T \rightarrow \infty  \) it goes linearly to zero.  }
\end{figure}
\end{itemize}

\subsubsection{The magnetization}
This can be obtained by differentiating the negative of the free energy with respect to the magnetic field \( H \):
\begin{equation}
  m = - \pdv{f_b}{H} = - \frac{1}{k_B T } \pdv{f_b}{h} = \pdv{}{h} \qty[  \log{\qty( \cosh(h)+\sqrt{\sinh^2(h)+e^{-4K}  } )} ]
\end{equation}
The result is
\begin{equation}
  m =   \frac{\sinh h + \frac{ \sinh h \cosh h}{\sqrt{\sinh^2 h + e^{-4K} }} }{\cosh h + \sqrt{\sinh^2 h + e^{-4K} } }
  \label{eq:8_1}
\end{equation}
\begin{itemize}
\item Case: \( T>0 \) fixed, \( H \rightarrow 0 \leftrightarrow h \rightarrow 0\).
\begin{subequations}
\begin{align}
  \sinh h & \sim h \sim 0, \\ \cosh h &\sim 1
\end{align}
\end{subequations}
In zero field \( h \rightarrow 0 \), we have \( m \rightarrow 0 \) for all \( T>0 \). It means that there is no spontaneous magnetization!
\end{itemize}


\subsubsection{The magnetic susceptibility}
\begin{equation}
  \chi_T  \equiv  \pdv{m}{H} = \frac{1}{k_B T} \pdv{m}{h}
\end{equation}
If we consider the case \( h \ll 1 \), it is convenient first expand the  \eqref{eq:8_1} for \( h \rightarrow 0 \) and take the derivative to get \( \chi _T \).

Since \( \sinh (h) \sim h+h^3 \) and \( \cosh (h) \sim 1 + h^2 \), we have
\begin{equation}
  m \overset{h \ll 1}{\sim } \frac{h (1+e^{2K} )}{1+e^{-2K} }
\end{equation}
If we now derive with respect to \emph{h}
\begin{equation}
  \chi _T = \frac{1}{k_B T} \pdv{m}{h} \overset{h \ll 1}{\approx } \frac{1}{k_B T} \frac{(1+e^{2K} )}{(1+e^{-2K} )}
\end{equation}
\begin{itemize}
\item Case: \( T \rightarrow \infty  \Leftrightarrow K \rightarrow 0\).
\begin{equation}
  e^{2K} \simeq e^{-2K} \simeq 1
\end{equation}
The \textit{Curie's Law} for paramagnetic systems is:
\begin{empheq}[box=\myyellowbox]{equation}
  \chi _T \sim \frac{1}{k_B T}
\end{empheq}
\item Case: \( T \rightarrow 0 \Leftrightarrow K \rightarrow \infty  \).
\begin{equation}
  e^{-2K} \simeq 0
\end{equation}
The \textit{Curie's Law} for paramagnetic systems is:
\begin{empheq}[box=\myyellowbox]{equation}
  \chi _T \sim \frac{1}{k_B T} e^{2K} \sim  \frac{1}{k_B T} e^{2J/k_B T}
\end{empheq}
\end{itemize}

\subsubsection{The correlation length}
\begin{equation}
  \xi ^{-1} = -\log{\qty(\frac{\lambda _-}{\lambda _+}) } = - \log{\qty[ \frac{\cosh h - \sqrt{\sinh^2 h + e^{-4K} } }{\cosh h + \sqrt{\sinh^2 h + e^{-4K} }}] }
\end{equation}
For  for \( h=0 \), we have \( \cosh h \rightarrow 1, \sinh h \rightarrow 0 \):
\begin{equation}
  \xi ^{-1} = - \log{\qty[\frac{1- e^{-2K} }{1+ e^{-2K} }] } =  - \log{\qty[\frac{1 }{\coth K }] }
\end{equation}
Therefore:
\begin{equation}
  \xi = \frac{1}{\log{ (\coth K )} }
\end{equation}
\begin{itemize}
\item Case: \( T \rightarrow 0 \Leftrightarrow K \rightarrow \infty  \).
\begin{equation}
  \coth K = \frac{e^{K} + e^{-K}  }{e^{K} - e^{-K}  } \overset{K \rightarrow \infty }{\simeq} 1 + 2 e^{-2K} + \dots \quad \overset{K \rightarrow \infty }{ \longrightarrow  } 1
\end{equation}
It implies
\begin{equation}
  \xi \overset{K \gg 1}{\sim } \frac{1}{\ln{(1+ 2 e^{-2K} )} } \sim \frac{e^{2K} }{2}
\end{equation}
Hence
\begin{equation}
  \xi \overset{T \rightarrow 0}{\sim } \frac{1}{2} e^{J/k_B T}
\end{equation}
It diverges exponentially \( \xi  \rightarrow \infty  \), as \( T \rightarrow 0 \).
\item Case: \( T \rightarrow \infty \Leftrightarrow K \rightarrow 0 \).
\begin{equation}
  \coth K = \frac{e^{K} + e^{-K}  }{e^{K} - e^{-K}  } \overset{K \rightarrow 0 }{\simeq}
  \frac{1+K+\frac{K^2}{2}+1-K+\frac{K^2}{2}}{1+K+\frac{K^2}{2}-1+K-\frac{K^2}{2}}
  \sim \frac{2+2 \frac{K^2}{2}}{2K} \sim \frac{1+K^2}{K}
\end{equation}
\begin{equation}
  \xi ^{-1} = \log{(\coth K)} \overset{K \rightarrow 0}{\sim } \ln{\frac{1}{K}} + \ln{(1+K^2)} \sim + \infty
\end{equation}
Therefore:
\begin{equation}
  \xi  \overset{K \rightarrow 0}{\longrightarrow} 0
\end{equation}
More precisely,
\begin{equation}
  \xi \overset{K \rightarrow 0}{\sim } \frac{1}{\ln{(1/K)} + \ln{(1+K^2)}  } \overset{K \rightarrow 0}{\sim } - \frac{1}{\ln{K} }
\end{equation}

\end{itemize}







\section{Classical Heisenberg model for \emph{d=1} }

Suppose to study something different from the Ising model, we do not anymore assume spin that can assume values as -1 or +1, but spin that can assume a continuous value. This is the classical Heisenberg model.

Take a \( d=1 \) dimensional lattice.
In the classical Heisenberg model the spins are unit length vectors \( \va{S}_i \), i.e. \( \va{S}_i \in \R^3 \), \( \abs{\va{S}_i}^2 = 1 \) (continuous values on the unit sphere):
\begin{equation}
  \va{S}_i = ( S_i^x,S_i^y,S_i^z)
\end{equation}
 with periodic boundary condition: \( \va{S}_{N+1} = \va{S}_1 \).

Assuming \( H=0 \), the model is defined through the following Hamiltonian::
\begin{equation}
  - \beta \mathcal{H} ( \{ \va{S} \}  ) = K \sum_{i=1}^{N} \va{S}_i \vdot \va{S}_{i+1} \quad  (\longrightarrow \sum_{i}^{} \va{h} \vdot \va{S}_i  )
\end{equation}
This model satisfies \( O(3) \) symmetry. In the transfer matrix formalism:
\begin{equation}
  Z_N (K) = \sum_{\{ \va{S} \}  }^{} e^{-\beta \mathcal{H}} = \sum_{\{ \va{S} \}  }^{} e^{K \sum_{i=1}^{N} \va{S}_i \vdot \va{S}_{i+1} } = \Tr(\mathbb{T}^N)
\end{equation}
where \( \bra{\va{S}_i} \mathbb{T} \ket{\va{S}_{i+1}} = e^{K \va{S}_i \vdot \va{S}_{i+1}} \).

Similarly to the Ising case:
\begin{equation}
  \mathbb{T} = \sum_{i}^{} \ket{t_i} \lambda _i \bra{t_i}
\end{equation}
and
\begin{equation}
  \mathbb{T}_D = \mathbb{P}^{-1} \mathbb{T}\mathbb{P}
\end{equation}
The problem is computing the eigenvalues \( \lambda _i \) of \( \mathbb{T} \).
Formally, we should find
\begin{equation}
  \exp [K \va{S}_1\vdot \va{S}_{2}] = \bra{\va{S}_1} \mathbb{T} \ket{\va{S}_2} = \sum_{i \in \text{\tiny eigenvalues}}^{} \lambda _i \braket{\va{S}_1}{t_i}  \braket{t_i}{\va{S}_2}
  = \sum_{i}^{} \lambda _i f_i ( \va{S}_1) f^* (\va{S}_2)
\end{equation}
\begin{remark}
  We start by noticing that the term \( e^{K \va{S}_1 \cdot \va{S}_2}  \) is similar to the plane wave \(e^{i \va{q}\vdot \va{r}}  \), that in scattering problems is usually expanded in spherical coordinates. Plane wave can be expanded as a sum of spherical harmonics
  \begin{equation}
    e^{i \va{q}\vdot \va{r}} = 4 \pi \sum_{l=0}^{\infty } \sum_{m=-l}^{l} (i)^l j_l (qr) Y_{lm}^* (\vu{q}) Y_{lm} (\vu{r})
  \end{equation}
  where
  \begin{equation}
    j_l (qr) = - \frac{(i)^l}{2} \int_{0}^{\pi} \sin(\theta ) e^{i q r \cos(\theta ) } P_l (\cos(\theta ) ) \dd[]{\theta }
  \end{equation}
  are the \emph{spherical Bessel functions}, while  the \( P_l (\cos(\theta ) ) \) are the \emph{Legendre polynomial} of order \emph{l}.
\end{remark}
From a formal comparison we have
  \begin{equation}
    \va{S}_1 \leftrightarrow \vu{S}_1, \qquad
      \begin{cases}
           i \va{q}\cdot \va{r} = iqr \\
           K \va{S}_1 \cdot \va{S}_2 = K \abs{\va{S}_1} \abs{\va{S}_2} = K
      \end{cases}
  \end{equation}
  multiplying by \( (-i) \)  we can write
  \begin{equation}
    qr = -iK \abs{\va{S}_1} \abs{\va{S}_2} = -iK
  \end{equation}
  In our case we have
  \( \vu{q} = \va{S}_1, \vu{r} = \va{S}_2 \). Hence,
  \begin{equation}
    e^{K \va{S}_1 \cdot \va{S}_2} = 4 \pi  \sum_{l=0}^{\infty } \sum_{m=-l}^{l} (i)^l j_l (-iK) Y_{lm}^* (\va{S}_1) Y_{lm} (\va{S}_2) = \sum_{i}^{} \lambda _i f_i (\va{S}_1) f^* (\va{S}_2)
  \end{equation}
where
\begin{equation}
  \lambda _i = \lambda _{lm} (K) = 4 \pi (i)^l j_l (-iK)
\end{equation}
\begin{remark}
Note that \( \lambda _i \) does not depend on \emph{m}!
\end{remark}


If \( l=0 \), the largest eigenvalue is:
\begin{equation}
  \lambda _+ = \lambda _0 (K) = 4 \pi j_0 (-iK) = 4 \pi \frac{\sin K }{K}
\end{equation}
and
\begin{equation}
  \lambda _- = \lambda _1 (K) = 4 \pi i j_1 (-iK) = 4 \pi \qty[\frac{\cosh K }{K}-\frac{\sinh K }{K^2}]
\end{equation}

\begin{exercise}
Given the largest eigenvalue \( \lambda _+ \):
\begin{equation}
  \lambda _+ = 4 \pi \frac{\sin(K) }{K}
\end{equation}
find the bulk free energy density of the model and discuss its behaviour in the limits of low (\( T \rightarrow 0 \)) and high (\( T \rightarrow \infty  \)) temperatures.
\end{exercise}


How can we violate the hypothesis of the Perron-Frobenius theorem hoping to find a phase transition also in a \( d=1 \) model?
 One of the hypothesis of the Perron-Frobenius theorem is the one in which \( A_{ij}>0  \) for all \(i,j \). Hence, one possibility is to build a model in which its transfer matrix has same \( A_{ij} \) that are equal to zero also for \( T \neq 0 \).

\end{document}
