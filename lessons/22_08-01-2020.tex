\documentclass[../main/main.tex]{subfiles}

\newdate{date1}{8}{1}{2020}
\newdate{date2}{9}{1}{2020}

\begin{document}
(VEDERE APPUNTI DA ALTRI PERCHè non ho scritto tutto)
\marginpar{ \textbf{Lecture 22.} \\  \displaydate{date1}. \\ Compiled:  \today.}

\section{Renormalization group}

The idea is to consider a partition function \( Z_N (\va{k}) \) and then integrate, in order to obtain another partition function \( Z_{N(/l^d)} (\va{k}') \).
We make the assumptions:
\begin{enumerate}
\item \( R_l (\va{k}) = \va{k}' \), \( R_l \) is analytic
\item  \( R_l \dot R_{l'}  = R_{l l'} \)   semigroup %circle! not dot
\end{enumerate}

Procedure:
\begin{itemize}
\item You find the fixed point
\item You linearized around the fixed point.
\item You find the behaviour.
\end{itemize}

\section{Ising model}
we use l = 2.

We have \( \sigma _i = \pm 1 \) and a weight
\begin{equation}
  w (\sigma _i, \sigma _j) = - \hat{g} - \frac{h}{z} (\sigma _i + \sigma _j) - J \sigma _i \sigma _j
\end{equation}
In that sense, the partition function can be written as
\begin{equation}
  Z = \sum_{\{ \sigma  \}  }^{}  e^{\sum_{\expval{ij} }^{} w (\sigma _i, \sigma _j)  }
\end{equation}

\begin{equation}
  Z (g, h, k) = \sum_{\{ \sigma  \}  }^{}   = \sum_{\{ \sigma ' \}  }^{}  \sum_{\{ S \}  }^{} e^{\sum_{i}^{} w (\sigma _i, \sigma _{i+1}) }
\end{equation}
(the \( \sigma _1' \) are the new red bubble, while the \( S_1 \) the rejected one in the second graph. In the first graph we had \( \sigma _1 \)  )

\begin{equation}
  w (\sigma _i, \sigma _{i+1}) =  g + \frac{h}{z} (\sigma _i + \sigma _{i+1}) + k \sigma _i \sigma _{i+1}
\end{equation}
Hence,
\begin{equation}
  Z (g, h, k)  = \sum_{\{ \sigma' \}_{N/2}  }^{}  \sum_{\{ S \}_{N/2}  }^{} e^{\sum_{i}^{N/2} \qty[ w (\sigma _i', S_i) + w (S_i, \sigma _{i+1}' )  ] }
  = \sum_{\{ \sigma' \}_{N/2}  }^{}  \prod_{i=1}^{N/2} \qty( \sum_{S = \pm 1}^{}
  e^{ w (\sigma _i', S_i) + w (S_i, \sigma _{i+1}' )   }  )
\end{equation}
and we define
\begin{equation}
  f ( \sigma _i', \sigma _{i+1}') =   e^{ w (\sigma _i', S_i) + w (S_i, \sigma _{i+1}' )   }
\end{equation}
Hence,
\begin{equation}
  f ( \sigma _i', \sigma _{i+1}')  = \exp [ w' (\sigma _i', \sigma _{i+1}')]
\end{equation}
So
\begin{equation}
  Z_N (g,h,k) = \sum_{\{ \sigma' \}_{N/2}  }^{}  e^{\sum_{i}^{N/2}  w' (\sigma _i', \sigma _{i+1}' )  }
  = Z_{N/2} (g',h',k')
\end{equation}
It means that this equations must holds for each \( \sigma _i' \) and \( \sigma _{i+1}' \).
\begin{equation}
  e^{g'+\frac{h'}{2} (\sigma _i'+ \sigma _{i+1}') + k' \sigma _i' \sigma _{i+1}'}
  = \sum_{S_i = \pm 1}^{}
    e^{g +\frac{h}{2} (\sigma _i'+ S_i) + k \sigma _i' S_i + g + \frac{h}{2} (S_i + \sigma _{i+1}) + k S_i \sigma _{i+1}'}
\end{equation}
\( \forall (\sigma _i', \sigma _{i+1}') \)
\begin{equation}
  \begin{cases}
   x \equiv e^K\\
   y \equiv e^h\\
   z \equiv e^g
  \end{cases}
\end{equation}
\begin{equation}
  z' y'^{\frac{\sigma _i' + \sigma _{i+1}'}{2}  } x'^{\sigma _i' \sigma _{i+1}'}
  = \sum_{S_i}^{} z y^{ \frac{\sigma _i'+S_i}{2}} x^{\sigma _i' S_i} z y^{S_i + \sigma _{i+1}'} x^{S_i \sigma _{i+1}'}
\end{equation}

Copiare tabella da qualcuno: (da 4 equazioni )
\( \sigma _i' \) \( \sigma _{i+1}' \)

++
\begin{equation}
  z' y' x' = z^2 y (x^2 y + x^{-2} y^{-1}) \quad \text{for } S_i = +1, S_i =-1
\end{equation}
--
\begin{equation}
  z' y'^{-1} x' = z^2 y^{-1} (x^{-2} y + x^{2} y^{-1})
\end{equation}
+-
\begin{equation}
    z'  x'^{-1} = z^2 y (y y^{-1})
\end{equation}
-+
\begin{equation}
    z'  x'^{-1} = z^2 y (y y^{-1})
\end{equation}

Now we multiply \( (1)(2)(3)(4) \)
\begin{equation}
  (1)(2)(3)(4) \Rightarrow  z'^{4} = \dots (see notes)
\end{equation}
\begin{equation}
  (1)/(2) \Rightarrow y'^2 = y^2 \frac{x^2 y + x^{-2} y^{-1}}{ x^{-2}y + x^2 y^{-1}}
\end{equation}
\begin{equation}
  \frac{(1)(2)}{(3)(4)} \Rightarrow x'^4 = \frac{(x^2 y + x^{-2} y^{-1} ) (x^{-2}y + x^2 y^{-1}) }{(y+y^{-1})^2}
\end{equation}
(equation above)
We obtain:
\begin{equation}
  2 h' = 2 h + \log() - \log()
\end{equation}
\begin{equation}
  4 k' = \log() + \log() - 2\log()
\end{equation}
Those are the normalization equation.
\begin{equation}
  (k',h') = \mathcal{R}_2 (k,h)
\end{equation}

\marginpar{ \textbf{Lecture 23.} \\  \displaydate{date2}. \\ Compiled:  \today.}
Consider (equation above).
We decide to consider the case \( h=0 \Rightarrow y=1\). Given \( y=1 \), we have
\( y'=1 \). The one that remains is the one for \( x' \).
\begin{equation}
  x'^4 = \frac{(x^2 + x^{-2})^2}{4}
\end{equation}
\begin{equation}
  x=e^{k} \quad \Rightarrow \quad x^{4k'} = \qty(\frac{e^{2k} + e^{-2k}  }{2})^2
\end{equation}
You can take the log if you wish:
\begin{empheq}[box=\myyellowbox]{equation}
  k' = \frac{1}{2} \ln{(\cosh (2k))}
\end{empheq}
That is what you want: \( k' \) wrote as a function of \( k \).
\begin{equation}
  e^{2k'} = \cosh (2k) = 2 \cosh^2 (k) - 1
\end{equation}
\begin{equation}
  e^{2k'} - 1 = 2 \cosh^2 (k) - 2 = 2 \sinh^2 (k)
\end{equation}
\begin{equation}
  e^{2k'} + 1 = 2 \cosh^2 (k)
\end{equation}
\begin{equation}
  \frac{e^{2k'} -1 }{e^{2k'} + 1  } = \tanh (k') = \tanh^2 (k)
\end{equation}
\begin{empheq}[box=\myyellowbox]{equation}
  k' = \tanh^{-1} \qty[ (\tanh (k))^2]
\end{empheq}
We have written \( k' \) in two different ways.
\begin{equation}
  y \equiv \tanh (k)
\end{equation}
\begin{empheq}[box=\myyellowbox]{equation}
  y' = y^2
\end{empheq}
that is much easier to study the fixed point. We have
\begin{equation}
  \begin{cases}
     y^* = 0 & \Rightarrow (k \rightarrow 0^+, T \rightarrow  \infty )\\
     y^* = 1 & \Rightarrow (k \rightarrow \infty, T \rightarrow 0^+)
  \end{cases}
\end{equation}
\begin{equation}
  y^* = 1^- \quad y_0 = 0.99 \Rightarrow \text{ UNSTABLE}
\end{equation}
\begin{equation}
  y^* = 0 \quad y_0 = 0^+ = 0.01 \Rightarrow \text{ STABLE}
\end{equation}
We have
\begin{equation}
  \tanh k' = (\tanh k)^l
\end{equation}

\section{Decimation of Ising on square lattice}

If you increase the dimension, the decimation procedure will transform as:
\begin{equation}
  g,k,h \rightarrow g',k',h',k_2'
\end{equation}
If you do another iteration, you will get another bond even in long range. You are increasing the size of the iteration of the order parameter. It is called proliferation.

(figure 1)
\( h=0 \)
\begin{equation}
  Z_N (k,g) = \sum_{\{ 0 \}  }^{} \sum_{\{ y \}  }^{} e^{\sum_{\expval{ij} }^{} S_i S_j }
  = Z_{N/2} (k',g',L',Q')
\end{equation}
(figure 2)




\begin{equation}
  k^* = \frac{3}{8} \ln{(\cosh 4k^*)} \Rightarrow k^* = 0.50
\end{equation}
Onsager solution \( k_c \) gives \( k_c \simeq 0.4406 \). It is not bad.

For the Onsanger solution we have \( \alpha =0 \), bacuse logaritmic divergence.

\section{}
The points is: we are not asble to do decimation because we have too many bonds. The idea is: why before doing renormalization, do we move the bonds?
\begin{equation}
  k' = \frac{1}{2} \ln{ \cosh (2 \cdot 2k)}
\end{equation}

This is just called Migdal-Kadanoff approximation.


\begin{equation}
  k' = \frac{1}{2} \cosh (4 \cdot 2 k)
\end{equation}
\begin{equation}
  k'_l = \frac{1}{2} \ln \cosh (l^{d-1}\cdot 2k)
\end{equation}
This is called variational renormalization group.

\section{}
Consider a one dimensional model.
\begin{equation}
  Z_N = \sum_{\{ S  \}  }^{} \prod_{i=1}^{N} \mathcolorbox{green!20}{ e^{W (S_i, S_{i+1})} }
\end{equation}
this is the starting point of any renormalization group.
We can rewrite the gree as
\begin{equation}
  e^{W (S_i, S_{i+1})} = \bra{S_i} T \ket{S_{i+1}}
\end{equation}
Hence, using the transfer matrix
\begin{equation}
  Z_N = \Tr (T^N) = \Tr(T^2)^{N/2}
\end{equation}
Just think what it is \( T^2 \): it is exactly the contraction, the decimation. It is telling you that actually \( T^2 \) is the transfer matrix description of the constgraings ising model. That is very powerful. If you know the \( T \) you know the \( T^2 \), so you know how to write the \( Z_N \)!

If you wish:
\begin{equation}
  T'_2 = T^2
\end{equation}

Try to solve the decimation using the transfer matrix, it is very useful!!!!

The renormalization equation can be written as
\begin{equation}
  T_l' ( \{ \va{k}' \}  ) = T^l ( \{ \va{k} \}  )
\end{equation}
\begin{equation}
  T_l' ( \{ \va{k}' \}  ) = T ( \{ l^{d-1} \va{k} \}  )^l
\end{equation}






\section{Spontaneous symmetry breaking}








\end{document}
