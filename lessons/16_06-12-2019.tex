\documentclass[../main/main.tex]{subfiles}

\newdate{date}{6}{12}{2019}


\begin{document}

\section{First-order phase transitions in Landau theory}
\marginpar{ \textbf{Lecture 16.} \\  \displaydate{date}. \\ Compiled:  \today.}

As we have seen, Landau theory is based on the assumption that the order parameter is small near the critical point, and we have seen in the example of the Ising model how it can describe a continuous phase transition (in fact, for \( t \rightarrow 0 \) we have \( \eta \rightarrow 0 \)).
However, because of the symmetry properties of the Ising model we have excluded any possible cubic term; what we now want to do is to consider a more general form of
\( \mathcal{L} \) which includes also a cubic term in \( \eta  \) (in the case in which the symmetry is not violated), and see that this leads to the occurrence of a first-order phase transition.
In fact, we want to generalize to include multicritical points, or phase transitions. Let us remember that in the Ising model we have phase transition derived by symmetry breaking, while now we have another type of phase transitions.

We have seen that since the order parameter is null for \( T > \bar{T}  \) the Landau free energy cannot contain any linear term in \( \eta  \). Let us therefore consider the simplest Landau free energy that depends on a particular field:
\begin{empheq}[box=\myyellowbox]{equation}
\mathcal{L} ( \eta ,t,h)= a t \eta ^2 - w \eta ^3 + \frac{b}{4} \eta ^4 - h \eta
\end{empheq}
where \( t \equiv \frac{T-T^*}{2} \) and \( w \) is an additional parameter that we fix to be positive, \( w>0 \); as in the previous case, we must have \( b>0 \) so that \( \eta  \) has finite values in the equilibrium configurations. In addiction,
\begin{equation*}
  a t = \frac{a}{2} ( T - T^*) \quad \begin{cases}
    > 0 & \text{ if } T > T^* \\
    <0 & \text{ if } T < T^*
\end{cases}
\end{equation*}
\begin{remark}
For \( w<0 \) the results are the same, but in the \( \eta <0 \) diagram.
\end{remark}
The temperature \( T^* \) is the one at which we have the continuous transition if \( w=0 \), but as we will see it doesn't have great significance now. The equilibrium configurations of the system,  will be given by:
\begin{equation*}
  \pdv{\mathcal{L}_G}{\eta } = 0 \quad \Rightarrow h = 2 a t \eta - 3 w \eta ^2 + b \eta ^3
\end{equation*}
In absence of external fields (\( h=0 \)), the equilibrium states becomes
\begin{equation*}
  h = 0 \quad \Rightarrow \eta ( 2 a t - 3 w \eta  + b \eta ^2) = 0
\end{equation*}
The solutions of this equation are
\begin{equation}
  \begin{cases}
   \bar{\eta } = 0 & \text{disordered phase}\\
  \bar{\eta_\pm } = \frac{1}{2b} \qty(3w \pm \sqrt{9 w^2 - 4abt} ) & \text{ordered phases}
  \end{cases}
\end{equation}
Let us rewrite the ordered solutions as
\begin{equation*}
  \bar{\eta }_{\pm} = \frac{1}{2b} \qty(3w \pm \sqrt{9 w^2 - 4abt} ) = c \pm \sqrt{c^2 - \frac{at}{b}}
\end{equation*}
with
\begin{equation*}
  c = \frac{3w}{4b}
\end{equation*}
However, these two last solutions are possible only if:
\begin{equation*}
  \bar{\eta }_{\pm} \in \R \iff c^2 - \frac{at}{b}>0  \iff t = \frac{T-T^*}{2} < \frac{c^2 b}{a} \equiv t^{**} \equiv  \frac{T^{**}-T^*}{2}
\end{equation*}
Hence, we have
\begin{equation*}
  T^{**} = T^* + \frac{2c^2b}{a} = T^* + 2t^{**}
\end{equation*}
so, since \( t^{**} \) is positive, this will occur at temperatures higher than \( T^* \).
Let us consider different cases:
\begin{itemize}
\item If \( t> t^{**} \) \( ( T > T^ {**}) \), then the system will be in the disordered phase and we have  \( \bar{\eta }_\pm \notin \R  \). The only real solution is \( \bar{\eta }=0  \) that is also the absolute minimum of \( \mathcal{L} \). The plot is shown in Figure \ref{fig:16_1}.
\begin{figure}[h!]
\centering
\includegraphics[width=0.6\textwidth]{../lessons/16_image/1.pdf}
\caption{\label{fig:16_1} Landau free energy for \( t> t^{**} \) \( ( T > T^ {**}) \). The point \( \bar{\eta }=0  \) is the absolute minimum.}
\end{figure}


\item If \( t \le t^{**} \) \( ( T \le T^ {**}) \), we have \( \bar{\eta }_\pm = c \pm \sqrt{c^2 - \frac{at}{b}} \in \R  \) are both possible solutions. One will be a local maximum and the other a local minimum.

  \begin{itemize}
  \item At \( T = T^{**}\), we have \( \bar{\eta }_- = \bar{\eta }_+   \) (flex point), as shown in Figure \ref{fig:16_2}.
  \begin{figure}[h!]
  \centering
  \includegraphics[width=0.6\textwidth]{../lessons/16_image/2.pdf}
  \caption{\label{fig:16_2} Landau free energy for \( t\le t^{**} \) \( ( T = T^ {**}) \). The point \( \bar{\eta }_- =\bar{\eta }_+   \) is a flex one.}
  \end{figure}


  \item For \( T_t < T < T^{**} \), a new minimum appears at \( \eta = \bar{\eta }_+   \), but we will have \( \mathcal{L} (\bar{\eta }_+) >0  \), so  this is only a local minimum (since \( \mathcal{L}(0)=0 \)):  in this range of temperatures the ordered phase is metastable. The plot is shown in Figure  \ref{fig:16_3}.
  \begin{figure}[h!]
  \centering
  \includegraphics[width=0.6\textwidth]{../lessons/16_image/3.pdf}
  \caption{\label{fig:16_3} Landau free energy for \( t\le t^{**} \) \( ( T_t <T \le T^ {**}) \). The point \( \bar{\eta }_+   \) is a local minimum.}
  \end{figure}


  \item If we further decrease the temperature \( T \), we will reach a temperature \( T=T_t \) for which \(\mathcal{L} (\bar{\eta }_+ ) = 0 = \mathcal{L}(0)  \):  at this point the ordered and disordered phase coexist, so this is the temperature of a new transition! The plot is shown in Figure  \ref{fig:16_4}. \( T_t \)  is given by the coexistence condition
  \begin{equation*}
    \mathcal{L} (\bar{\eta }_+ ) = \mathcal{L} (0)
  \end{equation*}
  that is the coexistence between the disordered and ordered phases. In fact, in the plot of Figure \ref{fig:16_4} we see that there are two minima in the same line, this is  a first order transition.

  \begin{figure}[h!]
  \centering
  \includegraphics[width=0.6\textwidth]{../lessons/16_image/4.pdf}
  \caption{\label{fig:16_4} Landau free energy for \( t\le t^{**} \) \( ( T = T_t ) \). The point \( \bar{\eta }_+   \) is a minimum. The ordered and disordered phase coexist.}
  \end{figure}


  \item Finally for \( T^* < T < T_t \), \( \mathcal{\bar{\eta }_+ } \) becomes negative and so now \( \eta = \bar{\eta }_+  \) is the global minimum of \( \mathcal{L} \):  the ordered phase becomes stable and the disordered phase metastable, indeed now \( \eta =0 \) is only a local minimum (see Figure \ref{fig:16_5}).

  \begin{figure}[h!]
  \centering
  \includegraphics[width=0.6\textwidth]{../lessons/16_image/5.jpg}
  \caption{\label{fig:16_5} Landau free energy for \( t\le t^{**} \) \( ( T^*< T <T_t ) \). The point \( \bar{\eta }_+   \) is the global minimum.}
  \end{figure}


  \item  If now \( T < T^*\), \( \mathcal{L} \) develops a new minimum for \( \eta <0 \), but it is only a local minimum (the asymmetry introduced by \( -w \eta ^3 \) ensures that \( \bar{\eta }_+  \) is always the global minimum).
  This means that also for \( T< T^* \)  the disordered phase with \( \bar{\eta }_+  \) continues to be the stable one, and so no phase transition occurs at \( T^* \)  any more; this is what we meant when we said that \( T^* \) is not a relevant temperature any more.
  \end{itemize}

  \begin{figure}[h!]
  \begin{minipage}[c]{0.55\linewidth}
  \subfloat[][First-order transition.]{ \includegraphics[width=0.8\textwidth]{../lessons/16_image/p_1.png}  \label{fig:} }
  \end{minipage}
  \begin{minipage}[]{0.55\linewidth}
  \centering
  \subfloat[][Same transition for lower values of the temperature.]{\includegraphics[width=0.8\textwidth]{../lessons/16_image/p_2.png}  \label{fig:} }
  \end{minipage}
  \caption{\label{fig:16_p} The notation in this plot is different from the one used previously. Here \( \bar{T} \equiv T^*  \), \( T^{*} \equiv T^{**} \) and \( T^{**} \equiv T_t \).  }
  \end{figure}

Therefore, we have seen that lowering the temperature of the system, the value of \( \eta  \) for which \( \mathcal{L} \) has a global minimum changes discontinuously from \( \eta =0 \)  to \( \bar{\eta }_+  \): this is a first-order transition. All the results obtained are shown in Figure \ref{fig:16_p}.

\end{itemize}


  As we said, at \( T=T_t \) the system undergoes a first order transition. It is defined by two conditions: it must be a minimum of \( \mathcal{L} \) and such that the value of \( \mathcal{L} \) in that minimum is zero.  Thus we can determine \( T_t \) as follows:
  \begin{equation*}
    \begin{cases}
     \pdv{\mathcal{L}}{\eta } = 0 = \eta \qty(2 a t - 3 w \eta +  b \eta ^2)  & \text{extreme condition}\\
    \mathcal{L} ( 0 ) = \mathcal{L} ( \eta _+)=0 = \eta ^2 (a t - w \eta + \frac{b}{4} \eta ^2)& \text{coexistence condition}
    \end{cases}
  \end{equation*}
  Therefore, for \( \eta \neq 0 \):
  \begin{equation*}
  \Rightarrow
    \begin{cases}
     2 a t - 3 w \eta +  b \eta ^2 = 0\\
     a t - w \eta + \frac{b}{4} \eta ^2 = 0
    \end{cases}
  \end{equation*}
  Solving with respect to \( \eta  \)  and \( t \), we get
  \begin{equation*}
    \begin{cases}
     \bar{\eta }_{t} = + \frac{2w}{b} >0 \\
      t_t = \frac{2w^2}{ab}
    \end{cases}
  \end{equation*}
Since by the definition \( t = (T-T^*)/2 \), we have:
  \begin{equation}
    T_t = T^* + \frac{4w^2}{ab}
  \end{equation}
  \begin{remark}
  Let us note that \( T_t >T^* \).
  \end{remark}
  Since at \( T= T_t \) there is a first order transition does the system display latent heat?
  \begin{equation*}
    s = \eval{- \pdv{\mathcal{L}}{T} }_{\eta _t} = - \frac{1}{2} a \bar{\eta }_t^2 = - \frac{a}{2} \qty(\frac{2w}{b})^2
  \end{equation*}
  Hence, there is an entropy jump.
  The latent heat absorbed to go from the ordered to the disordered phase is
  \begin{equation}
    q = - T_t s = \frac{a}{2} T_t \qty(\frac{2w}{b})^2
  \end{equation}


\section{Phase stability and behaviour of \( \chi _T \equiv \pdv{\eta }{h}  \)}

Let us derive the equation of state with respect to \( h \)
\begin{equation}
  \pdv{}{h} \qty(\pdv{\mathcal{L}_G}{\eta } = 0 )   = \pdv{}{h} \qty(2 a t \eta - 3 w \eta ^2 + 2 b \eta ^3 = h)
\end{equation}
\begin{equation}
  \Rightarrow \chi \qty(2 a t - 6 w \eta + 6 b \eta ^2 ) =1
\end{equation}
The results is
\begin{equation}
  \chi _T = \frac{1}{2 a t - 6 w \eta + 6 b \eta ^2}
  \label{eq:16_1}
\end{equation}
We now make use of equation \eqref{eq:16_1} to compute the limit of stability of the phases we have found.

\subsection{Computation of \( T^{**} \) }
\( T=T^{**} \) is the value below which the ordred phase \( (\bar{\eta } = \bar{\eta }_+ ) \) becomes a metastable state (local minima). Since this coincides with the flex point
\begin{equation}
  \pdv[2]{\mathcal{L}}{\eta } = 0 \quad \Rightarrow \chi ^{-1} = 2 a t - 6 w \eta + 6 b \eta ^2 = 0
\end{equation}
Remember that at \( T=T{**} \)  the two solutions \( \eta _\pm \) do coincide
\begin{equation}
  c^2 - \frac{at}{b} = 0 \Rightarrow \eta _\pm = \eta _2 = \frac{3 w}{4 b}
\end{equation}
Inserting in the expression for \( \chi ^{-1} \) since \( \chi ^{-1} =0 \) we have
\begin{equation}
  \chi ^{-1} = 0 = 2 a t^{**} - 6 w \bar{\eta }_2 + 6 \bar{\eta }_2^2
\end{equation}
\begin{equation}
  \iff t^{**} = \frac{q w^2}{16 a b} = \frac{1}{2} \qty(T^{**}-T^*)
\end{equation}
For \( T_t < T < T^{**} \) the ordered phase is metastable.

\subsection{Computation of \( T^* \)}
The instability of the disordered phase \( (\eta =0) \) is when \( \mathcal{L} \) presents a flex point at \( \eta =0 \), as in Figure \ref{fig:16_6}.
\begin{equation}
  \chi ^{-1} (\eta =0) = 0 \iff \chi ^{-1} (\eta =0) = 2 a t = 0 \iff t = 0 \quad \Rightarrow T = T^*
\end{equation}

\begin{figure}[h!]
\centering
\includegraphics[width=0.6\textwidth]{../lessons/16_image/6.pdf}
\caption{\label{fig:16_6} Description.}
\end{figure}


\section{Landau theory and multicritical points}
Tricritical point is a critical point that separates a line of first transition points from a line of critical points.
\begin{remark}
The introduction of a cubic term is not the only way to obtain a first order transition. Let allow the coefficient of \( \eta ^4 \) to change signs. We need the \( \eta ^6 \) term (See Blume-Emery-Griffith model).
\end{remark}
\begin{equation}
  \mathcal{L}_h (T, \Delta, \eta ) = \frac{a (t,\Delta )}{2} \eta ^2 + \frac{b(t,\Delta )}{4} \eta ^4 + \frac{c}{6} \eta ^6 - h \eta
\end{equation}
where we have two parameteres \( (T,\Delta ) \) and \( c>0 \), with \( \Delta  \) that is the disordered field (\( \% \, \text{He}^3\) in the BEG model).

Now, consider \( \Delta  \) (\( \Delta _c \) is a critical value). The phenomenology is

\begin{itemize}
\item \( \Delta < \Delta _c \): as \( T \) decreases, \( a(T,\Delta ) \) decreases and at \( T = T_c (\Delta ) \) becomes zero. In this region \( b(T,\Delta ) >0 \) and the system displays the standard \( (\eta ^4) \) critical point.
\begin{equation}
  T = T_c (\Delta ) \Rightarrow \begin{cases}
    a (T_c,\Delta ) = 0 \\
    b (T_c,\Delta ) > 0
\end{cases}
\end{equation}
\item \( \Delta > \Delta _c \): as \( T \) decreases, \( b(T,\Delta ) \) becomes zero 'before' \( a(T,\Delta ) \). In this case one can show the existence of a line of first transition points:
\begin{equation}
  b (\bar{T},\Delta  ) = 0 \quad \Rightarrow   \mathcal{L} = a \eta ^2 + c \eta ^6, \quad a>0
\end{equation}
\begin{equation}
  \pdv{\mathcal{L}}{\eta } = 2 a \eta + 6 c \eta ^5 \Rightarrow \begin{cases}
    \eta =0 \\
    \eta _{1,2,3,4}
  \end{cases}
\end{equation}

\begin{figure}[h!]
\centering
\includegraphics[width=0.6\textwidth]{../lessons/16_image/7.pdf}
\caption{\label{fig:16_7} Description.}
\end{figure}

If \( b>0 \), \( \eta ^6 \) can be neglected.
\item The Tricritical point is given by the values of \( \Delta = \Delta _t \)  and \( T= T_c \) such that
\begin{equation}
  a (\Delta _t, T_t) = b (\Delta _t, T_t) = 0
\end{equation}
At this point the system is described by the following Landau free-energy
\begin{equation}
  \mathcal{L}_t = c \eta ^6 - h \eta
\end{equation}
and equation of state
\begin{equation}
  h = 6 c \eta ^5
\end{equation}

\begin{figure}[h!]
\centering
\includegraphics[width=0.6\textwidth]{../lessons/16_image/8.pdf}
\caption{\label{fig:16_8} Description.}
\end{figure}

\end{itemize}

\section{Landau-de Gennes theory of liquid crystals}
A liquid crystal phase (LC) can be seen as an intermediate phase (mesophase) between a solid and a liquid phase.
In this phase the system flows as a fluid but it also displays an orientational order typical of a crystal. Because of this order this phase displays \emph{anisotropic} optical, magnetic and electrical properties.

Typical structural properties of the elementary constituents (molecules) giving rise to LC phase are the following
\begin{itemize}
\item Anisotropic (elangated) shape.
\item The long axis can be approximated as a rigid backbane.
\item Existence of strong dipoles and groups easy to polarize.
\end{itemize}
\subsection{LC-phases}
There are many possible LC phase, from \emph{nematic}, \emph{cholesteric}, \emph{smectic}, \emph{columnar}, etc. Here we gocus on the most common one the \emph{nematic phase}.

The nematic phase is characterized by a strong and long range orientational order. As a measure of this order, one consider the director \( \overset{\leftrightarrow}{n} (\va{r}) \). This is a 'two arrow vector' that gives the local average alignment of the elementary constituents. In this description the amplitude of \( \overset{\leftrightarrow}{n} (\va{r}) \)    is irrelevant and one takes \( \overset{\leftrightarrow}{n} (\va{r}) \) such that \( \abs{\overset{\leftrightarrow}{n} (\va{r})}  =1 \). Since there is no head-tail symmetry (apolar order), \( \overset{\leftrightarrow}{n} = -  \overset{\leftrightarrow}{n}  \).
From optical point of view the neumatic phase is birifrangent, i.e. two refraction indices, one parallel to \( \overset{\leftrightarrow}{n}  \) and one perpendicular to \( \overset{\leftrightarrow}{n}  \) (special index).

\begin{figure}[h!]
\begin{minipage}[c]{0.5\linewidth}
\subfloat[][Isotropic phase]{ \includegraphics[width=0.8\textwidth]{../lessons/16_image/9.pdf}  \label{fig:16_9_1} }
\end{minipage}
\begin{minipage}[]{0.5\linewidth}
\centering
\subfloat[][Nematic phase]{\includegraphics[width=0.8\textwidth]{../lessons/16_image/10.pdf}  \label{fig:16_9_2} }
\end{minipage}
\caption{\label{fig:} }
\end{figure}

\subsection{Order parameter}
A macroscopic definition of an order parameter for LC phase is based on the system response when subject to magnetic or electric fields. For instance, given an external magnetic field \( \va{H} \) the (diamagnetic) response of the system can be written as
\begin{equation}
  \va{M} = \bar{\bar{\chi } } \va{H}
\end{equation}
where the matrix \(  \bar{\bar{\chi } } \) is the response function.

In components
\begin{equation}
  M_ \alpha = \chi _{\alpha \beta } H _{\beta }
\end{equation}
with \( \alpha ,\beta =x,y,z \). For a static \( \va{H} \), \( \chi _{\alpha \beta } \) is symmetric.

 Clearly, for a isotropic fluid
\begin{equation}
  \chi _{\alpha \beta } = \chi  \delta _{\alpha \beta }
\end{equation}
For a LC in the neumatic phase one has
\begin{equation}
  \chi _{\alpha \beta } =
  \begin{pmatrix}
  \chi _\bot   &  0 & 0 \\
    0 &  \chi _ \bot & 0 \\
    0 &  0 &  \chi _\parallel
  \end{pmatrix}
\end{equation}
One can then define an order parameter based on \( \chi _{\alpha \beta } \) as
\begin{equation}
  Q_{\alpha \beta } = A \qty(\chi _{\alpha \beta } - \frac{1}{3} \delta _{\alpha \beta }\Tr \bar{\bar{\chi } }  )
\end{equation}
The order parameter is a second rank traceless tensor. It is possible to show that \(   Q_{\alpha \beta }  \) can be written in terms of the local average orientiational order of the elementary constituents, \(  \overset{\leftrightarrow}{n} (\va{r}) \) and the degree of local order given by a scalar \( S(\va{r}) \).
\begin{equation}
  Q_{\alpha \beta } (\va{r}) = S (\va{r}) \qty(n_{\alpha } (\va{r}) n_{\beta } (\va{r}) - \frac{1}{3}\delta _{\alpha \beta })
\end{equation}
Note that by construtction \( \bar{\bar{Q} }  \) is symmetric and traceless.

Its more general expression is
\begin{equation}
  Q_{\alpha \beta } =
  \begin{pmatrix}
  q_1   & q_2  & q_3 \\
  q_2   & q_4  & q_5 \\
  q_3   & q_5  & -q_1-q_4
  \end{pmatrix}
\end{equation}

\subsection{Landau free energy}
The free energy must be invariant to rotations of the system. Since \( Q_{\alpha \beta } \) transforms as a tensor under rotations, \( \mathcal{L} \) must be a combination of terms as \( \Tr \bar{\bar{Q} }^P \). By keeping terms up to fourth order
\begin{equation}
  \mathcal{L} = \mathcal{L}_0 + \frac{1}{2} A(T) \Tr \bar{\bar{Q} }^2 + \frac{1}{3} B(T) \Tr \bar{\bar{Q} }^3 + \frac{1}{4} C(T) \qty[ \qty(\Tr \bar{\bar{Q} }^2  )^2 + \Tr \bar{\bar{Q} }^4 ]
\end{equation}
In components,
\begin{equation}
  \mathcal{L} = \mathcal{L}_0 + \frac{1}{2} A(T) Q_{\alpha \beta }Q_{\beta \alpha } + \frac{1}{3} B(T) Q_{\alpha \beta } Q_{\beta \gamma  } Q_{\gamma  \alpha }+ \frac{1}{4} C(T) \qty(Q_{\alpha \beta } Q_{\beta  \alpha })^2
\end{equation}
\begin{remark}
Since each \( 3 \times 3 \) matrix satisfies the relation
\begin{equation}
   \Tr \bar{\bar{Q} }^4 = \frac{1}{2} \qty( \Tr \bar{\bar{Q} }^2)^2
\end{equation}
\end{remark}
the term proportional to \( C(T) \) can be written as \( \frac{1}{2}C(T)  \Tr \bar{\bar{Q} }^4 \).
\begin{remark}
The cubic term is here allowed since the rod like molecules have quadrupolar symmetry and the order parameter is a tensor for which the rotational invariance does not imply the absence of the odd powers.
\end{remark}
For the most general case of a \emph{biaxial nematic phase} \( \bar{\bar{Q} }  \) can be diagonalized giving
\begin{equation}
  Q_{\alpha \beta } =
  \begin{pmatrix}
  \frac{2}{3}s   & 0  & 0 \\
    0 &  - \frac{s+ \eta  }{3} & 0 \\
    0 &  0 & - \frac{(s- \eta  )}{3}
  \end{pmatrix}
\end{equation}
where \( \eta =0 \) corresponds to the standard uniaxial nematic phase.

In this cases
\begin{equation}
  Q_{\alpha \beta } =
  \begin{pmatrix}
  \frac{2}{3}s   & 0  & 0 \\
    0 &  - \frac{s  }{3} & 0 \\
    0 &  0 & - \frac{(s )}{3}
  \end{pmatrix}
\end{equation}
and
\begin{equation}
  \mathcal{L} = \mathcal{L}_0 + A(T) \frac{s^2}{3} + \frac{2}{27} B(T) s^3 + \frac{3}{81} C(T) s^4
\end{equation}
Assuming \( A(T) \simeq A (T-T^*) \), \( B(T) =B \) and \( C(T) = C \):
\begin{equation}
  \mathcal{L} = \mathcal{L}_0 + \frac{A}{3} (T-T^*) s^2 + \frac{2}{27} B s^3 + \frac{C}{9} s^4
\end{equation}
that has the same form of the one studied before for the first order phase transition.

In particular
\begin{equation}
  a = \frac{2}{3} A, \quad -w = \frac{2}{27} B, \quad b = \frac{2}{9} C
\end{equation}
It is then easy to see that the isotropic-nematic transition is of the first order and occur at
\begin{equation}
  T_t = \frac{w^2}{ba} + T^* = \frac{B^2}{27 A C} + T^*
\end{equation}



\chapter{Role of fluctuations in critical phenomena: Ginzburg criterium, Coarse-graining and Ginzburg-Landau theory of phase}

\section{Importance of fluctuations: the Ginzburg criterium}
Maybe mean field is not a very good approximation in proximity of the critical point. Question: how bad is the mean field approximation in proximity of the critical point?

Mean field approximations neglect the fluctuations of the order parameter in the computation of \( Z \).
This is a rather strong assumption since the correlation length
\begin{equation}
   \xi \overset{T \rightarrow T_c}{\sim } \abs{T-T_c}^{-\nu }
\end{equation}
diverges. Let us estimate the error one does in using the MF approximation.

To fix the ideas let us consider the Ising model. Since
\begin{equation}
  \expval{S_i S_j} \overset{MF}{\longrightarrow } \expval{S_i} \expval{S_j}
\end{equation}
For each pair of spin \( (S_i,S_j) \) the relative error is
\begin{equation}
  E_{ij} = \frac{ \abs{\expval{S_i S_j}  - \expval{S_i} \expval{S_j}}   }{\expval{S_i}\expval{S_j}  }
\end{equation}
We need to compute the correlation function
\begin{equation}
  G_c (i,j) \equiv \expval{S_i S_j} - \expval{S_i} \expval{S_j} = \expval{(S_i -\expval{S_i} )(S_j - \expval{S_j} )}
\end{equation}
Assuming \emph{translational invariance}
\begin{equation}
  G_c (i,j) \rightarrow G_c \qty(\abs{\va{r}_i - \va{r}_j} ) \rightarrow G_c (r)
\end{equation}
\begin{remark}
In order to compute \( G_c (r) \) we cannot assume omogeneity since \( \expval{S_i} = \expval{S_j} = m   \). It implies that \( G_c = 0 \) identically in MF. We have to allow the system to display disomogeneities in space!
\end{remark}

\begin{remark}
If we want to compute the error in the mean field, is always zero. So, if we want calculate the average with respect to fluctuations it does not work. We can either look at the variation in which the field is the internal one, or we can somehow try to make a variation not because of thermal fluctuations but because we control it. We do this by using an external field. This is the response theory with a variation of the field. (lesson)
\end{remark}

Let us discuss an important point. The function \( G_c \) does not only descrive the spatial correlation of the fluctuations but also, through the linear response theory, the way in which \( m \) varies in response to an external non-homogeneous magnetic field. Within a mean field theory this is the  only way to compute \( G_c \)!

Let us see how it works. In order to do that, instead using an \( H \) we use an \( H_i \).

\begin{equation}
  Z [H_i]= \Tr_{\{ S \} } \qty(e^{-\beta \qty(-J \sum_{\expval{ij} }^{} S_i S_j   - \sum_{i}^{} H_i S_i ) } )
\end{equation}
The definition of the thermal average is
\begin{equation}
  \expval{S_i} = \frac{\Tr_{\{ S \} } \qty(S_i e^{-\beta \qty(-J \sum_{\expval{ij} }^{} S_i S_j   - \sum_{i}^{} H_i S_i ) } ) }{Z[H_i]}
  = \beta ^{-1} \pdv{\ln{Z} }{H_i} = - \pdv{F}{H_i}
\end{equation}
Similarly one can show that
\begin{equation}
  \expval{S_i S_j} = \frac{\beta ^{-1}}{Z} \frac{\partial^2{Z} }{\partial{H_i} \partial{H_j}  } \quad (to \, do)
\end{equation}
Hence,
\begin{equation}
\begin{split}
  G_c (i,j) & = \frac{\beta ^{-2}}{Z}  \frac{\partial^2{Z} }{\partial{H_i} \partial{H_j}  }
  - \qty(\frac{\beta ^{-1}}{Z} \pdv{Z}{H_i} ) \qty(\frac{\beta ^{-1}}{Z} \pdv{Z}{H_j} ) \\
  & = \beta ^{-1} \frac{\partial^2{\ln{Z} } }{\partial{H_i} \partial{H_j}  }
  = - \frac{\partial^2{F(\{ H_i \}  )} }{\partial{H_i} \partial{H_j}  }
\end{split}
\end{equation}
\begin{remark}
  The response in \( i \) due to a variation of \( H \) in \( j \) is
  \begin{equation}
    \pdv{}{H_j} \expval{S_i} = \pdv{}{H_j} \qty[\beta ^{-1} \pdv{\ln{Z} }{H_i} ]  =\pdv{}{H_j} \qty[-\pdv{F}{H_i} ] =  \beta G_c (i,j)
  \end{equation}
\end{remark}

The \emph{generating functions} are:
\begin{itemize}
\item \( Z[H_i] \): generating function of \( G(i,j) \).
\item \( \ln{Z[H_i]} = - \beta F  \): generating function of \( G_c (i,j) \).
\end{itemize}

\begin{remark}
  (lesson)
  \begin{equation}
    \pdv{M}{H_j} = \sum_{i}^{} \pdv{\expval{S_i} }{H_j} = \sum_{i}^{} G_c (i,j)
  \end{equation}
  \begin{equation}
    H_j = H_j (H)
  \end{equation}
  \begin{equation}
    \pdv{M}{H} = \sum_{j}^{} \pdv{M}{H_j} \pdv{H_j}{H}
    =  \beta \sum_{ij}^{} G_c (i,j)
  \end{equation}
  the last one is the susceptibility \( \chi _T \).
   Therefore,
   \begin{equation}
     \chi _T = \beta \sum_{i,j}^{} G_c (i,j)
   \end{equation}
   \begin{equation}
     G_c (i,j) \rightarrow G_c (\abs{i-j} ) \sim G (\abs{\va{r}} )
   \end{equation}
\end{remark}
\section{Fluctuation-dissipation relation}
\begin{equation}
  Z_N = \Tr \exp [ \beta J \sum_{\expval{ij} }^{} S_i S_j + H \sum_{i}^{} S_i    ]
\end{equation}
where \( H_i = H\, \forall i \).
\begin{equation}
  M = \sum_{i}^{} \expval{S_i} = \frac{1}{Z} \Tr \sum_{i}^{} S_i    \exp [ \beta J \sum_{\expval{ij} }^{} S_i S_j + H \sum_{i}^{} S_i    ] = \frac{1}{\beta Z_N} \pdv{Z_N}{H}
\end{equation}
Similarly
\begin{equation}
  \sum_{ij}^{} \expval{S_i S_j} \overset{to do}{=} \frac{1}{\beta ^2 Z_n} \pdv[2]{Z_N}{H}
\end{equation}
\begin{equation}
\begin{split}
  \chi _T & = \pdv{m}{H} = \pdv{}{H} \qty[- \frac{1}{N} \pdv{F}{H} ] = \pdv{}{H}  \qty[\frac{1}{N} k_B T \pdv{\ln{Z} }{H} ] \\
  & = \frac{1}{N} k_B T \qty[\pdv[2]{\ln{Z} }{H} ] = \frac{1}{N} k_B T \qty[\frac{1}{Z} \pdv[2]{Z}{H} - \frac{1}{Z^2} \qty(\pdv{Z}{H} )^2 ] \\
  & = \frac{1}{N \beta } \qty[\beta ^2 \sum_{ij}^{} \expval{S_i S_j}  - \beta ^2 \qty(\sum_{i}^{} \expval{S_i}  )^2   ] \\
  & = \frac{\beta }{N} \sum_{ij}^{}  G_c (i,j) = \frac{\beta }{N} \sum_{ij}^{} G_c (\va{r}_i - \va{r}_j) \\
  & = \beta \sum_{i}^{} G_c (\va{x_i})
\end{split}
\end{equation}
where \( \va{x}_i \equiv \va{r}_i - \va{r}_j \).
\begin{equation}
  \Rightarrow \chi _T = (a^d k_B T)^{-1} \int_{\Omega }^{} \dd[d]{\va{r}} G_c (\va{r})
\end{equation}
this is the fluctuation dissipation relation.
\subsection{Computation of \( E_ R  \)}
The total relative error is the \( E_R (r)\) integrated over the region of radius \( R \le \xi  \), i.e. where correlations are not negligible.
\begin{equation}
  E_{TOT} = \frac{\int_{0}^{\xi } G_c (r) \dd[D]{\va{r}}  }{\int_{0}^{\xi }\expval{S_i} \expval{S_j}  \dd[D]{\va{r}} }
\end{equation}
If \( T < T_c \), there is a solution \( \eta (r) = \eta \neq 0 \)
\begin{equation}
  \Rightarrow \expval{S_i} \expval{S_j} \approx \eta ^2
\end{equation}
is uniform in the region \( \abs{\va{r}} < \xi   \).
\begin{equation}
  E_{TOT} \sim \frac{\int_{0}^{\xi } G_c (\va{r}) \dd[D]{\va{r}}  }{\int_{0}^{\xi } \eta ^2 \dd[D]{\va{r}}  } \ll 1
  \label{eq:16_2}
\end{equation}
This is the Ginzburg criterium. If satisfied the mean field theory is a valid approximation.
\subsection{Estimation of \( E_{TOT} \) as \( t \rightarrow 0^- \)  }
The numerator of the \eqref{eq:16_2} can be approximated as
\begin{equation}
  \int_{V_{\xi }}^{} G_c (r)\dd[D]{r} \overset{\substack{ \text{fluctuation} \\  \text{dissipation} } }{\sim } k_B T_c \chi _T \sim t^{-\gamma  }
\end{equation}
On the other hand, the denominator
\begin{equation}
  \int_{V_{\xi }}^{} \eta ^2 \dd[D]{r} \sim \xi ^D \abs{t}^{2 \beta } \sim t^{2 \beta - \nu D}
\end{equation}
To satisfy the Ginzburg criterium
\begin{equation}
  E_{TOT} \overset{t \rightarrow 0^-}{\sim } t^{- \gamma + \nu D - 2 \beta  } \ll 1
\end{equation}
As \( t \rightarrow 0^- \) this is possible only if \( - \gamma + \nu D - 2 \beta >0  \), hence
\begin{empheq}[box=\myyellowbox]{equation}
  D > \frac{\gamma + 2 \beta  }{\nu } \equiv D_c
\end{empheq}
is called the upper critical dimension.
\begin{itemize}
\item For \( D < D_c \): fluctuations are relevant and mean field is no a good approximation.
\item For \( D > D_c \): fluctuations are less important and mean field describes properly the critical point.
\item For \( D = D_c \): mean field critical exponents ok but strong correction to the scaling expected. For a Ising-like systems (in the mean field)
\begin{equation}
  \beta = \frac{1}{2}, \quad \gamma =1 \quad \Rightarrow D_c = \frac{2}{\nu}
\end{equation}
In order to compute \( D_c \) we need to compute \( \nu  \) withing the mean field approximation. We have to consider the system with spatial disomogeneitis!
\begin{remark}
We have \( \nu _{MF} = 1/2 \). Therefore, the dimension is \( D>4 \) for the mean field.
\end{remark}
\end{itemize}
Now we have a lower critical dimension (remember the last lessons!) and an upper one.

















\end{document}
