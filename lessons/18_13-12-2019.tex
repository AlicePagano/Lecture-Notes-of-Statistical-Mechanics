\documentclass[../main/main.tex]{subfiles}

\newdate{date}{13}{12}{2019}


\begin{document}

\marginpar{ \textbf{Lecture 18.} \\  \displaydate{date}. \\ Compiled:  \today.}

\begin{equation}
  \qty(- \grad ^2 + \xi ^{-2} (t)) G_c (\va{r}-\va{r}') = \frac{k_B T}{k} \delta (\va{r}-\va{r}')
\end{equation}
Let us try to do the Fourier transform.
Let us define
\begin{equation}
  \va{x} \equiv \va{r} - \va{r}'
\end{equation}
e let us call \( \widetilde{G} (q)  \) the Fourier transform of the function \( G \)
\begin{equation}
  \widetilde{G} (q) = \int_{- \infty }^{+ \infty } \dd[]{\abs{\va{x}} } G_c ( \abs{\va{x}} ) e^{- i q \abs{\va{x}} }
\end{equation}
we get
\begin{equation}
  \widetilde{G} ( q ) = \frac{k_B T}{k} \frac{1}{q^2 + \xi ^{-2}}
\end{equation}
At \( T = T_c \), we have  \( \xi \rightarrow \infty  \)  and \( \widetilde{G} (q) \simeq \frac{1}{q^2}  \).
We have
\begin{equation}
  G_c (\abs{\va{x}} ) = \abs{\va{x}}^{2-D}
\end{equation}
In this case we see immediately that \( \eta = 0 \). Go back and find why we have this.
\begin{equation}
  G (\va{x}) = \int_{}^{} \dd[D]{\va{q}}  \frac{1}{(2 \pi )^D} \frac{1}{q^2 + \xi ^{-2}} e^{i \va{q} \vdot \va{x}}
\end{equation}
Let us do it for \( D=3 \):
\begin{equation}
  \Rightarrow G(\abs{x} ) = \frac{4 \pi }{ (2 \pi )^3} \int_{0}^{\infty } \dd[]{q} \frac{q^2}{q^2+ \xi ^{-2}} \int_{-1}^{+1} \dd[]{(\cos \theta  )}    e^{i q \abs{\va{x}} \cos \theta  }
\end{equation}
we get
\begin{equation}
  = \frac{4 \pi }{ (2 \pi )^3} \abs{\va{x}} \int_{0}^{\infty } \dd[]{q} \frac{q \sin(q \abs{\va{x}} ) }{q^2 + \xi ^{-2}}
\end{equation}
At the end
\begin{equation}
  \Rightarrow G ( \abs{\va{x}} ) = \frac{1}{2 \pi } \frac{e^{- \frac{\abs{\va{x}} }{\xi }} }{\abs{\va{x}} }
\end{equation}
Can we rich the simple level of fluctuation? The simple level is the one that follow gaussian distribution.
Let us introduce fluctuations at the Gaussian level.













\end{document}
