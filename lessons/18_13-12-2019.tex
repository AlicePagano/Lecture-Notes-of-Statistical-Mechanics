\documentclass[../main/main.tex]{subfiles}

\newdate{date}{13}{12}{2019}


\begin{document}

\marginpar{ \textbf{Lecture 18.} \\  \displaydate{date}. \\ Compiled:  \today.}

\begin{equation}
  \qty(- \grad ^2 + \xi ^{-2} (t)) G_c (\va{r}-\va{r}') = \frac{k_B T}{k} \delta (\va{r}-\va{r}')
\end{equation}
Let us try to do the Fourier transform.
Let us define
\begin{equation}
  \va{x} \equiv \va{r} - \va{r}'
\end{equation}
e let us call \( \widetilde{G} (q)  \) the Fourier transform of the function \( G \)
\begin{equation}
  \widetilde{G} (q) = \int_{- \infty }^{+ \infty } \dd[]{\abs{\va{x}} } G_c ( \abs{\va{x}} ) e^{- i q \abs{\va{x}} }
\end{equation}
we get
\begin{equation}
  \widetilde{G} ( q ) = \frac{k_B T}{k} \frac{1}{q^2 + \xi ^{-2}}
\end{equation}
At \( T = T_c \), we have  \( \xi \rightarrow \infty  \)  and \( \widetilde{G} (q) \simeq \frac{1}{q^2}  \).
We have
\begin{equation}
  G_c (\abs{\va{x}} ) = \abs{\va{x}}^{2-D}
\end{equation}
In this case we see immediately that \( \eta = 0 \). Go back and find why we have this.
\begin{equation}
  G (\va{x}) = \int_{}^{} \dd[D]{\va{q}}  \frac{1}{(2 \pi )^D} \frac{1}{q^2 + \xi ^{-2}} e^{i \va{q} \vdot \va{x}}
\end{equation}
Let us do it for \( D=3 \):
\begin{equation}
  \Rightarrow G(\abs{x} ) = \frac{4 \pi }{ (2 \pi )^3} \int_{0}^{\infty } \dd[]{q} \frac{q^2}{q^2+ \xi ^{-2}} \int_{-1}^{+1} \dd[]{(\cos \theta  )}    e^{i q \abs{\va{x}} \cos \theta  }
\end{equation}
we get
\begin{equation}
  = \frac{4 \pi }{ (2 \pi )^3} \abs{\va{x}} \int_{0}^{\infty } \dd[]{q} \frac{q \sin(q \abs{\va{x}} ) }{q^2 + \xi ^{-2}}
\end{equation}
At the end
\begin{equation}
  \Rightarrow G ( \abs{\va{x}} ) = \frac{1}{2 \pi } \frac{e^{- \frac{\abs{\va{x}} }{\xi }} }{\abs{\va{x}} }
\end{equation}
Can we reach the simple level of fluctuation? The simple level is the one that follow gaussian distribution.
Let us introduce fluctuations at the Gaussian level.

Idea: consider a function (an usual function)...

Consider \( h=0 \): the saddle point solution is \( m_0 (r) = m_0 \)
\begin{equation}
  \beta \mathcal{H}_{eff} = \int_{}^{} \dd[D]{\va{r}} \qty( a t m^2 + \frac{b}{2} m^4 \frac{k}{2} \qty(\grad m)^2 )
\end{equation}
\begin{equation}
  m(\va{r}) = m_0 + \delta m(\va{r})
\end{equation}
we are assuming that the fluctuations \(  \delta m(\va{r})\) are small.
\begin{equation}
  \qty(\grad m)^2 = \qty(\grad \qty(m_0 + \delta m) )^2 = \qty(\grad \qty(\delta m) )^2
\end{equation}
\begin{equation}
  m^2 = m_0^2 + 2 m_0 \delta m + \qty(\delta m)^2
\end{equation}
\begin{equation}
  \beta \mathcal{H}_{eff} = V \underbrace{ \qty(a t m_0^2 + \frac{b}{2} m_0^4) }_{A_0}
  + \int_{}^{} \dd[D]{\va{r}} \qty(\frac{k}{2}\qty(\grad m)^2 + \qty(a t + 2 b m_0^2)\delta  m^2 + 2 b m_0 \delta m^3 + \frac{b}{2} \delta m^4 )
\end{equation}
\begin{equation}
  \qty( \underbrace{2 a t m_0}_{=0}  + \frac{b}{2} 4 m_0^3 )  \delta m
\end{equation}
the linear term  in \( m_0 \) is equal to zero by definition.

At \( T > T_c \) we know that \( m_0 = 0 \).
In this cases
\begin{equation}
  \beta \mathcal{H}_{eff}^> = \int_{}^{} \dd[D]{\va{r}} \qty(\frac{k}{2} \qty(\grad m)^2 + a t (\delta m)^2 + \cancel{\frac{b}{2} (\delta m)^4} )
\end{equation}
\begin{remark}
It is important to understand that these are fluctuations with respect to the solution.
\end{remark}
We cannot do again the saddle point, otherwise we do not get too much information.
We consider gaussian fluctuations: fluctuations that follow gaussian distribution.
Therefore, the term in \( (\delta m)^4 \)  is cancelled.
What it is the difference in the exponent respect the mean field?
Then we will to the same for \( T < T_c \), the story is the same. Now, we are just taking the gaussian term
\begin{equation}
  Z_{GL}^G = \int_{}^{} \text{D} \qty[\delta m] e^{- \int_{}^{} \dd[D]{r} \qty(\frac{k}{2} \qty(\grad \delta m)^2 + a t (\delta m)^2  )  }
 \end{equation}
Consider a system in a box of volume \( V = L^D \):
\begin{equation}
  \delta m (\va{r}) = \frac{1}{V} \sum_{}^{\va{k}}  e^{i \va{k}\vdot \va{r}} \delta m_{\va{k}}
\end{equation}
We have the integral
\begin{equation}
  \delta m_{\va{k}} = \int_{V}^{} \dd[D]{\va{r}} \delta m (\va{r}) e^{-i \va{k} \vdot \va{r} }
\end{equation}
with \( \va{k} = k_1, \dots, k_D = \frac{2 \pi  \bar{n} }{L}\)
We have \( \delta m_{\va{k}} \in \mathbb{C} \) but \( \delta m (\va{r}) \in \R \), hence
\begin{equation}
  \delta m_{\va{k}} = - \delta m_{-\va{k}}
\end{equation}
\begin{equation}
  \sum_{\va{k}}^{} \rightarrow \frac{V}{( 2 \pi )^D} \int_{\R}^{} \dd[]{\va{k}}
\end{equation}
\begin{equation}
  \frac{1}{V} \sum_{\va{k}}^{} e^{i \va{k} (\va{r}-\va{r}')} \rightarrow \frac{1}{V} \frac{V}{(2 \pi )^2} \int_{\R}^{} \dd[D]{\va{k}} e^{i \va{k} (\va{r}-\va{r}')}
  = \delta (\va{r}- \va{r}')
\end{equation}
\begin{equation}
  \frac{1}{V} \int_{}^{} \dd[D]{\va{r}} e^{i (\va{k} - \va{k}') \vdot  \va{r}} = \delta _{\va{k}\va{k}'}
\end{equation}
write immediately
\begin{equation}
  V \delta _{\va{k} \va{k}'} \overset{V \rightarrow \infty }{\longrightarrow  } (2 \pi )^D \delta (\va{k} - \va{k}')
\end{equation}
\begin{equation}
  a \Rightarrow \abs{\va{k}} \le \frac{\pi }{a} = \Lambda
\end{equation}
that is the ultraviolet cut-off.

Change now the notation:
\begin{equation}
  \delta m (\va{r}) \leftrightarrow \varphi (\va{r}), \quad k \leftrightarrow c
\end{equation}
and obtain
\begin{equation}
  \beta \mathcal{H}_{eff}^{G,>} = \int_{}^{} \dd[D]{\va{r}} \qty[\frac{c}{2} \qty(\grad \phi )^2 + a t \phi ^2 ]
\end{equation}
\begin{equation}
\begin{split}
  \int_{}^{} \dd[D]{r} \frac{c}{2} \qty(\grad \varphi )^2 & = \frac{c}{2} \frac{1}{V^2} \int_{}^{} \dd[D]{\va{r}} \qty(\grad \sum_{\va{k}}^{} e^{i \va{k} \vdot \va{r}} \varphi _{\va{k}}  ) \qty(\grad \sum_{\va{k}'}^{} e^{i \va{k}' \vdot \va{r} } \varphi _{\va{k}'}  )     \\
  & = \frac{c}{2} \sum_{\va{k} \va{k}'}^{}  \qty(- \va{k} \va{k}') \varphi _{\va{k}} \varphi _{\va{k}'} \underbrace{\int_{}^{} \dd[D]{\va{r}}  e^{i (\va{k}+\va{k}')\vdot \va{r}}}_{(2 \pi )^2 \delta (\va{k}+ \va{k}')}   \\
  & = \frac{c}{2V} \sum_{\va{k}}^{} \abs{\va{k}}^2 \varphi _{\va{k}} \varphi _{\va{-k}'}
\end{split}
\end{equation}
\begin{equation}
  \beta \mathcal{H}_{eff}^{G,>} \rightarrow \frac{1}{2V} \sum_{\va{k}}^{} \qty(2 a t + c k^2) \varphi _{\va{k}} \varphi _{-\va{k}'}
\end{equation}
\begin{equation}
  \int_{}^{} \text{D} \qty[\varphi (\va{r})] \rightarrow \int_{- \infty }^{+ \infty } \prod_{\abs{\va{k}} < \Lambda  }^{}   \dd[]{(\Re{\varphi _{\va{k}}})}   \dd[]{(\Im{\varphi _{\va{k}}})}
\end{equation}
with \( \varphi _{\va{k}} \in \mathbb{C} \).
\begin{equation}
  \varphi _{\va{k}}^* = \varphi _{-\va{k}}
\end{equation}
\begin{equation}
  \Re {\varphi _[\va{k}]} = \Re {\varphi _{-\va{k}}}, \quad   \Im{\varphi _[\va{k}]} = -\Im {\varphi _{-\va{k}}}
\end{equation}
\begin{equation}
  \Tr = \int_{- \infty }^{+ \infty } \prod_{\substack{ \abs{\va{k}} < \Lambda   \\ k_D > 0 } }^{}   \dd[]{\Re {\varphi _{\va{k}}}}  \dd[]{\Im {\varphi _{\va{k}}}}
\end{equation}
\begin{equation}
  \widetilde{Z}_{GC}^{G,>} = \frac{1}{2} \int_{- \infty }^{+ \infty } \prod_{\substack{ \va{k}   \\\abs{\va{k}} < \Lambda } }^{} \dd[]{\Re {\varphi _{\va{k}}}}  \dd[]{\Im {\varphi _{\va{k}}}}
  e^{- \beta \widetilde{\mathcal{H}}_{eff} [\varphi _{\va{k}}] }
\end{equation}
\begin{equation}
  x = \Re \varphi _{\va{k}}, \quad   y = \Im \varphi _{\va{k}}
\end{equation}
\begin{equation}
  \int_{-\infty }^{+\infty } \dd[]{x} \dd[]{y} e^{-A (x^2+y^2)} = \frac{\pi }{A}
\end{equation}
where \( A= \frac{1}{2V} \).
\begin{equation}
  e^{-\beta \widetilde{F}_{GL}^> } = \qty(\prod_{\substack{\va{k} \\ \abs{\va{k}} <\Lambda \\ k_D > 0  } }^{}  \frac{2 \pi V}{2 a t + c \abs{\va{k}}^2 } )
\end{equation}
\begin{equation}
  \widetilde{F}_{GL}^{G,>} = - \frac{1}{2} k_B T \sum_{\abs{\va{k}}> \Lambda  }^{}
  \log{\qty(\frac{2 \pi V}{2 a t + c \abs{\va{k}}^2 })}
\end{equation}
\begin{equation}
  c_V = - T \pdv[2]{F}{T} = \frac{A}{V} \sum_{\abs{\va{k}} < \Lambda  }^{}
  \frac{1}{2 a t + c \abs{\va{k}}^2 } - \frac{B}{V} \sum_{\abs{\va{k}}< \Lambda  }^{} \frac{1}{\qty(2 a t + c \abs{\va{k}}^2)^2 }
\end{equation}
Question: what happens if I introduce guassian fluctuations.

It turns out that when we study the asimptotic behaviour of these integrals.

For the first term it turns out that
\begin{equation}
1^{st} \propto
  \begin{cases}
   \xi ^{4-D} \sim t^{-\nu (4-D)}  & D < 4\\
  < \infty & D > 4
  \end{cases}
\end{equation}
\begin{equation}
2^{nd} \propto
  \begin{cases}
   \xi ^{2-D} \sim t^{-\nu (2-D)}  & D < 2\\
  < \infty & D > 2
  \end{cases}
\end{equation}
At the end, the behaviour of the specific heat
\begin{equation}
  c_V \sim \begin{cases}
    t^{-\nu (4-D)} & D < 4 \\
    \infty & D > 4
\end{cases}
\end{equation}











\end{document}
