\documentclass[../main/main.tex]{subfiles}

\newdate{date}{04}{12}{2019}


\begin{document}

\marginpar{ \textbf{Lecture 15.} \\  \displaydate{date}. \\ Compiled:  \today.}

As said, this first approximation is reasonable if either
\begin{enumerate}
\item \( \rho  \) is small enough. It implies that \( \abs{\va{r}_i - \va{r}_j} \gg 1 \) and hence \( \Phi _{ij} \ll 1 \).
\item Sufficiently high \( T \) such that \( \Phi (\abs{\va{r}_i - \va{r}_j} )/k_B T \ll 1 \). What is important it is the ration between \( \beta  \)  and \( \Phi _{ij} \).
\end{enumerate}
In either cases we have \( \exp (- \beta \Phi _{ij}) \rightarrow 1 \) and \( f_{ij} \rightarrow 0 \). By keeping only linear terms,  the configurational contribution to the partition function will be
\begin{equation*}
\begin{split}
  Q_N (V,T) &= \int_{V}^{} \dd[]{\va{r}_1} \dots \dd[]{\va{r}_N} \qty(1+ \sum_{i,j>i}^{} f_{ij} + \dots )
  = V^N + \sum_{i,j>i}^{} \int_{V}^{} \dd[]{\va{r}_1} \dots \int_{V}^{}  \dd[]{\va{r}_N} f_{ij} \\
  &= V^N + V^{N-2} \sum_{i,j>i}^{} \int_{V}^{} \dd[]{\va{r}_i} \dd[]{\va{r}_j} f_{ij} + \dots \\
\end{split}
\end{equation*}
We are summing up over all configurations \( ij \). Let us try to compute the double integral, with the definition of a new variable \( \va{r} = \va{r}_i - \va{r}_j \):
\begin{equation*}
  \int_{V}^{} \dd[]{\va{r}_i} \int_{V}^{} \dd[]{\va{r}_j} f_{ij} (\abs{\va{r}_i - \va{r}_j} ) \underset{\substack{ \text{translational} \\  \text{symmetry} } }{=}  \int_{V}^{} \dd[]{\va{r}_i} \int_{V}^{}  \dd[]{\va{r}} f (\va{r})
  = V \int_{V}^{} \dd[]{\va{r}}  f \qty(\abs{\va{r}} ) \equiv -2 B_2 V
\end{equation*}
Hence,
\begin{equation}
  B_2 \equiv - \frac{1}{2} \int_{V}^{} \dd[]{\va{r}}  f \qty(\abs{\va{r}} )
\end{equation}
From this we see precisely how the virial coefficient, which as we have already stated can be experimentally measured, is related to the microscopic properties of the interaction between the particles, represented by the Mayer function \( f \). It can also be shown that all the virial coefficients can be expressed in terms of integrals of products of Mayer functions: higher order coefficients involve the computation of increasingly difficult integrals, which can however be visualized in terms of graphs.

What we have seen now is how the cluster expansion works in general. Let us now apply it in order to find the virial expansion for real gases. From what we have found, the configurational partition function of the system becomes:
\begin{equation*}
  Q_N (V,T) = V^N - V^{N-1}  B_2 (T) \sum_{i,j>i}^{} 1 + \dots
\end{equation*}
The remaining sum is equal to\( N(N-1)\): in fact, for any of the \( N \) values that
\( i \) can assume,  \( j \)  can have \( N-1 \) values. These are all the possible connections (bonds) between pairs of particles \( (i,j) \) with \( j>i \).
Hence,
\begin{equation}
  Q_N (V,T) = V^N - V^{N-1}  B_2 (T)  N (N-1) + \dots
\end{equation}
and, considering that  \( N-1 \approx N \) for large \( N \), the complete partition function of the system will be:
\begin{equation}
  Z_N (V,T) = \qty(\frac{V^N}{N! \Lambda ^{3N}}) \qty(1- \frac{N^2}{V} B_2 (T)+ \dots)
\end{equation}
We recognise in this expression that \( (1-B_{2}N^{2}/V+\cdots ) \) is the correction \( \chi  \) to the ideal gas partition function that we have mentioned earlier; therefore, the free energy of the system will be:
\begin{equation}
  F_N = F_N^{ideal} - k_B T \ln{\qty[1-\frac{N^2}{V}B_2 (T)+ \dots] }
\end{equation}
and its pressure:
\begin{equation*}
  P_N = - \qty(\pdv{F_N}{V} )_{T,N} = \frac{N k_B T}{V} \qty(1+ \frac{\frac{N}{V}B_2}{1- \frac{N^2}{V}B_2})
  = \frac{N k_B T}{V} \qty(\frac{1 -\frac{N^2}{V} B_2 + \frac{N}{V} B_2 }{1- \frac{N^2}{V} B_2})
\end{equation*}
Expanding the denominator for \( \frac{N}{V} B_2 \ll 1 \) \( \rho \ll 1 \), one gets
\begin{empheq}[box=\myyellowbox]{equation}
  P_N \simeq \frac{N k_B T}{V} \qty(1+\frac{N}{V}B_2 + \dots)
  \label{eq:15_1}
\end{empheq}
here we see the correction to the ideal gas.
\begin{remark}
The equation \eqref{eq:15_1} gives an important relation between experimentally accessible observables as \( P_N \) and microscopic quantities such as \( f(\va{r}) \) (and hence \( \Phi (\va{r}) \)) trough the estimate of \( B_2 \).
Therefore, it is important computing \( B_2 \), because one time we have this we have the expansion. Or if we wish, by doing the fit of data at different temperature we obtain \( B_2 \) from the experiment and we can see \( f_{ij} \).
\end{remark}

The expansion in Eq.\eqref{eq:15_1} contains only low-order terms in the density \( N/V \), so strictly speaking it is valid only for low densities. To consider higher order terms in the virial expansion we need to consider higher order products of the \( f_{ij} \).
However, we can use a "trick" in order to extend its range; in fact, remembering that the McLaurin expansion \(  (1-x)^{-1} = 1 + x + \dots \), from the Eq.\eqref{eq:15_1} we can write:
\begin{equation*}
  \frac{PV}{N k_B T} \approx 1 + \rho B_2 + \dots \simeq \frac{1}{1-B_2 \rho }
\end{equation*}
and now re-expand \( (1-B_2 \rho )^{-1} \), so that we can express all the virial coefficients in terms of the first one:
\begin{equation*}
  \frac{1}{1-B_2 \rho } \simeq  1 + B_2 \rho + (B_2)^2 \rho ^2  + (B_2)^3 \rho ^3 + \dots
\end{equation*}
Hence,
\begin{equation*}
  \frac{P}{k_B T} = \rho  + B_2 \rho ^2 + (B_2)^2 \rho ^3 + (B_2)^3 \rho ^4 + \dots
\end{equation*}
Identifying the coefficients for each power we get, in the end:
\begin{equation*}
  B_3 \approx (B_2)^2, \quad B_4 \approx (B_2)^3, \quad \dots, \quad B_n \approx (B_2)^{n-1}
\end{equation*}
This is the approximation of higher order virial coefficients with powers of \( B_2 \).
\begin{remark}
One question at the exam can be: let us compute virial expansion of a gas in a potential.
\end{remark}


\subsection{ Computation of virial coefficients for some interaction potentials \( \pmb{\Phi } \)}
Let us now see this method in action by explicitly computing some coefficients \( B_2 \) for particular interaction potentials.

\subsubsection{Hard sphere potential}
The particles are interacting (it is not ideal!) and there is a size that is the range of the potential.
As a first trial, we use a hard sphere potential similar (see Figure \ref{fig:15_12}) to the one we have seen for the derivation of the Van der Waals equation:
\begin{equation}
  \Phi (r) = \begin{cases}
    \infty & r < \sigma \\
    0     & r \ge \sigma
\end{cases}
\end{equation}
(the difference with what we have seen in Van der Waals equation is that now the potential is purely repulsive, and has no attractive component).

\begin{figure}[h!]
\centering
\includegraphics[width=0.5\textwidth]{../lessons/15_image/1.pdf}
\caption{\label{fig:15_12} Plot of the hard sphere potential \( \Phi (r) \).}
\end{figure}

In this case,
\begin{equation}
  f(\va{r}) = e^{-\beta \Phi (r)} -1 = \begin{cases}
    -1 & r < \sigma\\
    0 & r \ge \sigma
  \end{cases}
\end{equation}
Therefore, from the definition of \( B_2 \) and shifting to spherical coordinates:
\begin{equation*}
  B_2 (T)= - \frac{1}{2} \int_{V}^{} \dd[]{\va{r}}  f \qty(\abs{\va{r}} )
  = - \frac{1}{2} 4 \pi \int_{0}^{+ \infty } \dd[]{r} r^2 \qty[e^{-\beta \Phi (r)}-1 ]
  = 2 \pi \int_{0}^{\sigma } \dd[]{r} r^2 = \frac{2}{3} \pi  \sigma ^3
\end{equation*}
Hence,
\begin{equation}
  \Rightarrow B_2^{HS} (T) = \frac{2}{3} \pi  \sigma ^3
\end{equation}
this is the second virial coefficient for a hard sphere gas.
As expected \(  B_2^{HS} \) does not depend on temperature (purely repulsive interaction). Finally, for hard spheres we have:
\begin{equation}
  P V = N k_B T \qty(1 + \frac{2}{3} \pi \sigma ^3 \frac{N}{V})
\end{equation}
Note that the excluded volume interaction (hard sphere term) increases the product \( PV \) with respect to the ideal gas.

\subsubsection{Square wall potential}
We now use a slight refinement of the previous potential:
\begin{equation}
   \Phi ({\vec {r}})={\begin{cases}+\infty &|{\va {r}}|<r_{0}\\-\varepsilon &r_{0}<|{\va {r}}|<r_{0}+\delta \\0&|{\va {r}}|>r_{0}+\delta \end{cases}}
\end{equation}
This can be seen as a hard sphere potential where the spheres have an attractive shell of thickness \( \delta  \). We thus have:
\begin{equation}
f({\va {r}})={\begin{cases}-1&|{\va {r}}|<r_{0}\\e^{\beta \varepsilon }-1&r_{0}<|{\va {r}}|<r_{0}+\delta \\0&|{\va {r}}|>r_{0}+\delta \end{cases}}
\end{equation}
so that:
\begin{equation*}
\begin{split}
  B_{2} & =-{\frac {1}{2}}\int f(|{\va {r}}|)d{\va {r}}=-{\frac {1}{2}}\int 4\pi r^{2}f(r)dr=\\
  &=-2\pi \left[\int _{0}^{r_{0}}(-r^{2})dr+\int _{r_{0}}^{r_{0}+\delta }\left(e^{\beta \varepsilon }-1\right)r^{2}dr\right]=\\
  &=-2\pi \left\lbrace -{\frac {r_{0}^{3}}{3}}+{\frac {e^{\beta \varepsilon }-1}{3}}\left[(r_{0}+\delta )^{3}-r_{0}^{3}\right]\right\rbrace =B_{2}^{\text{h.s.}}-{\frac {2}{3}}\pi \left(e^{\beta \varepsilon }-1\right)\left[(r_{0}+\delta )^{3}-r_{0}^{3}\right]
\end{split}
\end{equation*}
where  \( B_2^{HS} \) is the first virial coefficient of the hard sphere potential we have previously seen. Now, if the temperature is sufficiently high, namely \(  \beta \varepsilon \ll 1\), we can approximate \( e^{\beta \varepsilon }-1\approx \beta \varepsilon \), so that:
\begin{equation}
  B_{2}=B_2^{HS} -{\frac {2}{3}}\pi \beta \varepsilon r_{0}^{3}\left[\left(1+{\frac {\delta }{r_{0}}}\right)^{3}-1\right]
\end{equation}
For the sake of simplicity, defining:
\begin{equation*}
  \lambda \equiv \left(1+{\frac {\delta }{r_{0}}}\right)^{3}-1
\end{equation*}
we will have, in the end:
\begin{equation}
  { \frac {PV}{Nk_{B}T}}=1+B_{2}\rho =1+\left(B_2^{HS} -{\frac {2}{3}}{\frac {\pi \varepsilon }{k_{B}T}}r_{0}^{3}\lambda \right)\rho
\end{equation}
so in this case  \( B_2 \) actually depends on the temperature.



\subsubsection{Lennard-Jones potential}
This potential is a quite realistic representation of the interatomic interactions. It is defined as:
\begin{equation}
  \Phi = 4 \varepsilon \qty[\qty(\frac{\sigma }{r})^{12} - \qty(\frac{\sigma }{r})^6  ]
\end{equation}
which contains a long-range attractive term (the one proportional to \( 1/r^6 \), which can be justified in terms of electric dipole fluctuations) and a short-range repulsive one (proportional to \( 1/r^{12} \), which comes from the overlap of the electron orbitals, i.e.Pauli excluded principle). This potential is plotted in Figure \ref{fig:15_13}.  The minimum is in \( r_{min}=2^{1/\sigma } \).   We can play with the range of attraction by changing \( \sigma  \) or by changing the   \( \varepsilon  \).


\begin{figure}[h!]
\centering
\includegraphics[width=0.5\textwidth]{../lessons/15_image/2.pdf}
\caption{\label{fig:15_13} Plot of the Lennard-Jones potential \( \Phi  \).}
\end{figure}
% What it is important is that for the Lennard-Jones we have
% \begin{equation}
%   B_2 \overset{LJ}{=} B_2 (T)
% \end{equation}

With this interaction potential, the first virial coefficient is:
\begin{equation*}
  B_2 (T) = - 2 \pi \int_{0}^{\infty } r^2 \qty[e^{- \frac{4 \varepsilon }{k_B T} \qty[\qty(\frac{\sigma }{r})^{12} - \qty(\frac{\sigma }{r})^6] } -1] \dd[]{r}
\end{equation*}
which is not analytically computable. However, it can be simplified defining the variables
\begin{equation*}
  x = \frac{r}{\sigma }, \qquad \tau = \frac{k_B T}{\varepsilon }
\end{equation*}
 so that, integrating by parts \( \int_{}^{} f' g = fg - \int_{}^{} g' f    \) where \( f' = x^2 g = \exp [-()]   \), we obtain
\begin{equation*}
\begin{split}
  B_2 (T^*) &= \frac{2}{3} \pi \sigma ^3 \frac{4}{\tau } \int_{0}^{\infty } x^2  \qty(\frac{12}{x^{12}}- \frac{6}{x^6}) e^{- \frac{4}{\tau } \qty(\frac{1}{x^{12}} - \frac{1}{x^6}) }  \dd[]{x}  \\
  & = A \int_{0}^{\infty } \qty(\frac{12}{x^{16}}- \frac{6}{x^4}) e^{- \frac{4}{\tau } \qty(\frac{1}{x^{12}} - \frac{1}{x^6}) } \dd[]{x}
\end{split}
\end{equation*}
Now, we can expand the exponential and integrate term by term; this gives an expression of \( B_2 \) as a power series of \( 1/\tau  \):
\begin{equation}
  B_2 (\tau ) = - 2 A' \sum_{n=0}^{\infty } \frac{1}{4n!} \Gamma \qty(\frac{2n-1}{4}) \qty(\frac{1}{\tau })^{\frac{2n+1}{4}}
\end{equation}
where \( \Gamma  \) is the Euler function and \( A' \) is a constant.  Note that the attractive part of the Lennard-Jones potential has introduced in \( B_2 \) a dependence on the temperature.





\subsection{Higher order terms in the cluster expansion}
Let us consider again the formal expansion
\begin{equation*}
  \prod_{i}^{} \qty(\prod_{j>i}^{} (1+f_{ij}) ) = 1 + \sum_{i,j>i}^{} f_{ij}
  + \sum_{\substack{ i \\ j>i \\ l>k \\ k \ge i \\ (ij) \neq (kl)} }^{} f_{ij} f_{kl}     + \dots
\end{equation*}
The problem with this expansion is that it groups terms quite different from one another. Fro example the terms \( f_{12}f_{23} \) and \( f_{12}f_{34} \). Indeed the first term correspond to a diagram as in Figure \ref{fig:15_1_1}, while the second to two disconnected diagrams as in Figure \ref{fig:15_1_2}.
\begin{figure}[h!]
\begin{minipage}[c]{0.5\linewidth}
\subfloat[][Diagram of \( f_{12}f_{23} \). ]{ \includegraphics[width=0.6\textwidth]{../lessons/15_image/3.pdf}  \label{fig:15_1_1} }
\end{minipage}
\begin{minipage}[]{0.5\linewidth}
\centering
\subfloat[][Diagram of \(  f_{12}f_{34} \).]{\includegraphics[width=0.6\textwidth]{../lessons/15_image/4.pdf}  \label{fig:15_1_2} }
\end{minipage}
\caption{\label{fig:} }
\end{figure}

Another problem of the above expansion is that it does not recognize identical clusters formed by different particles. For example the terms \( f_{12} f_{23} \) and \( f_{12}f_{14} \) contribute in the same way to the partition function. It is then convenient to follow a diagrammatic approach similar to the Feymann approach in the reciprocal space.

\begin{minipage}[c]{0.7\linewidth}
For the linear term \( f_{ij} \) the only diagram is given by Figure \ref{fig:15_2}. As we have seen this has multeplicity
\begin{equation*}
\frac{N(N-1)}{2}
\end{equation*}
and the integral is of the form
\begin{equation*}
  \int_{}^{}  f_{12} \dd[]{\va{r}_1} \dd[]{\va{r}_2}  = V \int_{}^{} f(\va{r}) \dd[]{\va{r}}  = - 2 V B_2
\end{equation*}
\end{minipage}
\begin{minipage}[]{0.3\linewidth}
\centering
\includegraphics[width=0.3\textwidth]{../lessons/15_image/5.pdf}
\captionof{figure}{\label{fig:15_2} }
\end{minipage}

\vspace{0.5cm}
\begin{minipage}[c]{0.7\linewidth}
For the term \( f_{ij}f_{kl} \) we can have the case as in Figure \ref{fig:15_3},
that has molteplicity
\begin{equation*}
  \frac{N(N-1)}{2} \frac{(N-1)(N-3)}{2} \frac{1}{2}
\end{equation*}
 and the integral is of the  form
\begin{equation*}
  \int_{}^{}  f_{12} f_{34} \dd[]{\va{r}_1} \dd[]{\va{r}_2}  \dd[]{\va{r}_3}  \dd[]{\va{r}_4}
\end{equation*}
i.e. involving 4-particles
\begin{equation*}
\begin{split}
 &  \int_{}^{}  f( \abs{\va{r}_1 - \va{r}_2} )f( \abs{\va{r}_3 - \va{r}_4} ) \dd[]{\va{r}_1}  \dd[]{\va{r}_2}  \dd[]{\va{r}_3}  \dd[]{\va{r}_4}
  = \\
  & = V^2 \qty(\int_{}^{} f(\va{r})\dd[]{\va{r}}  )^2 = 4 V^2 B_2^2
\end{split}
\end{equation*}
\end{minipage}
\begin{minipage}[]{0.3\linewidth}
\centering
\includegraphics[width=0.6\textwidth]{../lessons/15_image/6.pdf}
\captionof{figure}{\label{fig:15_3} }
\end{minipage}

\vspace{0.5cm}

\begin{minipage}[c]{0.7\linewidth}
The next case if for instance as in Figure \ref{fig:15_4}. This involves 3 particles.
The multiplicity of this diagram is
\begin{equation*}
  \frac{N (N-1) (N-2)}{3!} \times 3
\end{equation*}
The integral is of the form
\begin{equation}
\begin{split}
  & \int_{}^{}  f_{12} f_{23} \dd[]{\va{r}_1}  \dd[]{\va{r}_2}   \dd[]{\va{r}_3}
  \simeq V  \qty(  \int_{}^{} \dd[]{r} f(r)  )^2  = \\
  & =   \int_{}^{}  f( \abs{\va{r}_1 - \va{r}_2} )f( \abs{\va{r}_2 - \va{r}_3} ) \dd[]{\va{r}_1}  \dd[]{\va{r}_2}  \dd[]{\va{r}_3}  = \\
  & = V \qty(\int_{}^{} f(\va{r})\dd[]{\va{r}}  )^2  = 4 V B_2^2
  \label{eq:15_3}
\end{split}
\end{equation}
\end{minipage}
\begin{minipage}[]{0.3\linewidth}
\centering
\includegraphics[width=0.8\textwidth]{../lessons/15_image/7.pdf}
\captionof{figure}{\label{fig:15_4} }
\end{minipage}

\vspace{0.5cm}

\begin{minipage}[c]{0.7\linewidth}
Another interesting diagram is the one in Figure \ref{fig:15_5}. Its molteplicity is
\begin{equation*}
  \frac{N(N-1)(N-2)}{3!}
\end{equation*}
The associated integral involves 3 particles and it is of the form
\begin{equation*}
\begin{split}
   & \int_{}^{} f_{12} f_{23} f_{31} \dd[]{\va{r}_1}  \dd[]{\va{r}_2} \dd[]{\va{r}_3} = \\
   & = \int_{}^{} f ( \abs{\va{r}_1 - \va{r}_2} ) f ( \abs{\va{r}_2 - \va{r}_3} )  f ( \abs{\va{r}_3 - \va{r}_1} )  \dd[]{\va{r}_1} \dd[]{\va{r}_2} \dd[]{\va{r}_3} \\
   & = \int_{}^{} f ( \abs{\va{r}_1 - \va{r}_2} ) f ( \abs{\va{r}_2 - \va{r}_3} )  f ( \abs{\va{r}_3 - \va{r}_1} )  \dd[]{\va{r}_2} \dd[]{\va{r}_{21}} \dd[]{\va{r}_{23}}
\end{split}
\end{equation*}
\end{minipage}
\begin{minipage}[]{0.3\linewidth}
\centering
\includegraphics[width=0.8\textwidth]{../lessons/15_image/8.pdf}
\captionof{figure}{\label{fig:15_5} }
\end{minipage}

\noindent On the other hand \( \va{r}_{13} = \va{r}_{23} - \va{r}_{21} \), which implies
\begin{equation*}
   f ( \abs{\va{r}_3 - \va{r}_1} )  =  f ( \abs{\va{r}_{23} - \va{r}_{21}} )
\end{equation*}
Hence,
\begin{equation*}
  \int_{}^{} f (\abs{\va{r}_{12}} ) f (\abs{\va{r}_{23}} ) f (\abs{\va{r}_{31}} ) \dd[]{\va{r}_{21}}  \dd[]{\va{r}_{23}}  \dd[]{\va{r}_{2}}
   =  \int_{}^{} f (\abs{\va{r}_{12}} ) f (\abs{\va{r}_{23}} ) f (\abs{\va{r}_{23}-\va{r}_{21}} )  \dd[]{\va{r}_{21}}  \dd[]{\va{r}_{23}}  \dd[]{\va{r}_{2}}
\end{equation*}
Let us call this integral
\begin{equation}
  \int_{}^{} f_{12} f_{23} f_{31} \dd[]{\va{r}_1}  \dd[]{\va{r}_2}   \dd[]{\va{r}_3}
  \equiv  3! V  \qty( B_3 - 2 B_2^2  )
  \label{eq:15_4}
\end{equation}
The configurational partition function with the terms in Eq.\ref{eq:15_3} and Eq.\ref{eq:15_4} becomes
\begin{small}
\begin{equation}
\begin{split}
Q_N (V,T) = &  V^N - V^N \frac{N(N-1)}{V} B_2 + V^N\frac{ N (N-1)(N-2)(N-3)}{8V^2} (4B_2^2)
 + V^N \frac{N(N-1)(N-2)}{2V^2} 4B_2^2 \\
& = V^N \qty(1 + \frac{N(N-1)}{V} B_2 + \frac{N(N-1)(N-2)(N-3)}{2V^2} B_2^2 + \frac{N(N-1)(N-3)}{V^2} B_3)
\end{split}
\end{equation}
\end{small}

Let us now face the problem in a slightly different ways. Let us remind that
\begin{equation}
  Q_N (V,T) = \sum_{diagrams}^{} \int_{}^{} \prod_{kl}^{} f_{kl} \dd[3N]{r}
\end{equation}
where the sum is over all possible diagrams, i.e. all possible ways in which ones can draw edges between pairs of points \( (k,l) \).
For each such diagrams I have to product between all edge and then integrate over the configurational space (\( N \)  points).

 Let us now consider only \emph{connected} diagrams for \( i \) sites. In other words given \( i \) points (\( i \) particles)  from a system of \( N \) points and I consider all the possible ways I can connect these \( i  \) points (an example is shown in Figure \ref{fig:15_6}).
\begin{figure}[h!]
   \centering
   \includegraphics[width=0.7\textwidth]{../lessons/15_image/9.pdf}
   \caption{\label{fig:15_6} Example of connected diagrams for \( i=4 \) sites.}
   \end{figure}

For each diagram we take the product \( \prod_{kl}^{} f_{kl}   \) and then integrate over the position of the \( i \) points (\( i \) particles). For a fixed diagram:
\begin{equation*}
  \int_{}^{} \prod_{kl \in \, diagram }^{} f_{kl} \dd[]{\va{r}_1} \dots \dd[]{\va{r}_i}
\end{equation*}

\begin{example}{Diagram for \( \pmb{i=4} \) sites}{}
  For example the diagram 1 in Figure \ref{fig:15_6} gives the contribution
  \begin{equation*}
    \int_{}^{} f_{12} f_{13} f_{34} \dd[]{\va{r}_1}  \dd[]{\va{r}_2} \dd[]{\va{r}_3} \dd[]{\va{r}_4}
  \end{equation*}
  The diagram 2 gives
  \begin{equation*}
    \int_{}^{} f_{12} f_{13} f_{23} f_{34} \dd[]{\va{r}_1}  \dd[]{\va{r}_2} \dd[]{\va{r}_3} \dd[]{\va{r}_4}
  \end{equation*}
  and so on.
\end{example}

\noindent Finally, we sum over all these connected diagrams of \( i \) points:
\begin{equation*}
  \sum_{\substack{ \text{connected} \\  \text{diagrams} } }^{}    \int_{}^{} \prod_{lk \in \, diagram }^{} f_{kl} \dd[]{\va{r}_1} \dots \dd[]{\va{r}_i}
\end{equation*}
the results is what we call \( (i! V B_i) \) and defines \( B_i \). Let us analyze what happens for different values of \( i \) points:
\begin{itemize}
\item case \( i=1 \): clearly  \( B_1 =1 \);
\item case \( i=2 \): just one edge, hence we have just one connected diagram. The integral becomes:
\begin{equation*}
  \int_{}^{} f_{12} \dd[]{\va{r}_1} \dd[]{\va{r}_2} = - 2 V B_2
\end{equation*}
\item case \( i=3 \): the connected diagrams are shown in Figure \ref{fig:15_7}.
\begin{figure}[h!]
\centering
\includegraphics[width=0.8\textwidth]{../lessons/15_image/10.pdf}
\caption{\label{fig:15_7} Connected diagrams for \( i=3 \) points.}
\end{figure}

\begin{equation*}
\begin{split}
  & \sum_{\substack{ \text{connected diagrams } \\ \text{of \( i=3 \) points}} }^{} \int_{}^{} \prod_{kl \in \, diagram }^{} f_{kl} \dd[]{\va{r}_1}   \dd[]{\va{r}_2}    \dd[]{\va{r}_3} = \\
   &=  \underbrace{ \int_{}^{} f_{12} f_{23} \dd[]{\va{r}_1} \dd[]{\va{r}_2} \dd[]{\va{r}_3}
    + \int_{}^{} f_{12}f_{13} \dd[]{\va{r}_1} \dd[]{\va{r}_2} \dd[]{\va{r}_3}
    + \int_{}^{} f_{13}f_{23} \dd[]{\va{r}_1} \dd[]{\va{r}_2} \dd[]{\va{r}_3} }_{3 V \qty(\int_{}^{} f (\va{r})\dd[]{\va{r}}  )^2 }   \\
  & + \underbrace{\int_{}^{} f_{12} f_{23} f_{13} \dd[]{\va{r}_1} \dd[]{\va{r}_2} \dd[]{\va{r}_3}}_{3! V (B_3 - 2 B_2^2)}
\end{split}
\end{equation*}
Hence,
\begin{equation*}
\begin{split}
  \sum_{\substack{ \text{connected diagrams } \\ \text{of \( i=3 \) points}} }^{} \int_{}^{} \prod_{kl \in \, diagram }^{} f_{kl} \dd[]{\va{r}_1}   \dd[]{\va{r}_2}    \dd[]{\va{r}_3}
  & = 3 V (-2B_2)^2 + 6V (B_3 - 2 B_2) \\
  & = 6V B_3 = 3! V B_3
\end{split}
\end{equation*}

\end{itemize}

Eventually, for the partition function we have to sum over all possible clusters.
One possible procedure is:
\begin{enumerate}
\item given the \( N \) points we can partition them into connected clusters. For all \( i \) points we can make \( m_i \) clusters of that size \( i \).
\begin{equation*}
  \sum_{i}^{} i m_i = N
\end{equation*}
For each cluster of size \( i \) we have a term \( (i!VB_i) \). If there are \( m_i \) of them we have a weight \( (i! V B_i)^{m_i} \).
\item Now, we have to count in how many ways we can make the partition of \( N \) in a set of \( \{ m_i \}   \) clusters. Clearly if we permute the label of the \( N \) vertices we have possible different clusters. In principle, this degenerancy is proportional to \( N \)!

On the other hand, if one changes the order of the labels within a cluster (in \( i! \) ways) this does not change the cluster and since there are \( m_i \) clusters of size \( i \) we have to divide by \( (i!)^{m_i} \).

Moreover, since there are \( m_i \) clusters one can swap them (in \( m_i! \) ways). The degenerancy is \( \frac{N!}{m_i! (i!)^{m_i}} \). Therefore,
\begin{equation}
  Q_N (V,T) = \sum_{\{ m_i \}  }^{} \prod_{i}^{}    \frac{N!}{m_i! (i!)^{m_i}} \qty(i! V B_i)^{m_i}
\end{equation}
\end{enumerate}



\begin{figure}[h!]
\begin{minipage}[c]{0.5\linewidth}
\subfloat[][Description]{ \includegraphics[width=0.8\textwidth]{../lessons/15_image/11.pdf}  \label{fig:15_8} }
\end{minipage}
\begin{minipage}[]{0.5\linewidth}
\centering
\subfloat[][Description]{\includegraphics[width=0.8\textwidth]{../lessons/15_image/12.pdf}  \label{fig:15_9} }
\end{minipage}
\caption{\label{fig:} }
\end{figure}


\begin{exercise}{\( \pmb{N=9} \) points}{}
Consider the \( N=9 \) points in Figure \ref{fig:15_8}.
  \begin{enumerate}
  \item Partition these points into clusters, as in Figure \ref{fig:15_9}.

   For this partition \( \{ m_i \}   \) we have \( m_4=1,m_2=2,m_1=1 \). Now, the cluster of size 4 can be connected in a given different ways \( (4! V B_4)^1 \).
   \item Compute the degenerancy of this case (More on Huang chapter 10).
  \end{enumerate}
\end{exercise}














\chapter{Landau theory of phase transition for homogeneous systems}

\section{Landau theory of phase transition (uniform systems)}

It is a phenomenological mean field theory of phase transitions for uniform systems (no spatial variation of the order parameter). It is based on the following assumptions:
\begin{enumerate}
\item Existence of an unfirom order parameter \( \eta   \). Remember the definition of the order parameter:
\begin{equation}
  \eta   = \begin{cases}
    0 & T \ge \bar{T} \text{ (disordered or symmetric phase)}\\
    \neq 0 & T < \bar{T}  \text{ (ordered symmetry is broken)}
\end{cases}
\end{equation}
Well known examples are
\begin{equation}
  \begin{cases}
   \eta \rightarrow m \\
  \eta  \rightarrow \rho _L - \rho _G
  \end{cases}
\end{equation}
\item The free energy is an \emph{analyc function}  of the order parameter \( \eta   \). It is because you are doing the expansion close to...etc etc. Therefore, \( \mathcal{L} = \mathcal{L} (\eta ) \).
\item The form of \( \mathcal{L} \) must satisfy the underlying symmetry of the system.
\item Equilibrium states correspond to the absolute minima of \( \mathcal{L} \).
\end{enumerate}
\begin{remark}
Since \( \mathcal{L} \) os analytic it can be formally expanded in power of \( \eta  \), for \( \eta \sim 0 \).
\begin{equation}
  \mathcal{L} (\eta ) \approx a_0 + a_1 \eta + a_2 \eta ^2 + a_3 \eta ^3 + \dots
\end{equation}
\end{remark}

\section{Symmetries}
To fix the ideas let us consider the theory for the Ising model. In this case \( \eta  \) is a scalar (magnetization).

For \( T > \bar{T}  \) (critical point) we expect a paramagnetic phase. \( \mathcal{L} \) has a minimum at \( \eta =0 \), hence
\begin{equation}
  \pdv{\mathcal{L}}{\eta } = a_1 + 2 a _2 \eta + 3 a _3 \eta ^2 + \dots = 0
\end{equation}
\( \eta = 0 \) is a solution if and only if \( a_1 = 0 \).
 \begin{remark}
No linear term must be present!
\end{remark}
Since Ising has \( \mathbb{Z}^2 \) symmetry, we should require
\begin{equation}
  \mathcal{L} (-\eta ) =   \mathcal{L} (\eta )
\end{equation}
which implies
\begin{equation}
  a_k = 0 \quad \forall k  \text{ odd}
\end{equation}
Moreover, since \( \mathcal{L} \) is analytic, terms proportional to \( \abs{\eta }  \)  are excluded.
The minimal expression for \( \mathcal{L}(\eta ) \) that describes the equilibrium phase diagram of an Ising-like system is
\begin{equation}
  \mathcal{L} (\eta ) \simeq a_0 (J,T) + a_2 (J,T) \eta  ^2 + a_4  (J,T) \eta  ^4 + O(\eta ^6)
\end{equation}
The coefficients of the expansion are functions of the physical parameters, \( J \) and \( T \).

Since for \( T > \bar{T}  \)  , \( \eta = \bar{\eta } = 0  \) and
\begin{equation}
  \mathcal{L} (\eta =0) = a_0
\end{equation}
\( a_0 (T,J) \) value of \( \mathcal{L} \) in the paramagnetic phase. Since what matters is the free-energy difference we can put \( a_0 =0 \) identically.

Moreover, in order to have \( \eta = \bar{\eta } \neq 0 < \infty   \) for \( T < \bar{T} \) (thermodynamic stability) we should impose that the coefficient of the highest power of \( \eta  \) is always positive. In this case
\begin{equation}
  a_4 (J,T)>0
\end{equation}
   Indeed if this condition is violated \( \mathcal{L} \) reaches it s absolute minimum for \( \abs{\eta } \rightarrow \infty   \)!
Therefore,
\begin{equation}
    \mathcal{L} (\eta ) \simeq  a_2 \eta  ^2 + a_4  \eta  ^4
\end{equation}
where he term \( a_4 \) it is positive and fixed.

If we now fix \( J \) and expand the coefficients \( a_2 \) and \( a_4 \) as a function of \( t \equiv \frac{T - \bar{T} }{\bar{T} } \),
\begin{subequations}
\begin{align}
  a_2 & \sim  a_2^0 +    \frac{T - \bar{T} }{\bar{T} } \frac{a}{2} \\
  a_4 & \sim \frac{b}{4} + \dots
\end{align}
\end{subequations}
By choosing \( a_2^0 = 0\) the sign of \( a_2 \) is determined by the one of
\begin{equation}
  t \equiv \frac{T - \bar{T} }{\bar{T} }
\end{equation}
In particular, at \( T = \bar{T}  \), one has \( a_2 =0 \).
Hence, for scalar, Ising like systems the minimal Landau free energy is given by
\begin{equation}
    \mathcal{L} = \mathcolorbox{green!20}{\frac{a}{2}} t  \eta ^2 + \frac{b}{4} \eta ^4 + O ( \eta ^6)
\end{equation}
\begin{remark}
Does not matter the coefficient in green in front, so in the next part of the course we will change it. If it is written in this way we have always \( a>0 \). We have also \( b>0 \).
\end{remark}



\section{Equilibrium phases}
Now, the equilibrium states
\begin{equation}
  \pdv{\mathcal{L}}{\eta } = 0 \quad \Rightarrow  a t \eta + b \eta ^3 = 0
\end{equation}
Hence,
\begin{equation}
  \bar{\eta } =
  \begin{cases}
   0 & T > \bar{T} \\
   \pm \sqrt{\frac{-at}{b}} & T < \bar{T}
  \end{cases}
\end{equation}
At \( T= \bar{T}  \)  the 3 solutions coincide!
\begin{itemize}
\item Case \( T> \bar{T}  \) \( (t>0) \): only one solution \( \bar{\eta }=0  \).
\begin{equation}
  \pdv[2]{\mathcal{L}}{\eta } = a t + 3 b \eta ^2 \ge 0
\end{equation}
for \( \bar{\eta } = 0  \) and \( t>0 \) implying that \( \eta = \bar{\eta }  \) is a global minimum, as in Figure \ref{fig:15_10}.
\begin{figure}[h!]
\centering
\includegraphics[width=0.5\textwidth]{../lessons/15_image/13.pdf}
\caption{\label{fig:15_10} Description.}
\end{figure}
\item Case \( T< \bar{T}  \) \( (t<0) \): 3 solutions \( \bar{\eta } = 0  \) and \( \bar{\eta } = \pm \sqrt{- \frac{at}{b}}   \).
Let us see wheter they are minima or local maxima.
\begin{equation}
  \eval{\pdv[2]{\mathcal{L}}{\eta } }_{\bar{\eta } =0 } = a t < 0 \quad \Rightarrow \bar{\eta } = 0 \text{ local maximum (no equilibrium)}
\end{equation}

\begin{equation}
  \eval{\pdv[2]{\mathcal{L}}{\eta } }_{\bar{\eta } = \pm \sqrt{- \frac{at}{b}}  } = a t + 3 b \qty(- \frac{at}{b}) = -2at
\end{equation}
since \( t<0 \), \( -2at >0 \) and hence \( \bar{\eta } = \pm \sqrt{- \frac{at}{b}}   \) are 2 minima!

\begin{figure}[h!]
\centering
\includegraphics[width=0.5\textwidth]{../lessons/15_image/14.pdf}
\caption{\label{fig:15_11} Description.}
\end{figure}

\begin{equation}
  \mathcal{L} \qty(\eta = \pm \sqrt{- \frac{at}{b}} ) = - \frac{a^2 t^2}{2b} + \frac{a^2 t^2}{4b} = - \frac{a^2 t^2}{4b} < 0
\end{equation}
The 2 minima are related by the group symmetry \( \mathbb{Z}^2 \)  \( (\bar{\eta } \rightarrow - \bar{\eta }  ) \).
\end{itemize}

\begin{remark}
Note that, in presence of an external magnetic field \( h \), one should consider the Legendre transform of \( \mathcal{L} \) obtaining its Gibbs version:

\begin{equation}
  \mathcal{L} =  \frac{a}{2} t \eta ^2 + \frac{b}{4} \eta ^4 - h \eta
\end{equation}
we have inserted a field coupled with the order parameter.
\end{remark}

\section{Critical exponents in Landau's theory}
Consider \( t \equiv \frac{T- \bar{T} }{\bar{T} } \).

\subsubsection{Exponent \( \beta  \)}
We have  \( \eta \sim t^\beta  \) for \( h=0 \), \( t \rightarrow 0^- \). Since \( t<0 \), the minima of \( \mathcal{L} \) are
\begin{equation}
  \bar{\eta } = \pm \sqrt{- \frac{at}{b}} \quad \Rightarrow  \beta = \frac{1}{2}
\end{equation}
as expected.
\subsubsection{Exponent \( \alpha  \)}
We have \( C \sim t^{-\alpha } \) for  \( h=0 \), \( \abs{t}  \rightarrow 0 \). Two cases:
\begin{itemize}
\item \( t>0 \): \( \bar{\eta }=0  \) and \( \mathcal{L} (\bar{\eta } )= 0 \).
\item \( t < 0 \): \( \mathcal{L}_{min} = \mathcal{L} \qty(\bar{\eta } = \pm \sqrt{- \frac{at}{b}}  ) = - \frac{a^2t^2}{4b}  \), that implies
\begin{equation}
  \mathcal{L}_{min} =
    \begin{cases}
     0 & t > 0\\
     - \frac{a^2 t^2}{4b} & t < 0
    \end{cases}
\end{equation}
Hence,
\begin{equation}
  c_V = - T \pdv[2]{\mathcal{L}}{T} = - T \pdv[2]{}{T} \qty(- \frac{a^2}{4b \bar{T}^2 }(T- \bar{T} )^2)
\end{equation}
We have
\begin{equation}
  \pdv{}{T} (\dots) = - \frac{a^2}{2 b \bar{T}^2 } (T- \bar{T} )
\end{equation}
\begin{equation}
  \pdv[2]{}{T} = \pdv{}{T} \qty[ - \frac{a^2}{2 b \bar{T}^2 } (T- \bar{T} )]   = - \frac{a^2}{2 b \bar{T}^2 }
\end{equation}
\begin{equation}
c_V
  \begin{cases}
   0 & T > \bar{T} \\
   \frac{a^2}{2 b \bar{T}^2 } T & T < \bar{T}
  \end{cases}
\end{equation}
We have \( t \rightarrow 0^- \) if and only if \( T \rightarrow \bar{T}^-  \), which implies \( c_V \rightarrow \frac{a^2}{2b \bar{T} } \) that is constant.
\end{itemize}
In obth cases \( \alpha =0 \).
\subsubsection{Exponent \( \delta  \)}
We have \( h \sim \eta ^ \delta  \) at \( T = \bar{T}  \). Let us start from the equation of state. This is obtained by computing the \( \pdv{}{\eta }  \) of the Gibbs version.
\begin{equation}
  \pdv{\mathcal{L}_G}{\eta } = a t \eta + b \eta ^3 - h = 0
\end{equation}
that is the condition for equilibrium. The equation of state is
\begin{equation}
  h = a t \eta + b \eta ^3
  \label{eq:15_2}
\end{equation}
Equation \eqref{eq:15_2} tells us that, for fixed \( h \), the extreme points of \( \mathcal{L} \) are given by the values of \( \eta  \) that satisfies \eqref{eq:15_2}.

\begin{figure}[h!]
\centering
\includegraphics[width=0.5\textwidth]{../lessons/15_image/15.pdf}
\caption{\label{fig:} Description.}
\end{figure}
At \( T= \bar{T}  \) \( (t=0) \) we have \( h \sim \eta ^3 \), hence \( \delta =3 \).

\subsubsection{Exponent \( \gamma   \)}
\( \chi _T \sim t^{- \gamma  } \) for \( h=0 \), \( \abs{t} \rightarrow 0  \). Let us derive the equation of state \eqref{eq:15_2} with respect to \( h \):
\begin{equation}
  a t \pdv{\eta }{h} + 3 b \eta ^3 \pdv{\eta }{h} = 1
\end{equation}
and since
\begin{equation}
  \chi = \pdv{\eta }{h}
\end{equation}
we have
\begin{equation}
  \chi = \frac{1}{at+3 b \eta ^2}
\end{equation}
\begin{itemize}
\item Case \( t>0 \), \( \bar{\eta } = 0  \): \( \chi _T = - \frac{1}{at} \).
\item Case \( t<0 \), \( \bar{\eta } = \pm \qty(- \frac{at}{b})^{1/2}   \): \( \chi _T = - \frac{1}{2at} \).
\end{itemize}
In both cases \( \chi _T \sim  1/t \) and this gives
\begin{equation}
  \gamma = \gamma' = 1
\end{equation}

\subsubsection{Summary}
In summary the Landau theory gives the following (mean field) values of the critical exponents
\begin{equation}
  \beta = \frac{1}{2}, \quad \alpha =0, \quad \delta =3, \quad \gamma =1
\end{equation}
Landau theory does not depend on the system dimension \( D \) (as expected since isa mean field theory) but only on its symmetries.

\begin{remark}
For a \( O(n) \) (vector) model the order parameter \( \eta  \) becomes a vector field \( \va{\eta } \) with \( n \) compnents and
\begin{equation}
  \mathcal{L}_G ( \va{\eta }) = \frac{a}{2} t \va{\eta } \vdot \va{\eta } + \frac{b}{4} \qty(\va{\eta } \vdot \va{\eta } )^2 - \va{h} \vdot \va{\eta } + O \qty(\qty(\va{\eta } \vdot \va{\eta })^3 )
\end{equation}
\end{remark}










\end{document}
