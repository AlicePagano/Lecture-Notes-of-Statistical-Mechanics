\documentclass[../main/main.tex]{subfiles}

\newdate{date}{04}{12}{2019}


\begin{document}

\marginpar{ \textbf{Lecture 15.} \\  \displaydate{date}. \\ Compiled:  \today.}

\begin{equation}
  e^{-\beta \sum_{i,j>i}^{} \Phi _{ij}  } = \prod_{i}^{} \qty(\prod_{j>i}^{} \qty(1+f_{ij})   ) = 1 + \sum_{i,j>i}^{} f_{ij} + \sum_{\substack{i,j>i \\ k,l> k \\
  k \ge i \\ (i,j) \neq (k,l) } }^{} f_{ij} f_{kl}   + O (f^3)
\end{equation}
where
\begin{equation}
  f_{ij} \equiv  e^{-\beta \Phi _{ij}} -1
\end{equation}
The \( f_{ij} \)  will be small enough when \( T \) is very large or \( \Phi _{ij} \) is small enough because you are in small density. What is important it is the ration between \( \beta  \)  and \( \Phi _{ij} \).

In the case \( \Phi _{ij} \ll 1 \) you keep other terms contributions. In the other cases you can keep the linear term.

The partition function is:
\begin{equation}
  Q_N (V,T) = \int_{V}^{} \dd[]{\va{r}_1} \dots \dd[]{\va{r}_N} \qty(1+ \sum_{i,j>i}^{} f_{ij} + \dots ) = V^N + V^{N-2} \sum_{i,j>i}^{} \int_{}^{} \dd[]{\va{r}_i} \dd[]{\va{r}_j} f_{ij} + \dots
\end{equation}
We are summing up over all configurations \( ij \).
Let us try to compute the term in the integral:
\begin{equation}
  \int_{}^{} \dd[]{\va{r}_i} \dd[]{\va{r}_j} f_{ij} = \int_{}^{} \dd[]{\va{r}_i} \dd[]{\va{r}_j} f \qty(\abs{\va{r}_i - \va{r}_j} )
  = V \int_{V}^{} \dd[]{\va{r}}  f \qty(\abs{\va{r}} ) \equiv -2 b_2 V
\end{equation}
so, what is important it is the relative distance. \( \va{r} \) gives us the position from the center we have choosen.
\begin{equation}
  b_2 \equiv - \frac{1}{2} \int_{V}^{} \dd[]{\va{r}}  f \qty(\abs{\va{r}} )
\end{equation}
Rewrite again the partition function:
\begin{equation}
  Q_N (V,T) = V^N - V^{N-1} N (N-1) b_2
\end{equation}
\begin{equation}
  Z_N (V,T) = \qty(\frac{V^N}{N! \Lambda ^{3N}}) \qty(1- \frac{N^2}{V}b_2 + \dots)
\end{equation}
\begin{remark}
I do not care about the \( (N-1) \) term, because \( N \) is big enough!
\end{remark}
The free energy is:
\begin{equation}
  F_N = F_N^{il} - k_B T \ln{\qty[1-\frac{N^2}{V}b_2 + \dots] }
\end{equation}
\begin{equation}
  P_N = - \qty(\pdv{F_N}{V} )_{T,N} = \frac{N k_B T}{V} \qty(1+ \frac{\frac{N}{V}b_2}{1- \frac{N}{V}b_2})
  \approx \frac{N k_B T}{V} \qty(1+\frac{N}{V}b_2 + \dots)
\end{equation}
here you see the ideal gas and the correction to the ideal gas.

Therefore, it is important computing \( b_2 \), because one time you have this ypu have the expansion. Or if you wish, by doing the fit of data at different temperature you obtain \( b_2 \) from the experiment and you see \( f_{ij} \). You can use macroscopic to obtain information about the potential in the microscopic.

In principle, from the expansion I realized that for example in a generic expansion
\begin{equation}
  (1-x)^{-1} = 1 + x + \dots
\end{equation}
so, the our expansion is something like this.
\begin{equation}
  \frac{PV}{N k_B T} \approx 1 + \rho b_2 \simeq \frac{1}{1-b_2 \rho } \approx 1 + b_2 \rho + \underbrace{(b_2)^2 \rho ^2}_{b_3 \approx b_2^2}  + \underbrace{(b_2)^3 \rho ^3}_{b_4 \approx b_2^3}
\end{equation}
\begin{example}
Exam: let us compute virial expansion of a gas in a potential.
\end{example}
The simple one is the hardcore potential.
\subsubsection{Gas of hard spheres}
The particles are interacting (it is not ideal!) and there is a size that is the range of the potential.
\begin{equation}
  \Phi (r) = \begin{cases}
    \infty & r < \sigma \\
    0     & r \ge \sigma
\end{cases}
\end{equation}
(insert plot of that potential!)

Hence,
\begin{equation}
  e^{-\beta \Phi } = \begin{cases}
    0 & r < \sigma\\
    1 & r \ge \sigma
  \end{cases}
\end{equation}
\begin{equation}
  \Rightarrow b_2 = - \frac{1}{2} \int_{V}^{} \dd[]{\va{r}}  f \qty(\abs{\va{r}} )
  = - \frac{1}{2} 4 \pi \int_{}^{} \dd[]{r} r^2 \qty[e^{-\beta \Phi }-1 ]
  = 2 \pi \int_{0}^{\sigma } \dd[]{r} r^2 = \frac{2}{3} \pi  \sigma ^3
\end{equation}
There is no condensation in the gas spheres.
The pressure is increased because of therm.
It is very easy, but it is interesting for introducing attraction.
Let us say, that the potential is not anymore zero but is \( - \varepsilon  \) when it is in the case \( r \ge \sigma \). Or consider a case in which it is \( - \varepsilon  \) between \( [\sigma ,2 \sigma ] \), then it goes to zero. (Insert graphics of these last potentials!!!!).

We can consider a Lennard-Jones potential:
\begin{equation}
  \Phi = 4 \varepsilon \qty[\qty(\frac{\sigma }{r})^{12} - \qty(\frac{\sigma }{r})^6  ]
\end{equation}
(insert graphics of the Lennard-Jones potential!!!!). The minimum is in \( r_{min}=2^{1/\sigma } \). You can play with the change of attraction by changing \( \sigma  \) or by changin the   \( \varepsilon  \).
What it is important is that for the Lennard-Jones we have
\begin{equation}
  b_2 \overset{LJ}{=} b_2 (T)
\end{equation}

We have pairs \( ij \) or \( kl \) (insert picture).   For instance, \( 12 \) and \( 34 \):
\begin{equation}
  \int_{V}^{} \dd[]{\va{r}_1}  \dd[]{\va{r}_2}   \dd[]{\va{r}_3}   \dd[]{\va{r}_4} f_{12} f_{34}
  = V^2 \qty(  \int_{}^{} \dd[]{r} f(r)  )^2
\end{equation}
\begin{equation}
  \frac{N(N-1)}{2} \qty[\frac{(N-2)(N-3)}{2}] \frac{1}{2}
\end{equation}

If we have \( 123 \)(insert picture) (three particles involved) the integral is:
\begin{equation}
  \int_{}^{} \dd[]{\va{r}_1}  \dd[]{\va{r}_2}   \dd[]{\va{r}_3}  f_{12} f_{23}
  \simeq V  \qty(  \int_{}^{} \dd[]{r} f(r)  )^2
\end{equation}

In the case (insert picture) were \( 123 \) are all interacting we have
\begin{equation}
  \int_{}^{} \dd[]{\va{r}_1}  \dd[]{\va{r}_2}   \dd[]{\va{r}_3}  f_{12} f_{23} f_{31}
  \simeq V  \qty( b_3 - 2 b_2^2  )
\end{equation}








\end{document}
