\documentclass[11pt, a4paper, twoside, openright]{book}
\usepackage{subfiles}

% Algorithms
	%\usepackage{algpseudocode}
	%\usepackage{algorithm}

% Babel
\usepackage[english]{babel}

% Code writing
	%\usepackage[procnames]{listings}

% Font
\usepackage[utf8]{inputenc}
\usepackage[T1]{fontenc}
\usepackage{amssymb,amsmath,amsthm,amsfonts}
\usepackage{eucal}
\usepackage{textcomp}

% Footnote


% Hyperref
\usepackage[hyphens]{url}
\usepackage{cite}
\usepackage{hyperref}
\usepackage{nameref}
\usepackage{url}
% Images
\usepackage[pdftex]{graphicx}
	%\usepackage{subfigure}
\usepackage{subfig}
\usepackage{eso-pic}
\usepackage{caption}
\usepackage{wrapfig}

% List
\usepackage{enumerate}

% SI units
\usepackage{siunitx}

% Standalone
\usepackage[subpreambles=true]{standalone}
\usepackage{import}

% Tables
\usepackage{tabularx}
\usepackage{booktabs}
\usepackage{multirow}

% TiKz and graphs
\usepackage{pgf,tikz,pgfplots}
% \usepackage{gnuplottex}
\usepackage{bm}
\usepackage{relsize}
%\usepackage[compat=1.1.0]{tikz-feynman}
\usepackage{circuitikz}

% Typeset
%\usepackage[top=2cm,bottom=2cm,left=2cm,right=2cm]{geometry}
\usepackage[top=2cm,bottom=2cm,left=2cm,right=2cm]{geometry}
\usepackage{fancyhdr}
\usepackage{indentfirst}
\usepackage{titlesec}
\usepackage{setspace}
\usepackage{xspace}
% \usepackage{parskip}  % Elimina il separatore a inizio paragrafo
\usepackage{afterpage}
\usepackage{comment}

%Python
\usepackage{xcolor}
\usepackage{listings}
\usepackage{framed}

%Per scrivere matrice identità
\usepackage{bbold}
%Per semplificazione formule
\usepackage{cancel}

%Evidenziare formule
\usepackage{empheq}
	%oppure
	%\usepackage{xcolor}
\usepackage{soul}

%Evidenziare testo con mdframed
\usepackage{mdframed}

%Note a margine
\usepackage{marginnote}

%Display data
\usepackage{datetime}

%Physics
\usepackage{physics}


\captionsetup[table]{font=small,labelfont={bf},skip=10pt}
\captionsetup[figure]{font=small,labelfont={bf},skip=10pt}

%intestazione pagina
\pagestyle{fancy}
\fancyhf{}
\fancyhead[RE]{\ifnum\value{chapter}>0\nouppercase{\leftmark}\fi}
\fancyhead[LE]{\small\textbf{\thepage}}
\fancyhead[LO]{\nouppercase{\rightmark}}
\fancyhead[RO]{\small\textbf{\thepage}}

%link ipertestuale per indice
\hypersetup{
	colorlinks=false,
	linkcolor=black,
	filecolor=blue,
	citecolor = blue,      
	urlcolor=blue,
	}
	
%%%%%indent%%%
\setlength{\parindent}{15pt}
\setlength{\parskip}{0pt}


%boh
\renewcommand{\chaptermark}[1]{%
 \markboth{\MakeUppercase{%
 \chaptername}\ \thechapter.%
 \ #1}{}}
 
 
 %Python in latex
 \definecolor{codegreen}{rgb}{0,0.6,0}
\definecolor{codegray}{rgb}{0.5,0.5,0.5}
\definecolor{codepurple}{rgb}{0.58,0,0.82}
\definecolor{backcolour}{rgb}{0.95,0.95,0.92}
\definecolor{commentcolour}{rgb}{0.43,0.63,0.65}

\definecolor{shadecolor}{rgb}{0.93, 0.93, 0.93}
\definecolor{darkgreen}{rgb}{0.0, 0.5, 0.0}
\definecolor{darkred}{rgb}{0.8, 0.0, 0.0}
\definecolor{violet}{rgb}{0.55, 0.0, 0.55}

\lstdefinestyle{mystyle}{ %Stile python code
    backgroundcolor=\color{shadecolor},   
    commentstyle=\color{commentcolour},
    keywordstyle=\color{darkgreen},
    numberstyle=\tiny\color{codegray},
    stringstyle=\color{darkred},
    basicstyle=\ttfamily\footnotesize,
    breakatwhitespace=false,         
    breaklines=true,                 
    captionpos=b,                    
    keepspaces=true,                 
    numbers=left,                    
    numbersep=5pt,                  
    showspaces=false,                
    showstringspaces=false,
    showtabs=false,                  
    tabsize=2
}
 
\lstset{style=mystyle}
% Derivatives
\renewcommand{\d}[0]{\mathrm{d}}
\newcommand{\dev}[2]{\displaystyle \frac{\d #1}{\d #2}}
\newcommand{\pdev}[2]{\displaystyle \frac{\partial #1}{\partial #2}}
\newcommand{\ndev}[3]{\displaystyle \frac{\d^{#3} #1}{\d #2^{#3} } }
\newcommand{\npdev}[3]{\displaystyle \frac{\partial^{#3} #1}{\partial #2^{#3} } }


%% Norms
\newcommand{\absvec}[1]{| \vec{#1} |}
\newcommand{\normvec}[1]{|\!| \vec{#1} |\!|}

\newcommand{\vmed}[1]{\left \langle #1 \right \rangle}
\newcommand{\vmedvec}[1]{\langle #1 \rangle}
\newcommand{\R}[0]{\mathbb{R}}
\renewcommand{\H}[0]{\operatorname{H}}

%Evidenziare formule
\newcommand{\mathcolorbox}[2]{\colorbox{#1}{$\displaystyle #2$}}
\newcommand{\hlfancy}[2]{\sethlcolor{#1}\hl{#2}}

%Theorem
%\newtheorem{theorem}{Theorem}[section]
%\newtheorem{corollary}{Corollary}[theorem]
%\newtheorem{lemma}[theorem]{Lemma}

%%%%%%%%%%%%%%%%%%%%Theorem, Corollary, Lemma, Proposition%%%%%%%%%%%%%%%%%
\usepackage[many,most,theorems]{tcolorbox}

\newtcbtheorem{theorem}{Theorem}{ % frame stuff
    boxrule = 1pt,
    breakable,
    enhanced,
    frame empty,
    interior style= {orange!20},
    %interior empty,
    colframe=black,
    borderline ={1pt}{0pt}{black},
    left=0.2cm,
    % title stuff
    attach boxed title to top left={yshift=-2mm,xshift=0mm},
    coltitle=black,
    fonttitle=\bfseries,
    colbacktitle=white,
    fontupper=\slshape,
    boxed title style={boxrule=1pt,sharp corners}}{theorem} 

\newtcbtheorem{corollary}{Corollary}{ % frame stuff
    boxrule = 1pt,
    breakable,
    enhanced,
    frame empty,
    interior style= {orange!20},
    %interior empty,
    colframe=black,
    borderline ={1pt}{0pt}{black},
    left=0.2cm,
    % title stuff
    attach boxed title to top left={yshift=-2mm,xshift=0mm},
    coltitle=black,
    fonttitle=\bfseries,
    colbacktitle=white,
    fontupper=\slshape,
    boxed title style={boxrule=1pt,sharp corners}}{corollary} 
    
\newtcbtheorem{lemma}{Lemma}{ % frame stuff
    boxrule = 1pt,
    breakable,
    enhanced,
    frame empty,
    interior style= {orange!20},
    %interior empty,
    colframe=black,
    borderline ={1pt}{0pt}{black},
    left=0.2cm,
    % title stuff
    attach boxed title to top left={yshift=-2mm,xshift=0mm},
    coltitle=black,
    fonttitle=\bfseries,
    colbacktitle=white,
    fontupper=\slshape,
    boxed title style={boxrule=1pt,sharp corners}}{lemma} 


%%%%%%%%%%%%%%%%%%%%Definition%%%%%%%%%%%%%%%%%


\newtcbtheorem{definition}{Definition}{ % frame stuff
    boxrule = 1pt,
    breakable,
    enhanced,
    frame empty,
    interior style= {blue!10},
    %interior empty,
    colframe=black,
    borderline ={1pt}{0pt}{black},
    left=0.2cm,
    % title stuff
    attach boxed title to top left={yshift=-2mm,xshift=0mm},
    coltitle=black,
    fonttitle=\bfseries,
    colbacktitle=white,
    boxed title style={boxrule=1pt,sharp corners}}{definition} 


%\theoremstyle{definition}
%\newtheorem{definition}{Definition}%[section]



%%%%%%%%%%%%%%%%%%%%Exercise and example%%%%%%%%%%%%%%%%%

\newtcbtheorem{exercise}{Exercise}{ % frame stuff
    boxrule = 1pt,
    breakable,
    enhanced,
    frame empty,
    interior style= {blue!6},
    %interior empty,
    colframe=black,
    borderline ={1pt}{0pt}{black},
    left=0.2cm,
    % title stuff
    attach boxed title to top left={yshift=-2mm,xshift=0mm},
    coltitle=black,
    fonttitle=\bfseries,
    colbacktitle=white,
    boxed title style={boxrule=1pt,sharp corners}}{exercise} 

\newtcbtheorem{example}{Example}{ % frame stuff
    boxrule = 1pt,
    enhanced,
    frame empty,
    interior style= {green!6},%{left color=yellow!70,right color=green!70},
    %interior empty,
    colframe=black,
    borderline ={1pt}{0pt}{black},
    breakable,
    left=0.2cm,
    % title stuff
    attach boxed title to top left={yshift=-2mm,xshift=0mm},
    coltitle=black,
    fonttitle=\bfseries,
    colbacktitle=white,
    boxed title style={boxrule=1pt,sharp corners}}{example}
  
%\newtheorem{exercise}{Exercise}
%\newtheorem{example}{Example}

%%%%%%%%%%%%%%%%%%%%%%%%%%%%%%%%%%%

\theoremstyle{remark}
\newtheorem*{remark}{Remark}
\newtheorem{observation}{Observation}
%Evidenziare testo
\newtheorem*{solution}{Solution}

\newcommand\mybox[1]{%
  \fbox{\begin{minipage}{0.9\textwidth}#1\end{minipage}}}

  %Spiegazioni/verifiche
\newenvironment{greenbox}{\begin{mdframed}[hidealllines=true,backgroundcolor=green!20,innerleftmargin=3pt,innerrightmargin=3pt,innertopmargin=3pt,innerbottommargin=3pt]}{\end{mdframed}}

\newenvironment{bluebox}{\begin{mdframed}[hidealllines=true,backgroundcolor=blue!10,innerleftmargin=3pt,innerrightmargin=3pt,innertopmargin=3pt,innerbottommargin=3pt]}{\end{mdframed}}

\newenvironment{yellowbox}{\begin{mdframed}[hidealllines=true,backgroundcolor=yellow!20,innerleftmargin=3pt,innerrightmargin=3pt,innertopmargin=3pt,innerbottommargin=3pt]}{\end{mdframed}}

\newenvironment{redbox}{\begin{mdframed}[hidealllines=true,backgroundcolor=red!20,innerleftmargin=3pt,innerrightmargin=3pt,innertopmargin=3pt,innerbottommargin=3pt]}{\end{mdframed}}

\newenvironment{orangebox}{\begin{mdframed}[hidealllines=true,backgroundcolor=orange!20,innerleftmargin=3pt,innerrightmargin=3pt,innertopmargin=3pt,innerbottommargin=3pt]}{\end{mdframed}}

%emph equation
\newcommand*\myyellowbox[1]{%
  \colorbox{yellow!40}{\hspace{1em}#1\hspace{1em}}}

\newcommand*\mygreenbox[1]{%
  \colorbox{green!20}{\hspace{1em}#1\hspace{1em}}}
  
  
  
  
  

%Ti voglio un mondo di bene <3






\begin{document}

%%%%%%FRONTESPIZIO%%%%%%

\frontmatter
%%%%%%FRONTESPIZIO%%%%%%
\begin{titlepage} % Suppresses headers and footers on the title page

	\centering % Centre everything on the title page
	
	\scshape % Use small caps for all text on the title page
	
	\vspace*{\baselineskip} % White space at the top of the page
	
	%------------------------------------------------
	%	Title
	%------------------------------------------------
	
	\rule{\textwidth}{1.6pt}\vspace*{-\baselineskip}\vspace*{2pt} % Thick horizontal rule
	\rule{\textwidth}{0.4pt} % Thin horizontal rule
	
	\vspace{0.75\baselineskip} % Whitespace above the title
	
	{\LARGE LECTURE NOTES\\ OF\\ STATISTICAL MECHANICS\\} % Title
	
	\vspace{0.75\baselineskip} % Whitespace below the title
	
	\rule{\textwidth}{0.4pt}\vspace*{-\baselineskip}\vspace{3.2pt} % Thin horizontal rule
	\rule{\textwidth}{1.6pt} % Thick horizontal rule
	
	\vspace{2\baselineskip} % Whitespace after the title block
	
	%------------------------------------------------
	%	Subtitle
	%------------------------------------------------
	
	Collection of the lectures notes of professor Orlandini. % Subtitle or further description
	
	\vspace*{3\baselineskip} % Whitespace under the subtitle
	
	%------------------------------------------------
	%	Editor(s)
	%------------------------------------------------
	
	Edited By
	
	\vspace{0.5\baselineskip} % Whitespace before the editors
	
	{\scshape\Large Alice Pagano \\} % Editor list
	
	\vspace{0.5\baselineskip} % Whitespace below the editor list
	
	\textit{The University of Padua } % Editor affiliation
	
	\vfill % Whitespace between editor names and publisher logo
	
%	%------------------------------------------------
%	%	Publisher
%	%------------------------------------------------
%	
%	\plogo % Publisher logo
%	
%	\vspace{0.3\baselineskip} % Whitespace under the publisher logo
%	
%	2017 % Publication year
%	
%	{\large publisher} % Publisher

\end{titlepage}

\clearpage{\pagestyle{empty}\cleardoublepage}


\tableofcontents


\pagestyle{plain}

\chapter{Introduction}

The goal of statistical mechanics \cite{introduction} is to predict the macroscopic properties of bodies, most especially their thermodynamic properties, on the basis of their microscopic structure.

The macroscopic properties of greatest interest to statistical mechanics are those relating to thermodynamic equilibrium. As a consequence, the concept of thermodynamic equilibrium occupies a central position in the field. 

The microscopic structure of systems examined by statistical mechanics can be described by means of mechanical models: for example, gases can be represented as systems of particles that interact by means of a phenomenologically determined potential. Other examples of mechanical models are those that represent polymers as a chain of interconnected particles, or the classical model of crystalline systems, in which particles are arranged in space according to a regular pattern, and oscillate around the minimum of the potential energy due to their mutual interaction. The models to be examined can be, and recently increasingly are, more abstract, however, and exhibit only a faint resemblance to the basic mechanical description (more specifically, to the quantum nature of matter). The explanation of the success of such abstract models is itself the topic of one of the more interesting chapters of statistical mechanics: the theory of universality and its foundation in the renormalization group.

The models of systems dealt with by statistical mechanics have some common characteristics. We are in any case dealing with systems with a large number of degrees of freedom:  the reason lies in the corpuscular (atomic) nature of matter.
 The degrees of freedom that one considers should have more or less comparable effects on the global behavior of the system.  This state of affairs excludes the application of the methods of statistical mechanics to cases in which a restricted number of degrees of freedom “dominates” the others—for example, in celestial mechanics, although the number of degrees of freedom of the planetary system is immense, an approximation in which each planet is considered as a particle is a good start. In this case, we can state that the translational degrees of freedom (three per planet)—possibly with the addition of the rotational degrees of freedom, also a finite number—dominate all others.
 These considerations also make attempts to apply statistical concepts to the human sciences problematic because, for instance, it is clear that, even if the behavior of a nation’s political system includes a very high number of degrees of freedom, it is possible to identify some degrees of freedom that are disproportionately important compared to the rest.
  On the other hand, statistical methods can also be applied to systems that are not strictly speaking mechanical—for example, neural networks (understood as models of the brain’s components), urban thoroughfares (traffic models), or problems of a geometric nature (percolation).
 
 The simplest statistical mechanical model is that of a large number of identical particles, free of mutual interaction, inside a container with impenetrable and perfectly elastic walls. This is the model of the ideal gas, which describes the behavior of real gases quite well at low densities, and more specifically allows one to derive the well-known equation of state.

The introduction of pair interactions between the particles of the ideal gas allows us to obtain the standard model for simple fluids. Generally speaking, this model cannot be resolved exactly and is studied by means of perturbation or numerical techniques. It allows one to describe the behavior of real gases (especially noble gases), and the liquid–vapor transition (boiling and condensation).

The preceding models are of a classical (nonquantum) nature and can be applied only when the temperatures are not too low. The quantum effects that follow from the inability to distinguish particles are very important for phenomenology, and they can be dealt with at the introductory level if one omits interactions between particles. 

In many of the statistical models we will describe, however, the system’s fundamental elements will not be “particles,” and the fundamental degrees of freedom will not be mechanical (position and velocity or impulse). If we want to understand the origin of ferromagnetism, for example, we should isolate only those degrees of freedom that are relevant to the phenomenon being examined (the orientation of the electrons’ magnetic moment) from all those that are otherwise pertinent to the material in question.

The simplest case is that in which there are only two values—in this fashion, we obtain a simple model of ferromagnetism, known as the Ising model, which is by far the most studied model in statistical mechanics. The ferromagnetic solid is therefore represented as a regular lattice in space, each point of which is associated with a degree of freedom, called spin, which can assume the values +1 and -1. This model allows one to describe the paramagnet-ferromagnet transition, as well as other similar transitions.

In this course, classical statistical mechanics of system at equilibrium is treated. 
The exam is divided into two parts: first, common oral exam (same exercise and question for everyone, it is a written part), second part, oral.

\clearpage
Outline of the course:

\begin{enumerate}
\item Brief recap of thermodinamics.
\item Equilibrium phases and thermodynamics of the phase transitions.
\item Statistical mechanics and theory of ensambles.
\item Thermodinamic limit and phase transitions in statistical mechanics.
\item Order parameter and critical point.
\item The role of modelling in the physics of phase transitions.
\item The Ising model.
\item Exact solutions of the Ising model.
\item Transfer matrix method.
\item Role of dimension and range of interactions in critical phenomena (lower critical dimension).
\item Approximations: \emph{Meanfield theory Weiss} and \emph{variational mean field}.
\item Landau theory of phase transitions: the role of symmetries.
\item Relevance of fluctuations: the \emph{Geinzburg criterium} and the notion of the \emph{upper critical dimension}.
\item The Ginzburg-Landau model.
\item Landau theory for non-homogeneous system. The \( \nu  \) exponent.
\item Gaussian fluctuations in the G-L theory.
\item Widom's scaling theory.
\item Kadauoff's theory of scaling.
\item The theory of renormalisations group and the origin of \emph{universality} in critical phenomena.
\item Spontaneous symmetry breaking.

\end{enumerate}






\mainmatter
\pagestyle{fancy}


\subfile{../lessons/1_09-10-2019.tex}
\subfile{../lessons/2_11-10-2019.tex}
\subfile{../lessons/3_16-10-2019.tex}
\subfile{../lessons/4_18-10-2019.tex}
\subfile{../lessons/5_23-10-2019.tex}

\subfile{../lessons/6_25-10-2019.tex}
\subfile{../lessons/7_30-10-2019.tex}
\subfile{../lessons/8_06-11-2019.tex}
\subfile{../lessons/9_08-11-2019.tex}
\subfile{../lessons/10_13-11-2019.tex}
\subfile{../lessons/11_20-11-2019.tex}
\subfile{../lessons/12_22-11-2019.tex}
\subfile{../lessons/13_27-11-2019.tex}
\subfile{../lessons/14_29-11-2019.tex}
\subfile{../lessons/15_04-12-2019.tex}
\subfile{../lessons/16_06-12-2019.tex}
\subfile{../lessons/17_11-12-2019.tex}
\subfile{../lessons/18_13-12-2019.tex}
\backmatter
\pagestyle{plain}

\chapter{Conclusions}

%%%BIBLIOGRAFIA%%%

\cleardoublepage
\addcontentsline{toc}{chapter}{\bibname}
\begin{thebibliography}{99}


\bibitem{introduction} 
Luca Peliti
\textit{Statistical Mechanics in a Nutshell}.

\bibitem{3_lesson_1} 
Wikipedia 
\url{https://en.wikipedia.org/wiki/Lever_rule}.

\bibitem{3_lesson_2} 
Herbert B. Callen
\textit{Thermodynamics and an introduction to thermostatistics, second edition}.

\bibitem{3_lesson_3} 
J.M.Yeomans
\textit{Statistical Mechanics of Phase Transitions}.

\bibitem{9_lesson_1} 
L. Onsanger.
\textit{Phy.Rev.65, 117 (1944)}. 

\bibitem{9_lesson_2} 
T.D Schultz et al.
\textit{Rev.Mad.Phys.36, 856 (1964)}. 

\bibitem{9_lesson_3} 
R.J. Baxter and I.G. Enting.
\textit{J. Phys. A 35, 5189, (1978)}. 


\bibitem{10_lesson_1} 
Griffiths
\textit{Phy.Rev 65, 117(1944)}. 

 

%\bibitem{Divincenzo} 
%David P. DiVincenzo
%\textit{The Physical Implementation of Quantum Computation}.
%IBM T.J. Watson Research Center, Yorktown Heights, NY 10598 USA
%February 1, 2008.
%arXiv:quant-ph/0002077
%
%\bibitem{DocumentationQiskit} 
%The Qiskit Developers.
%\textit{Qiskit API documentation}.
%Release 0.8.0, 9 March 2019.
%\url{https://qiskit.org/documentation/index.html}

\end{thebibliography}

\end{document}
